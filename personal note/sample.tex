\documentclass[11pt, oneside]{article}
\usepackage{amssymb, amsmath, enumitem, mathtools}
\usepackage{geometry}
\geometry{
    letterpaper,
    top=1.in,
    bottom=1.1in,
    left=.9in,
    right=.9in,
    headheight=15pt,
    footskip=40pt
}
\setlength{\columnsep}{2em}
\usepackage[english]{babel}
\usepackage[utf8]{inputenc}
\usepackage{textpos,siunitx,listings}
\usepackage{bm, csquotes, comment}
\usepackage{tensor, braket}
\usepackage{graphicx, xcolor}
\usepackage{setspace}
% \usepackage{booktabs, bookmark}
% \usepackage{unicode-math}

\usepackage[colorlinks]{hyperref}
\hypersetup{
    linkcolor=blue,
    citecolor=blue,
    filecolor=magenta,
    urlcolor=teal
}

\usepackage{refcount}
% \numberwithin{equation}{section}

\setlist[itemize]{
    itemsep=-0.1em,
    label={$\boldsymbol{\cdot}$},
    leftmargin=1.5em,
    labelsep=.7em,
    topsep=0pt
}

\usepackage{natbib}
% \setcitestyle{author,open={(},close={)}}

% \newcommand{\pardev}[2]{\frac{\partial #1}{\partial #2}}
% \renewcommand{\div}{\nabla\cdot}
% \newcommand{\christoffel}[3]{\Gamma^{#1}_{#2#3}}
\newcommand{\cf}{\noindent\textsf{cf) }}
\newcommand{\solarmass}{M_\odot}

% \usepackage{palatino}
% \usepackage{tgtermes}
% \usepackage{tgheros}
% \usepackage{charter}
\usepackage{times}
% \usepackage{fourier}
% \usepackage{utopia}

\usepackage{helvet}

% ---------------------------------------------------------------------------
%
%           FORMATTING
% 
% ---------------------------------------------------------------------------
\usepackage{titling}
% \pretitle{\begin{flushleft}\LARGE} % makes document title flush right
% \posttitle{\end{flushleft}}
% \preauthor{\begin{flushleft}\large} % makes author flush right
% \postauthor{\end{flushleft}}
% \predate{\begin{flushright}\large} % makes date title flush right
% \postdate{\end{flushright}}

% -------- toc formatting
\usepackage{tocloft}
% \cftsetindents{section}{0em}{2em}
% \cftsetindents{subsection}{1em}{3em}
% \makeatletter
% \renewcommand{\l@subsection}{\@dottedtocline{1}{4.0em}{3.6em}}
% \renewcommand{\l@subsection}{\@dottedtocline{2}{4.0em}{3.6em}}
% \renewcommand{\l@subsubsection}{\@dottedtocline{3}{7.4em}{4.5em}}
% \makeatother

% -------- indentation
% \usepackage{indentfirst}
\setlength{\parindent}{0pt}
% \usepackage{parskip}
\setlength{\parskip}{.2em}

\setlength{\footnotesep}{.5em}
% \setstretch{1.3}
% \renewcommand{\baselinestretch}{1.0}

% -------- section titles
\usepackage{sectsty, titlecaps}
\sectionfont{\vspace{2em}\fontsize{12}{12}\centering\bfseries}
\subsectionfont{\vspace{1em}\fontsize{11}{10}\normalfont\MakeUppercase}
\subsubsectionfont{\vspace{.0em}\fontsize{10}{10}\normalfont\em}

\usepackage{titlesec}
% \titlelabel{\thetitle.\hspace{1em}}
\renewcommand\thesection{\Roman{section}}
\renewcommand\thesubsection{\arabic{subsection}}
% \renewcommand\thesubsubsection{\thesubsection.\arabic{subsubsection}}

% \usepackage[style=nature, sorting=ynt]{biblatex}
% \addbibresource{references.bib}
% \DeclareFieldFormat{labelnumberwidth}{}
% \setlength{\biblabelsep}{0pt}

% ---------------------------------------------------------------------------
%
%           DOCUMENT INFO
% 
% ---------------------------------------------------------------------------
\title{
%     \vspace{-3em}
    {\bfseries template - personal notes}
    % \vspace{-.5em}
}

\author{
    Yoonsoo Kim
    \\[1em]
    Last revision : \today
%     \vspace{.5em}
%     \hrule
    \vspace{-5em}
}
\date{}

% -------- header 
\usepackage{fancyhdr}
% \pagestyle{fancy}
% \fancyhf{}
% \lhead{BHL - literature review}
\rhead{}
% \cfoot{\thepage}

\begin{document}

\maketitle

% \setstretch{1.2}
\setcounter{tocdepth}{2}
\tableofcontents
% \setstretch{1.0}
% \vspace{1em}
% \hrule
% \vspace{1em}

\newpage

% ---------------------------------------------------------------------------
% 
%                    Section
% 
% ---------------------------------------------------------------------------
\section{Stellar evolution remnant}


\begin{description}[style=multiline, leftmargin=8em,
    labelindent=0em, font=\textbf]
\item[Mass prescriptions] \begin{itemize}
    \item WD : $<8\solarmass$
    \item NS : $8-20\solarmass$
    \item BH : not well constrained?
\end{itemize}
\end{description}


\begin{description}[style=multiline, leftmargin=8em,
    labelindent=0em, font=\textbf]
\item[Natal kicks] \begin{itemize}
     \item BH : not well constrained
     \item NS : Maxwellian, $\sigma \sim 270\si[]{\kilo\meter\per\second}$?
     (see \cite{Hobbs2005})
\end{itemize}
\end{description}


% ---------------------------------------------------------------------------
% 
%                    Section
% 
% ---------------------------------------------------------------------------
\section{Neutron stars}

$M\sim 1.4\solarmass$, $R\sim 10\si[]{\km}$.

Highest mass observed : $2.14^{+0.10}_{-0.09}\solarmass$ \citep{NANOGrav2019}

In general, softer equation of state yields more compact stars, small radii,
relatively small max mass. See \cite{Ozel2016} for a review.

absolute upper limit to the NS spin frequency 
- mass shedding limit - constraint on the EOS 
about 1000Hz

the highest observed spin rate  : 716Hz \citep{Hessels2006}


% --------------------------------------------------- %
%               subsection
% --------------------------------------------------- %
\subsection{Formation}

Progenitor : Type II SNe ($ > 8 \solarmass$). Neutrinos play a crucial role in
the collapse. Interior of proto-NS lose energy rapidly by neutrino emission.

Gravitational binding energy released during collapse is about 
\begin{equation}
    \frac{3}{5}\frac{GM^2}{R} \sim 0.1 Mc^2
    \, \sim 10^{53}\text{erg}
\end{equation}

Kinetic energy of the expanding SN remnant is about $\sim 10^{51}\text{erg}$;
nearly all the energy is carried off by neutrinos and anti-neutrinos.


% --------------------------------------------------- %
%               subsection
% --------------------------------------------------- %
\subsection{State of matter}

Typical Fermi energies of a NS are of the order MeV $\sim
10^{10}\si[]{\kelvin}$, therefore the temperature becomes only important when it
is very high; otherwise $T=0$ (cold matter) is a good approximation.

There is no `pure' neutron stars\footnote{Free neutron lifetime is about 15
minutes \citep{UCNtau2021}.} - an actual NS contains a mixture of $n$, $p$,
and $e$.

$\beta$-decay and inverse $\beta$-decay : 
\begin{align}
    \label{eq:beta decay}
    n &\quad\rightarrow\quad p + e + \bar{\nu}_e \\
    \label{eq:inverse beta decay} p + e & \quad\rightarrow\quad n + \nu_e
\end{align}

Fermi energy level rises more quickly for electrons, so the neutron decay
\eqref{eq:beta decay} becomes energetically less favorable around a critical
density $\sim 10^{7}\si[]{\gram\per\cubic\cm}$ (Pauli blocking) and as the 
density more electrons are captured into neutrons: beta equilibrium. This
results in the formation of Coulomb crystals of progressively more neutron-rich
nuclei \footnote{Laboratory nuclear matter has nearly equal number of neutrons
and protons.}.

At matter densities a few times the $n_0$, the nucleon Fermi energies becomes so
large that it is energetically favorable to transform into heavier baryons such
as hyperons via electroweak interactions.




% ---------------------------------------------------------------------------
% 
%                    Section
% 
% ---------------------------------------------------------------------------
\section{Astrophysical black holes}

% --------------------------------------------------- %
%               subsection
% --------------------------------------------------- %
\subsection{How to observe}

\begin{itemize}
    \item X-ray binaries \\
        LMXB : \\
        HMXB : 
    \item Microlensing e.g. OGLE
    \item LIGO-VIRGO observations
\end{itemize}

% --------------------------------------------------- %
%               subsection
% --------------------------------------------------- %
\subsection{Mass}

stellar mass BHs : stellar evolution from massive MS star with $8-20\solarmass$;
IMBH : $10^2 - 10^4 \solarmass$, extremely rare, GW190521;
SMBH : $\sim 10^8 \solarmass$;





% ---------------------------------------------------------------------------
\setlength{\bibsep}{1ex}

% \nocite{*}
% \printbibliography[heading=none]

% \vspace{1em}
% \newpage
\bibliography{references}
% \nocite{*}
% \bibliographystyle{apalike}
% \bibliographystyle{abbrvnat}
% \bibliographystyle{unsrtnat}
% \bibliographystyle{plainnat}
% \bibliographystyle{../aasjournal}
% \bibliographystyle{../apsrmp4-2}
% \bibliographystyle{../apsrev4-2}
\bibliographystyle{../mnras}

% ---------------------------------------------------------------------------
\end{document}
