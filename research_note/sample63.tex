
%% Beginning of file 'sample63.tex'
%%
%% Modified 2019 June
%%
%% This is a sample manuscript marked up using the
%% AASTeX v6.3 LaTeX 2e macros.
%%
%% AASTeX is now based on Alexey Vikhlinin's emulateapj.cls 
%% (Copyright 2000-2015).  See the classfile for details.

%% AASTeX requires revtex4-1.cls (http://publish.aps.org/revtex4/) and
%% other external packages (latexsym, graphicx, amssymb, longtable, and epsf).
%% All of these external packages should already be present in the modern TeX 
%% distributions.  If not they can also be obtained at www.ctan.org.

%% The first piece of markup in an AASTeX v6.x document is the \documentclass
%% command. LaTeX will ignore any data that comes before this command. The 
%% documentclass can take an optional argument to modify the output style.
%% The command below calls the preprint style which will produce a tightly 
%% typeset, one-column, single-spaced document.  It is the default and thus
%% does not need to be explicitly stated.
%%
%%
%% using aastex version 6.3
\documentclass[preprint]{aastex63}
\usepackage[T1]{fontenc}
\usepackage{amsmath}
%% The default is a single spaced, 10 point font, single spaced article.
%% There are 5 other style options available via an optional argument. They
%% can be invoked like this:
%%
%% \documentclass[arguments]{aastex63}
%% 
%% where the layout options are:
%%
%%  twocolumn   : two text columns, 10 point font, single spaced article.
%%                This is the most compact and represent the final published
%%                derived PDF copy of the accepted manuscript from the publisher
%%  manuscript  : one text column, 12 point font, double spaced article.
%%  preprint    : one text column, 12 point font, single spaced article.  
%%  preprint2   : two text columns, 12 point font, single spaced article.
%%  modern      : a stylish, single text column, 12 point font, article with
%% 		  wider left and right margins. This uses the Daniel
%% 		  Foreman-Mackey and David Hogg design.
%%  RNAAS       : Preferred style for Research Notes which are by design 
%%                lacking an abstract and brief. DO NOT use \begin{abstract}
%%                and \end{abstract} with this style.
%%
%% Note that you can submit to the AAS Journals in any of these 6 styles.
%%
%% There are other optional arguments one can invoke to allow other stylistic
%% actions. The available options are:
%%
%%   astrosymb    : Loads Astrosymb font and define \astrocommands. 
%%   tighten      : Makes baselineskip slightly smaller, only works with 
%%                  the twocolumn substyle.
%%   times        : uses times font instead of the default
%%   linenumbers  : turn on lineno package.
%%   trackchanges : required to see the revision mark up and print its output
%%   longauthor   : Do not use the more compressed footnote style (default) for 
%%                  the author/collaboration/affiliations. Instead print all
%%                  affiliation information after each name. Creates a much 
%%                  longer author list but may be desirable for short 
%%                  author papers.
%% twocolappendix : make 2 column appendix.
%%   anonymous    : Do not show the authors, affiliations and acknowledgments 
%%                  for dual anonymous review.
%%
%% these can be used in any combination, e.g.
%%
%% \documentclass[twocolumn,linenumbers,trackchanges]{aastex63}
%%
%% AASTeX v6.* now includes \hyperref support. While we have built in specific
%% defaults into the classfile you can manually override them with the
%% \hypersetup command. For example,
%%
%% \hypersetup{linkcolor=red,citecolor=green,filecolor=cyan,urlcolor=magenta}
%%
%% will change the color of the internal links to red, the links to the
%% bibliography to green, the file links to cyan, and the external links to
%% magenta. Additional information on \hyperref options can be found here:
%% https://www.tug.org/applications/hyperref/manual.html#x1-40003
%%
%% Note that in v6.3 "bookmarks" has been changed to "true" in hyperref
%% to improve the accessibility of the compiled pdf file.
%%
%% If you want to create your own macros, you can do so
%% using \newcommand. Your macros should appear before
%% the \begin{document} command.
%%
\newcommand{\vdag}{(v)^\dagger}
\newcommand\aastex{AAS\TeX}
\newcommand\latex{La\TeX}

%% Reintroduced the \received and \accepted commands from AASTeX v5.2
% \received{June 1, 2019}
\revised{January 10, 2019}
% \accepted{\today}

%% The following command can be used to set the latex table counters.  It
%% is needed in this document because it uses a mix of latex tabular and
%% AASTeX deluxetables.  In general it should not be needed.
%\setcounter{table}{1}

%%%%%%%%%%%%%%%%%%%%%%%%%%%%%%%%%%%%%%%%%%%%%%%%%%%%%%%%%%%%%%%%%%%%%%%%%%%%%%%%
%%
%% The following section outlines numerous optional output that
%% can be displayed in the front matter or as running meta-data.
%%
%% If you wish, you may supply running head information, although
%% this information may be modified by the editorial offices.
% \shorttitle{Sample article}
% \shortauthors{Schwarz et al.}
%%
%% You can add a light gray and diagonal water-mark to the first page 
%% with this command:
%% \watermark{text}
%% where "text", e.g. DRAFT, is the text to appear.  If the text is 
%% long you can control the water-mark size with:
%% \setwatermarkfontsize{dimension}
%% where dimension is any recognized LaTeX dimension, e.g. pt, in, etc.
%%
%%%%%%%%%%%%%%%%%%%%%%%%%%%%%%%%%%%%%%%%%%%%%%%%%%%%%%%%%%%%%%%%%%%%%%%%%%%%%%%%

%% This is the end of the preamble.  Indicate the beginning of the
%% manuscript itself with \begin{document}.

\begin{document}

\title{Here goes the title}

\author{Yoonsoo Kim}
\noaffiliation
% \affiliation{American Astronomical Society \\
% 1667 K Street NW, Suite 800 \\
% Washington, DC 20006, USA}

% \begin{abstract}
% \end{abstract}

% \keywords{editorials, notices --- 
% miscellaneous --- catalogs --- surveys}

\section{Introduction} \label{sec:intro}

\latex\ was written in 1985 by Leslie Lamport who based it on the \TeX\ typesetting language
which itself was created by Donald E. Knuth in 1978.  In 1988 a suite of
\latex\ macros were developed to investigate electronic submission and
publication of AAS Journal articles \citep{1989BAAS...21..780H}.  Shortly
afterwards, Chris Biemesdefer merged these macros and more into a \latex\
2.08 style file called \aastex.  These early \aastex\ versions introduced
many common commands and practices that authors take for granted today.
Substantial revisions
were made by Lee Brotzman and Pierre Landau when the package was updated to
v4.0.  AASTeX v5.0, written in 1995 by Arthur Ogawa, upgraded to \latex\ 2e
which uses the document class in lieu of a style file.  Other improvements
to version 5 included hypertext support, landscape deluxetables and
improved figure support to facilitate electronic submission.  
\aastex\ v5.2 was released in 2005 and introduced additional graphics
support plus new mark up to identifier astronomical objects, datasets and
facilities.


\section{Floats} \label{sec:floats}

Floats are non-text items that generally can not be split over a page.
They also have captions and can be numbered for reference.  Primarily these
are figures and tables but authors can define their own. \latex\ tries to
place a float where indicated in the manuscript but will move it later if
there is not enough room at that location, hence the term ``float''.

Authors are encouraged to embed their tables and figures within the text as
they are mentioned.  Please do not place the figures and text at the end of
the article as was the old practice.  Editors and the vast majority of
referees find it much easier to read a manuscript with embedded figures and
tables.

Depending on the number of floats and the particular amount of text and
equations present in a manuscript the ultimate location of any specific
float can be hard to predict prior to compilation. It is recommended that
authors textbf{not} spend significant time trying to get float placement
perfect for peer review.  The AAS Journal's publisher has sophisticated
typesetting software that will produce the optimal layout during
production.

Note that authors of Research Notes are only allowed one float, either one
table or one figure.

For authors that do want to take the time to optimize the locations of
their floats there are some techniques that can be used.  The simplest
solution is to placing a float earlier in the text to get the position
right but this option will break down if the manuscript is altered.
A better method is to force \latex\ to place a
float in a general area with the use of the optional {\tt\string [placement
specifier]} parameter for figures and tables. This parameter goes after
{\tt\string \begin\{figure\}}, {\tt\string \begin\{table\}}, and
{\tt\string \begin\{deluxetable\}}.  The main arguments the specifier takes
are ``h'', ``t'', ``b'', and ``!''.  These tell \latex\ to place the float
\underline{h}ere (or as close as possible to this location as possible), at
the \underline{t}op of the page, and at the \underline{b}ottom of the page.
The last argument, ``!'', tells \latex\ to override its internal method of
calculating the float position.  A sequence of rules can be created by
using multiple arguments.  For example, {\tt\string \begin\{figure\}[htb!]}
tells \latex\ to try the current location first, then the top of the page
and finally the bottom of the page without regard to what it thinks the
proper position should be.  Many of the tables and figures in this article
use a placement specifier to set their positions.

Note that the \latex\ {\tt\string tabular} environment is not a float.  Only
when a {\tt\string tabular} is surrounded by {\tt\string\begin\{table\}} ...
{\tt\string\end\{table\}} is it a true float and the rules and suggestions
above apply.

In AASTeX v6.3 all deluxetables are float tables and thus if they are
longer than a page will spill off the bottom. Long deluxetables should
begin with the {\tt\string\startlongtable} command. This initiates a
longtable environment.  Authors might have to use {\tt\string\clearpage} to
isolate a long table or optimally place it within the surrounding text.

\subsection{Tables} \label{subsec:tables}

Tables can be constructed with \latex's standard table environment or the
\aastex's deluxetable environment. The deluxetable construct handles long
tables better but has a larger overhead due to the greater amount of
defined mark up used set up and manipulate the table structure.  The choice
of which to use is up to the author.  Examples of both environments are
used in this manuscript. 

Tables longer than 200 data lines and complex tables should only have a
short example table with the full data set available in the machine
readable format.  The machine readable table will be available in the HTML
version of the article with just a short example in the PDF. Authors are
required to indicate in the table comments that the data in machine 
readable format in the full article.
Authors are encouraged to create their own machine
readable tables using the online tool at
\url{http://authortools.aas.org/MRT/upload.html}.

\aastex\ v6 introduced five new table features that were designed to make
table construction easier and the resulting display better for AAS Journal
authors.  The items are:

\begin{enumerate}
\item Declaring math mode in specific columns,
\item Column decimal alignment, 
\item Automatic column header numbering,
\item Hiding columns, and
\item Splitting wide tables into two or three parts.
\end{enumerate}

Full details on how to create each type are given in the following 
sections. Additional details are available in the AASTeX
guidelines at \url{http://journals.aas.org/authors/aastex.html}

\subsubsection{Column math mode}

Both the \latex\ tabular and \aastex\ deluxetable require an argument to
define the alignment and number of columns.  The most common values are
``c'', ``l'' and ``r'' for \underline{c}enter, \underline{l}eft, and
\underline{r}ight justification.  If these values are capitalized, e.g.
``C'', ``L'', or ``R'', then that specific column will automatically be in math
mode meaning that \$s are not required.  Note that having embedded dollar
signs in the table does not affect the output. 

\subsubsection{Decimal alignment}

Aligning a column by the decimal point can be difficult with only center,
left, and right justification options.  It is possible to use phantom calls
in the data, e.g. {\tt\string\phn}, to align columns by hand but this can
be tedious in long or complex tables.  To address this \aastex\ introduces
the {\tt\string\decimals} command and a new column justification option,
``D'', to align data in that column on the decimal.  In deluxetable the
{\tt\string\decimals} command is invoked before the {\tt\string\startdata}
call but can be anywhere in \latex's tabular environment.  

Two other important thing to note when using decimal alignment is that each
decimal column \textit{must end with a space before the ampersand}, e.g.
``\&\&'' is not allowed.  Empty decimal columns are indicated with a decimal,
e.g. ``.''.  Do not use deluxetable's {\tt\string\nodata} command.

The ``D'' alignment token works by splitting the column into two parts on the
decimal.  While this is invisible to the user one must be aware of how it
works so that the headers are accounted for correctly.  All decimal column
headers need to span two columns to get the alignment correct. This can be
done with a multicolumn call, e.g {\tt\string\multicolumn2c\{\}} or
{\tt\string\multicolumn\{2\}\{c\}\{\}}, or use the new
{\tt\string\twocolhead\{\}} command in deluxetable.  Since \latex\ is
splitting these columns into two it is important to get the table width
right so that they appear joined on the page.  You may have to run the
\latex\ compiler twice to get it right.  

\subsubsection{Automatic column header numbering} \label{subsubsec:autonumber}

The command {\tt\string\colnumbers} can be included to automatically number
each column as the last row in the header. Per the AAS Journal table format
standards, each column index numbers will be surrounded by parentheses. In
a \latex\ tabular environment the {\tt\string\colnumbers} should be invoked
at the location where the author wants the numbers to appear, e.g. after
the last line of specified table header rows. In deluxetable this command
has to come before {\tt\string\startdata}.  {\tt\string\colnumbers} will
not increment for columns hidden by the ``h'' command, see Section
\ref{subsubsec:hide}. 

Note that when using decimal alignment in a table the command 
{\tt\string\decimalcolnumbers} must be used instead of 
{\tt\string\colnumbers} and {\tt\string\decimals}. 

\subsubsection{Hiding columns} \label{subsubsec:hide}

Entire columns can be \underline{h}idden from display simply by changing
the specified column identifier to ``h''.  In the \latex\ tabular environment
this column identifier conceals the entire column including the header
columns.   In \aastex's deluxetables the header row is specifically
declared with the {\tt\string\tablehead} call and each header column is
marked with {\tt\string\colhead} call.  In order to make a specific header
disappear with the ``h'' column identifier in deluxetable use 
{\tt\string\nocolhead} instead to suppress that particular column header.

Authors can use this option in many different ways.  Since column data can
be easily suppressed authors can include extra information and hid it
based on the comments of co-authors or referees.  For wide tables that will
have a machine readable version, authors could put all the information in
the \latex\ table but use this option to hid as many columns as needed until
it fits on a page. This concealed column table would serve as the
example table for the full machine readable version.  Regardless of how
columns are obscured, authors are responsible for removing any unneeded
column data or alerting the editorial office about how to treat these
columns during production for the final typeset article.

Table \ref{tab:messier} provides some basic information about the first ten
Messier Objects and illustrates how many of these new features can be used
together.  It has automatic column numbering, decimal alignment of the
distances, and one concealed column.  The Common name column
is the third in the \latex\ deluxetable but does not appear when the article
is compiled. This hidden column can be shown simply by changing the ``h'' in
the column identifier preamble to another valid value.  This table also
uses {\tt\string\tablenum} to renumber the table because a \latex\ tabular
table was inserted before it.

\begin{deluxetable*}{cchlDlc}
\tablenum{1}
\tablecaption{Fun facts about the first 10 messier objects\label{tab:messier}}
\tablewidth{0pt}
\tablehead{
\colhead{Messier} & \colhead{NGC/IC} & \nocolhead{Common} & \colhead{Object} &
\multicolumn2c{Distance} & \colhead{} & \colhead{V} \\
\colhead{Number} & \colhead{Number} & \nocolhead{Name} & \colhead{Type} &
\multicolumn2c{(kpc)} & \colhead{Constellation} & \colhead{(mag)}
}
\decimalcolnumbers
\startdata
M1 & NGC 1952 & Crab Nebula & Supernova remnant & 2 & Taurus & 8.4 \\
M2 & NGC 7089 & Messier 2 & Cluster, globular & 11.5 & Aquarius & 6.3 \\
M3 & NGC 5272 & Messier 3 & Cluster, globular & 10.4 & Canes Venatici &  6.2 \\
M4 & NGC 6121 & Messier 4 & Cluster, globular & 2.2 & Scorpius & 5.9 \\
M5 & NGC 5904 & Messier 5 & Cluster, globular & 24.5 & Serpens & 5.9 \\
M6 & NGC 6405 & Butterfly Cluster & Cluster, open & 0.31 & Scorpius & 4.2 \\
M7 & NGC 6475 & Ptolemy Cluster & Cluster, open & 0.3 & Scorpius & 3.3 \\
M8 & NGC 6523 & Lagoon Nebula & Nebula with cluster & 1.25 & Sagittarius & 6.0 \\
M9 & NGC 6333 & Messier 9 & Cluster, globular & 7.91 & Ophiuchus & 8.4 \\
M10 & NGC 6254 & Messier 10 & Cluster, globular & 4.42 & Ophiuchus & 6.4 \\
\enddata
\tablecomments{This table ``hides'' the third column in the \latex\ when compiled.
The Distance is also centered on the decimals.  Note that when using decimal
alignment you need to include the {\tt\string\decimals} command before
{\tt\string\startdata} and all of the values in that column have to have a
space before the next ampersand.}
\end{deluxetable*}

\subsubsection{Splitting a table into multiple horizontal components}

Since the AAS Journals are now all electronic with no print version there is
no reason why tables can not be as wide as authors need them to be.
However, there are some artificial limitations based on the width of a
print page.  The old way around this limitation was to rotate into 
landscape mode and use the smallest available table font
sizes, e.g. {\tt\string\tablewidth}, to get the table to fit.
Unfortunately, this was not always enough but now along with the hide column
option outlined in Section \ref{subsubsec:hide} there is a new way to break
a table into two or three components so that it flows down a page by
invoking a new table type, splittabular or splitdeluxetable. Within these
tables a new ``B'' column separator is introduced.  Much like the vertical
bar option, ``$\vert$'', that produces a vertical table lines 
the new ``B'' separator indicates where to \underline{B}reak
a table.  Up to two ``B''s may be included.

Table 2 % \ref{tab:deluxesplit} this freaks it out when it is used!
shows how to split a wide deluxetable into three parts with
the {\tt\string\splitdeluxetable} command.  The {\tt\string\colnumbers}
option is on to show how the automatic column numbering carries through the
second table component, see Section \ref{subsubsec:autonumber}.

\begin{splitdeluxetable*}{lccccBcccccBcccc}
\tabletypesize{\scriptsize}
\tablewidth{0pt} 
\tablenum{5}
\tablecaption{Measurements of Emission Lines: two breaks \label{tab:deluxesplit}}
\tablehead{
\colhead{Model} & \colhead{Component}& \colhead{Shift} & \colhead{FWHM} &
\multicolumn{10}{c}{Flux} \\
\colhead{} & \colhead{} & \colhead{($\rm
km~s^{-1}$)}& \colhead{($\rm km~s^{-1}$)} & \multicolumn{10}{c}{($\rm
10^{-17}~erg~s^{-1}~cm^{-2}$)} \\
\cline{5-14}
\colhead{} & \colhead{} &
\colhead{} & \colhead{} & \colhead{Ly$\alpha$} & \colhead{N\,{\footnotesize
V}} & \colhead{Si\,{\footnotesize IV}} & \colhead{C\,{\footnotesize IV}} &
\colhead{Mg\,{\footnotesize II}} & \colhead{H$\gamma$} & \colhead{H$\beta$}
& \colhead{H$\alpha$} & \colhead{He\,{\footnotesize I}} &
\colhead{Pa$\gamma$}
} 
\colnumbers
\startdata 
{       }& BELs& -97.13 &    9117$\pm      38$&    1033$\pm      33$&$< 35$&$<     166$&     637$\pm      31$&    1951$\pm      26$&     991$\pm 30$&    3502$\pm      42$&   20285$\pm      80$&    2025$\pm     116$& 1289$\pm     107$\\ 
{Model 1}& IELs& -4049.123 & 1974$\pm      22$&    2495$\pm      30$&$<     42$&$<     109$&     995$\pm 186$&      83$\pm      30$&      75$\pm      23$&     130$\pm      25$& 357$\pm      94$&     194$\pm      64$& 36$\pm      23$\\
{       }& NELs& \nodata &     641$\pm       4$&     449$\pm 23$&$<      6$&$<       9$&       --            &     275$\pm      18$& 150$\pm      11$&     313$\pm      12$&     958$\pm      43$&     318$\pm 34$& 151$\pm       17$\\
\hline
{       }& BELs& -85 &    8991$\pm      41$& 988$\pm      29$&$<     24$&$<     173$&     623$\pm      28$&    1945$\pm 29$&     989$\pm      27$&    3498$\pm      37$&   20288$\pm      73$& 2047$\pm     143$& 1376$\pm     167$\\
{Model 2}& IELs& -51000 &    2025$\pm      26$& 2494$\pm      32$&$<     37$&$<     124$&    1005$\pm     190$&      72$\pm 28$&      72$\pm      21$&     113$\pm      18$&     271$\pm      85$& 205$\pm      72$& 34$\pm      21$\\
{       }& NELs& 52 &     637$\pm      10$&     477$\pm 17$&$<      4$&$<       8$&       --            &     278$\pm      17$& 153$\pm      10$&     317$\pm      15$&     969$\pm      40$&     325$\pm 37$&
     147$\pm       22$\\
\enddata
\tablecomments{This is an example of how to split a deluxetable. You can
split any table with this command into two or three parts.  The location of
the split is given by the author based on the placement of the ``B''
indicators in the column identifier preamble.  For more information please
look at the new \aastex\ instructions.}
\end{splitdeluxetable*}

\subsection{Figures\label{subsec:figures}}

%% The "ht!" tells LaTeX to put the figure "here" first, at the "top" next
%% and to override the normal way of calculating a float position
\begin{figure}[ht!]
\plotone{cost.pdf}
\caption{The subscription (squares) and author publication (asterisks) 
costs from 1991 to 2013. Subscription cost are on the left Y axis while
the author costs are on the right Y axis. All numbers in US dollars and
adjusted for inflation. The author charges also account for the change
from page charges to digital quanta in April 2011.  \label{fig:general}}
\end{figure}

Authors can include a wide number of different graphics with their articles
but encapsulated postscript (EPS) or portable document format (PDF) are
encouraged. These range from general figures all authors are familiar with
to new enhanced graphics that can only be fully experienced in HTML.  The
later include figure sets, animations and interactive figures.  All
enhanced graphics require a static two dimensional representation in the
manuscript to serve as the example for the reader. All figures should
include detailed and descriptive captions.  These captions are absolutely
critical for readers for whom the enhanced figure is inaccessible either
due to a disability or offline access.  This portion of the article
provides examples for setting up all these types in with the latest version
of \aastex.

\subsection{General figures\label{subsec:general}}

\aastex\ has a {\tt\string\plotone} command to display a figure consisting
of one EPS/PDF file.  Figure \ref{fig:general} is an example which shows
the approximate changes in the subscription costs and author publication
charges from 1991 to 2013 in the AAS Journals.  For a general figure
consisting of two EPS/PDF files the {\tt\string\plottwo} command can be
used to position the two image files side by side.

Both {\tt\string\plotone} and {\tt\string\plottwo} take a
{\tt\string\caption} and an optional {\tt\string\figurenum} command to
specify the figure number\footnote{It is better to not use
{\tt\string\figurenum} and let LaTeX auto-increment all the figures. If you
do use this command you need to mark all of them accordingly.}.  Each is
based on the {\tt\string graphicx} package command,
{\tt\string\includegraphics}.  Authors are welcome to use
{\tt\string\includegraphics} along with its optional arguments that control
the height, width, scale, and position angle of a file within the figure.
More information on the full usage of {\tt\string\includegraphics} can be
found at \break
\url{https://en.wikibooks.org/wiki/LaTeX/Importing\_Graphics\#Including\_graphics}.

\subsection{Grid figures}

Including more than two EPS/PDF files in a single figure call can be tricky to
easily format.  To make the process easier for authors \aastex\ v6 offers
{\tt\string\gridline} which allows any number of individual EPS/PDF file
calls within a single figure.  Each file cited in a {\tt\string\gridline}
will be displayed in a row.  By adding more {\tt\string\gridline} calls an
author can easily construct a matrix X by Y individual files as a
single general figure.

For each {\tt\string\gridline} command a EPS/PDF file is called by one of
four different commands.  These are {\tt\string\fig},
{\tt\string\rightfig}, {\tt\string\leftfig}, and {\tt\string\boxedfig}.
The first file call specifies no image position justification while the
next two will right and left justify the image, respectively.  The
{\tt\string\boxedfig} is similar to {\tt\string\fig} except that a box is
drawn around the figure file when displayed. Each of these commands takes
three arguments.  The first is the file name.  The second is the width that
file should be displayed at.  While any natural \latex\ unit is allowed, it
is recommended that author use fractional units with the
{\tt\string\textwidth}.  The last argument is text for a subcaption.

% Figure \ref{fig:pyramid} shows an inverted pyramid of individual
% figure constructed with six individual EPS files using the
% {\tt\string\gridline} option.
% \begin{figure*}
% \gridline{\fig{V2491_Cyg.pdf}{0.3\textwidth}{(a)}
%           \fig{HV_Cet.pdf}{0.3\textwidth}{(b)}
%           \fig{LMC_2009.pdf}{0.3\textwidth}{(c)}
%           }
% \gridline{\fig{RS_Oph.pdf}{0.3\textwidth}{(d)}
%           \fig{U_Sco.pdf}{0.3\textwidth}{(e)}
%           }
% \gridline{\fig{KT_Eri.pdf}{0.3\textwidth}{(f)}}
% \caption{Inverted pyramid figure of six individual files. The nova are
% (a) V2491 Cyg, (b) HV Cet, (c) LMC 2009, (d) RS Oph, (e) U Sco, and (f) 
% KT Eri. These individual figures are taken from \citet{2011ApJS..197...31S}.
% \label{fig:pyramid}}
% \end{figure*}

\subsection{Enhanced graphics}

Enhanced graphics have an example figure to serve as an example for the
reader and the full graphical item available in the published HTML article.
This includes Figure sets, animations, and interactive figures. The 
Astronomy Image Explorer (\url{http://www.astroexplorer.org/}) provides 
access to all the figures published in the AAS Journals since they offered
an electronic version which was in the mid 1990s. You can filter image
searches by specific terms, year, journal, or type. The type filter is 
particularly useful for finding all published enhanced graphics. As of
June 2019 there are over 3000 videos, 1000 figure sets, and 65 interactive
figures. The next sections describe how to include these types of graphics
in your own manuscripts.

\subsubsection{Figure sets}

The grid commands given above works great for a limited set of individual
figure files but what do you do if you have many 10s or 100s or even 1000s of
individual figure files? Figure sets represents a virtual flip book of a
large group of similar style figures.  The derived PDF article will only
shows an example figure while the enhanced content is available in the
figure set in the HTML edition.  The advantage of a figure set gives the
reader the ability to easily sort through a large collection to find
individual component figures.  The advantage to the author is that grouping
similar figures into a figure set can result in significant cost savings in
terms of reduced publication charges, see Appendix B. All of the figure set
components, along with their html framework, are also available to the reader
for download in a single .tar.gz package.

Special \latex\ mark up is required to create a figure set.  Prior to
\aastex\ v6 the underlying mark up commands had to be inserted by hand
but is now included.  Note that when an article with figure set is compiled
in \latex\ none of the component figures are shown and a floating Figure
Set caption will appear in the resulting PDF.


Authors are encouraged to use an online tool at
\url{http://authortools.aas.org/FIGSETS/make-figset.html} to generate their
own specific figure set mark up to incorporate into their \latex\ articles.

\subsubsection{Animations \label{animation}}

Authors may, and are in fact encouraged, to include animations in their
manuscripts. The video will stream inline with the published article and
also be available for download.  When writing the manuscript, a stand alone
figure is necessary to serve as an example for the reader.  Ideally, this
is a single still frame from the animation but in some case the animation
may only represent a small portion of the example figure, say one many
panels as shown in Figure \ref{fig:video}. Regardless, it is very
important that the author provide descriptive text in the figure caption
including start and stop times and the video duration. Authors should
review the AAS animation guidelines in the graphics guide at
\url{https://journals.aas.org/graphics-guide/#animations}.

\begin{figure}
\begin{interactive}{animation}{movie.mp4}
\plotone{f4.pdf}
\end{interactive}
\caption{Figure 1 from \citet{2018ApJ...868L..33L}. AIA 171\AA (a,b), 
AIA 131\AA (c), and AIA 304\AA images are shown. The red rectangle 
in (a) shows the field of view of the other panels. An animation of 
panels (b-d) is available. It covers 8 hours of observing beginning 
at 01:00 UT on 2012 January 19. The video duration is 20 seconds. 
\label{fig:video}}
\end{figure}

Animations and interactive figures (Section \ref{sec:interactive}) should 
use the {\tt\string\begin{interactive}} environment in the figure call. This 
environment
places a blue border around the figure to indicate that the figure is 
enhanced in the published HTML article. The
command also serves to alert the publisher what files are used to generate
the dynamic HTML content. {\tt\string\interactive} takes two arguments. The
first details the type and currently only three are allowed. The types are
{\tt\string js} for generic javascript interactive figures, 
{\tt\string animation} for inline videos, and 
{\tt\string timeseries} for interactive light curves produced
by astropy \citet{2013A&A...558A..33A}\footnote{To be release in the 
summer of 2019}. If these types are not provide the compiler will issue an
error and quit. The second argument is the file that produces the enhanced
feature in the HTML article.

\subsubsection{Interactive figures \label{sec:interactive}}

Interactive figures give the reader the ability to manipulate the
information contained in an image which can add clarity or help further the
author's narrative.  These figures consist of two parts, a static 
representative figure for the manuscript and the dynamic javascript plus
HTML framework that allows for interactive control.

An example of an interactive figure is a 3D model.
The underlying figure is a X3D file while x3dom.js is the javascript driver
that displays it. An author created interface is added via a html wrapper.
The first 3D model published by the AAS Journals using this technique was
\citet{2014ApJ...793..127V}.  

% Figure \ref{fig:interactive} provides an interactive example which can be
run locally to demonstrate how a simple javascript plus html interface
allows a reader to switch between figures. The necessary files for this
particular interactive figure are in the {\tt\string interactive.tar.gz}
file included with this package. Unpack the file and point the browser to
the local html file. In this case, the javascript that runs the interactive
buttons is embedded in the html file but it could just as easily be calls
to external javascript libraries. Ideally, the javascript should be
included with the submitted package of interactive files to minimize
external dependencies within the published article.

% \begin{figure}
% \begin{interactive}{js}{interactive.tar.gz}
% \plotone{f5.pdf}
% \end{interactive}
% \caption{Figure 4 from \citet{2018AJ....156...82C}. \emph{Upper panel}: the
% cumulative median observing time to measure the $3\sigma$ RV masses of TESS
% planets as a function of host star spectral type and up to $10^3$ hours.
% The \emph{dashed blue curves} represent the results from the optical
% spectrograph whereas the \emph{solid red curves} represent the near-IR
% spectrograph. \emph{Lower panel}: the time derivative of the cumulative
% observing time curves used to indicate the RV planet detection efficiency.
% The \emph{horizontal dashed line} highlights the value of the detection
% efficiency at 20 hours per detection.  Note that unlike the lower panels,
% the upper panels do not share a common ordinate due to the differing number
% of planet detections around stars in each spectral type bin. The
% interactive version has two buttons that allows one to turn the optical and
% NIR layers. \label{fig:interactive}}
% \end{figure}


\section{Displaying mathematics} \label{sec:displaymath}

The most common mathematical symbols and formulas are in the amsmath
package.  \aastex\ requires this package so there is no need to
specifically call for it in the document preamble.  Most modern \latex\
distributions already contain this package.  If you do not have this
package or the other required packages, revtex4-1, latexsym, graphicx,
amssymb, longtable, and epsf, they can be obtained from 
\url{http://www.ctan.org}

Mathematics can be displayed either within the text, e.g. $E = mc^2$, or
separate from in an equation.  In order to be properly rendered, all inline
math text has to be declared by surrounding the math by dollar signs (\$).

A complex equation example with inline math as part of the explanation
follows.

\begin{equation}
\bar v(p_2,\sigma_2)P_{-\tau}\hat a_1\hat a_2\cdots
\hat a_nu(p_1,\sigma_1) ,
\end{equation}
where $p$ and $\sigma$ label the initial $e^{\pm}$ four-momenta
and helicities $(\sigma = \pm 1)$, $\hat a_i=a^\mu_i\gamma_\nu$
and $P_\tau=\frac{1}{2}(1+\tau\gamma_5)$ is a chirality projection
operator $(\tau = \pm1)$.  This produces a single line formula.  \latex\ will
auto-number this and any subsequent equations.  If no number is desired then
the {\tt\string equation} call should be replaced with {\tt\string displaymath}.

\latex\ can also handle a a multi-line equation.  Use {\tt\string eqnarray}
for more than one line and end each line with a
\textbackslash\textbackslash.  Each line will be numbered unless the
\textbackslash\textbackslash\ is preceded by a {\tt\string\nonumber}
command.  Alignment points can be added with ampersands (\&).  There should be
two ampersands per line. In the examples they are centered on the equal
symbol.


% \acknowledgments
% We thank ...

\vspace{5mm}
\appendix

\section{Sample Appendix Section}

This is appendix

\bibliography{sample63}{}
\bibliographystyle{aasjournal}

%% This command is needed to show the entire author+affiliation list when
%% the collaboration and author truncation commands are used.  It has to
%% go at the end of the manuscript.
%\allauthors

%% Include this line if you are using the \added, \replaced, \deleted
%% commands to see a summary list of all changes at the end of the article.
%\listofchanges

\end{document}

% End of file `sample63.tex'.
