% \iffalse meta-comment balanced on line 109
% ltxgrid.dtx: package to change page grid, MVL.
% Copyright (c) 2009 Arthur Ogawa
%
% Disclaimer
%   This file is distributed WITHOUT ANY WARRANTY;
%   without even the implied warranty of
%   MERCHANTABILITY or FITNESS FOR A PARTICULAR PURPOSE.
% License
%   You may distribute this file under the conditions of the 
%   LaTeX Project Public License 1.3c or later 
%   (http://www.latex-project.org/lppl.txt).
% ReadMe
%   For the documentation and more detailed instructions for
%   installation, typeset this document with \LaTeX.
% Maintenance Status
%   This work has the LPPL maintenance status "maintained";
%   Current Maintainer of this work is Arthur Ogawa.
%
% This work consists of the main source file ltxgrid.dtx
% and the derived files
%    ltxgrid.sty, ltxgrid.pdf
% Distribution:
%    CTAN:macros/latex/contrib/revtex/
%
% Unpacking:
%    tex ltxgrid.dtx
%
% Documentation:
%    latex ltxgrid.dtx; ...
%
%    Programm calls to get the documentation (example):
%       pdflatex ltxgrid.dtx
%       makeindex -s gind.ist ltxgrid
%       makeindex -s gglo.ist -o ltxgrid.gls ltxgrid.glo
%       pdflatex ltxgrid.dtx
%       makeindex -s gind.ist ltxgrid.idx
%       pdflatex ltxgrid.dtx
%
% Installation:
%    TDS:doc/latex/revtex/
%    TDS:tex/latex/revtex/
%    TDS:source/latex/revtex/
%
% Thanks, Heiko!
%    This method of letting a single .dtx file serve as both
%    documentation (via latex) and installer (via tex) follows
%    the example of Heiko Oberdiek. Thanks!
%<*ignore>
\begingroup
  \def\x{LaTeX2e}%
\expandafter\endgroup
\ifcase
 0\expandafter\ifx\csname processbatchFile\endcsname\relax\else1\fi\ifx\fmtname\x\else 1\fi
 \relax
\else
 \csname fi\endcsname
%</ignore>
%<*install>
%% This file will generate documentation and runtime files
%% from ltxgrid.dtx when run through LaTeX or TeX.
\input docstrip
\preamble

This is a generated file;
altering it directly is inadvisable;
instead, modify the original source file.
See the URL in the file README-LTXGRID.tex.

License
   You may distribute this file under the conditions of the 
   LaTeX Project Public License 1.3c or later 
   (http://www.latex-project.org/lppl.txt).
   
   This file is distributed WITHOUT ANY WARRANTY; 
   without even the implied warranty of MERCHANTABILITY 
   or FITNESS FOR A PARTICULAR PURPOSE.

\endpreamble
\askforoverwritefalse
\keepsilent
 \generate{%
  %{ignore}
% \usedir{tex/latex/revtex}%
  \file{ltxgrid.sty}{%
   \from{ltxgrid.dtx}{package,kernel}%
  }%
 }%
\ifToplevel{
\Msg{***********************************************************}
\Msg{*}
\Msg{* To finish the installation, please move}
\Msg{*    ltxgrid.sty}
\Msg{* into a directory searched by TeX;}
\Msg{* in a TDS-compliant installation:}
\Msg{* texmf/tex/macros/latex/revtex/.}
\Msg{*}
\Msg{* To produce the documentation,
       run ltxgrid.dtx through LaTeX.}
\Msg{*}
\Msg{* Happy TeXing}
\Msg{***********************************************************}
}
\endbatchfile
%</install>
%<*ignore>
\fi
%</ignore>
% \fi
%
% \GetFileInfo{ltxgrid.dtx}\CheckSum{5898}
%
% \iffalse ltxdoc klootch
%<*package>
%%%  @LaTeX-file{
%%%     filename        = "ltxgrid.dtx",
%%%     version         = "4.2c",
%%%     date            = "2019/01/18",
%%%     time            = "14:29:48 UT-5",
%%%     checksum        = "5462",
%%%     author          = "Arthur Ogawa (mailto:arthur_ogawa at sbcglobal.net),
%%%                        commissioned by the American Physical Society.
%%%                        ",
%%%     copyright       = "Copyright (C) 1999, 2009 Arthur Ogawa,
%%%                        distributed under the terms of the 
%%%                        LaTeX Project Public License 1.3c, see
%%%                        ftp://ctan.tug.org/macros/latex/base/lppl.txt
%%%                        ",
%%%     address         = "Arthur Ogawa,
%%%                        USA",
%%%     telephone       = "",
%%%     FAX             = "",
%%%     email           = "mailto colon arthur_ogawa at sbcglobal.net",
%%%     codetable       = "ISO/ASCII",
%%%     keywords        = "latex, page grid, main vertical list",
%%%     supported       = "yes",
%%%     abstract        = "package to change page grid, MVL",
%%%     docstring       = "The checksum field above generated by ltxdoc",
%%%  }
%</package>
% \fi
%
% \iffalse ltxdoc klootch
% The following references the \file{README-LTXGRID} file,
% which contains basic information about this package.
% The contents of this file are generated when
% you typeset the programmer's documentation.
% Search on "{filecontents*}{README-LTXGRID}" to locate it.
% \fi\input{README-LTXGRID}%
%
% \subsection{Bill of Materials}
%
% Following is a list of the files in this distribution arranged
% according to provenance.
%
% \subsubsection{Primary Source}%
% One single file generates all.
%\begin{verbatim}
%ltxgrid.dtx
%\end{verbatim}
%
% \subsubsection{Generated by \texttt{latex ltxgrid.dtx}}%
% Typesetting the source file under \LaTeX\
% generates the readme and the installer.
%\begin{verbatim}
%README-LTXGRID   ltxgrid.ins
%\end{verbatim}
%
% \subsubsection{Generated by \texttt{tex ltxgrid.ins}}%
% Typesetting the installer generates
% the package files.
%\begin{verbatim}
%ltxgrid.sty
%\end{verbatim}
%
% \subsubsection{Documentation}%
% The following are the online documentation:
% \begin{verbatim}
%ltxgrid.pdf
% \end{verbatim}
%
% \subsubsection{Auxiliary}%
% The following are auxiliary files generated
% in the course of running \LaTeX:
% \begin{verbatim}
%ltxgrid.aux ltxgrid.idx ltxgrid.ind ltxgrid.log ltxgrid.toc
% \end{verbatim}
%
% \section{Code common to all modules}%
%
% The following may look a bit klootchy, but we
% want to require only one place in this file
% where the version number is stated,
% and we also want to ensure that the version
% number is embedded into every generated file.
%
% Now we declare that
% these files can only be used with \LaTeXe.
% An appropriate message is displayed if
% a different \TeX{} format is used.
%    \begin{macrocode}
%<*driver|package>
\NeedsTeXFormat{LaTeX2e}[1995/12/01]%
%</driver|package>
%    \end{macrocode}
% As desired, the following modules all
% take common version information:
%    \begin{macrocode}
%<kernel&!package&!doc>\typeout{%
%<*package|doc>
\ProvidesFile{%
%</package|doc>
%<*kernel|package|doc>
ltxgrid%
%</kernel|package|doc>
%<*doc>
.dtx%
%</doc>
%<package>.sty%
%<*package|doc>
}%
%</package|doc>
%    \end{macrocode}
%
% The following line contains, for once and for all,
% the version and date information.
% By various means, this information is reproduced
% consistently in all generated files and in the
% typeset documentation.
% Give credit where due. 
%    \begin{macrocode}
%<*doc|package|kernel>
%<version>
 [2019/01/18/14:29:48 4.2c page grid package (portions licensed from W. E. Baxter web at superscript.com)]% \fileversion
%</doc|package|kernel>
%<kernel&!package&!doc>}%
%    \end{macrocode}
%
%
% \section{The driver module \texttt{doc}}
%
% This module, consisting of the present section,
% typesets the programmer's documentation,
% generating the \file{.ins} installer and \file{README-LTXGRID} as required.
%
% Because the only uncommented-out lines of code at the beginning of
% this file constitute the \file{doc} module itself,
% we can simply typeset the \file{.dtx} file directly,
% and there is thus rarely any need to
% generate the ``doc'' {\sc docstrip} module.
% Module delimiters are nonetheless required so that
% this code does not find its way into the other modules.
%
% The \enve{document} command concludes the typesetting run.
%
%    \begin{macrocode}
%<*driver>
%    \end{macrocode}
%
% \subsection{The Preamble}
% The programmers documentation is formatted
% with the \classname{ltxdoc} class with local customizations,
% and with the usual code line indexing.
%    \begin{macrocode}
\documentclass{ltxdoc}
\RequirePackage{ltxdocext}%
\RequirePackage[colorlinks=true,linkcolor=blue]{hyperref}%
%\ifx\package@font\@undefined\else
% \expandafter\expandafter
% \expandafter\RequirePackage
% \expandafter\expandafter
% \expandafter{%
%              \csname package@font\endcsname
%             }%
%\fi
\CodelineIndex\EnableCrossrefs % makeindex -s gind.ist ltxgrid
\RecordChanges % makeindex -s gglo.ist -o ltxgrid.gls ltxgrid.glo
%    \end{macrocode}
%
% \subsubsection{Docstrip and info directives}
%    We use so many {\sc docstrip} modules that we set the
%    \texttt{StandardModuleDepth} counter to 1.
%    \begin{macrocode}
\setcounter{StandardModuleDepth}{1}
%    \end{macrocode}
%    The following command retrieves the date and version information
%    from this file.
%    \begin{macrocode}
\expandafter\GetFileInfo\expandafter{\jobname.dtx}%
%    \end{macrocode}
%
% \subsection{The ``Read Me'' File}
% As promised above, here is the contents of the
% ``Read Me'' file. That file serves a double purpose,
% since it also constitutes the beginining of the
% programmer's documentation. What better thing, after
% all, to have appear at the beginning of the
% typeset documentation?
%
% A good discussion of how to write a ReadMe file can be found in
% Engst, Tonya, ``Writing a ReadMe File? Read This''
% \emph{MacTech} October 1998, p. 58.
%
% Note the appearance of the
% \cmd\StopEventually\ command, which marks the
% dividing line between the user documentation
% and the programmer documentation.
%
% The usual user will not be asked to
% do a full build, not to speak
% of the bootstrap.
% Instructions for carrying these processes
% begin the programmer's manual.
%
%    \begin{macrocode}
\begin{filecontents*}{README-LTXGRID}
\title{%
 A \LaTeX\ Package for changing the page grid and MVL%
 \thanks{%
  This file has version number \fileversion,
  last revised \filedate.%
 }%
 \thanks{%
  Version \fileversion\ \copyright\ 2019 American Physical Society
 }%
}%
\author{%
 Arthur Ogawa%
  \thanks{\texttt{mailto:arthur\_ogawa at sbcglobal.net}}%
}%
%\iffalse
% For version number and date,
% search on "\fileversion" in the .dtx file,
% or see the end of the README-LTXGRID file.
%\fi
\maketitle

This file embodies the \classname{ltxgrid} package,
the implementation and its user documentation.

The distribution point for this work is
\url{journals.aps.org/revtex},
which contains the REV\TeX\ package, and includes source and documentation for this package.

The \classname{ltxgrid} package was commissioned by the American Physical Society
and is distributed under the terms of the \LaTeX\ Project Public License 1.3c,
the same license under which all the portions of \LaTeX\ itself is distributed.
Please see \url{http://ctan.tug.org/macros/latex/base/lppl.txt} for details.

To use this document class, you must have a working
\TeX\ installation equipped with \LaTeXe\ 
and possibly pdftex and Adobe Acrobat Reader or equivalent.

To install, retrieve the distribution,
unpack it into a directory on the target computer,
into a location in your filesystem where it will be found by \LaTeX;
in a TDS-compliant installation this would be:
\file{texmf/tex/macros/latex/revtex/.}

To use, read the user documentation \file{src/ltxgrid.pdf}.

\tableofcontents

\section{Processing Instructions}

The package file \file{ltxgrid.sty}
is generated from this file, \file{ltxgrid.dtx},
using the {\sc docstrip} facility of \LaTeX
via |tex ltxgrid.dtx|.
The typeset documentation that you are now reading is generated from
this same file by typesetting it with \LaTeX\ or pdftex
via |latex ltxgrid.dtx| or |pdflatex ltxgrid.dtx|.

\subsection{Build Instructions}

You may bootstrap this suite of files solely from \file{ltxgrid.dtx}.
Prepare by installing \LaTeXe\ (and either tex or pdftex) on your computer,
then carry out the following steps:
\begin{enumerate}
\item
Within an otherwise empty directory,
typeset \file{ltxgrid.dtx} with \TeX\ or pdftex;
thereby generating the package file \file{ltxgrid.sty}.

\item
Now typeset \file{ltxgrid.dtx} with \LaTeX\ or pdflatex;
you will obtain the typeset documentation you are now reading,
along with the file \file{README-LTXGRID}.

Note: you will have to run \LaTeX\ twice, then \file{makeindex}, then
\LaTeX\ again in order to obtain a valid index and table of contents.

\item
Install the following files into indicated locations within your 
TDS-compliant \texttt{texmf} tree (you may need root access):
\begin{itemize}
\item
\file{$TEXMF/}\file{tex/}\file{latex/}\file{revtex/}\classname{ltxgrid.sty}
\item
\file{$TEXMF/}\file{source/}\file{latex/}\file{revtex/}\classname{ltxgrid.dtx}
\item
\file{$TEXMF/}\file{doc/}\file{latex/}\file{revtex/}\classname{ltxgrid.pdf}
\end{itemize}
where \file{TEXMF/} stands for \file{texmf-local/}, or some other \texttt{texmf} tree in your installation.
\item
Run \texttt{mktexlsr} on directory \file{$TEXMF/} (you may need root access).
\item
Build and installation are now complete; 
now put a \cmd\usepackage\texttt{\{ltxgrid\}} in your document preamble!
(Note: \texttt{ltxgrid} requires package \texttt{ltxutil}.)
\end{enumerate}

\subsection{Change Log}
\changes{4.0a}{2001/06/18}{Introduce \cs{marry@height} }
\changes{4.0a}{2001/06/18}{Introduce \cs{set@marry@height} }
\changes{4.0a}{2008/06/26 }{\cs{@yfloat}: de-fang \cs{set@footnotewidth} (see ltxutil.dtx): we have already done its job. }
\changes{4.1a}{2008/06/29}{Change \cs{LT@array@new}: restore \cs{@tabularcr} and \cs{@xtabularcr}}
\changes{4.1a}{2008/06/29}{Change \cs{LT@array@new}: set \cs{LT@LL@FM@cr} to \cs{@arraycr@array} instead of \cs{@arraycr}}
\changes{4.1a}{2008/06/29}{Repair error in \cs{endlongtable@new} involving \cs{@ifx}: argument not delimited.}
\changes{4.1b}{2008/08/04}{Get rid of the \cs{reserved@a} idiom}
\changes{4.1b}{2008/08/04}{Turn off the \cs{set@footnotewidth} mechanism; a float `knows' its proper typesetting context}
\changes{4.1b}{2008/08/04}{(AO, 452) Support length checking: show size of shipped out text.}
\changes{4.1b}{2008/08/04}{(AO, 456) Compatibility with other packages that override the output routine, following suggestion by David Kastrup.}
\changes{4.1b}{2008/08/04}{}
\changes{4.1b}{2008/08/04}{Box \cs{footbox} changed to box \cs{footsofar}}
\changes{4.1b}{2008/08/04}{Change \cs{@combinepage} to \cs{@combinepage} with argument}
\changes{4.1b}{2008/08/04}{Change \cs{@makecol} to \cs{@makecolumn} with argument}
\changes{4.1b}{2008/08/04}{Change \cs{set@colroom} to \cs{set@colht}}
\changes{4.1b}{2008/08/04}{New procedure \cs{@iffpsbit} replaces \cs{@getfpsbit}}
\changes{4.1b}{2008/08/04}{New procedure \cs{@output@combined@page}}
\changes{4.1b}{2008/08/04}{New procedure for showing a box contents, \cs{trace@box}}
\changes{4.1b}{2008/08/04}{Procedure \cs{@outputpage@head} headpatches \cs{@outputpage}}%
\changes{4.1b}{2008/08/04}{Procedure \cs{@outputpage@tail} tailpatches \cs{@outputpage}}%
\changes{4.1b}{2008/08/04}{Procedure \cs{balance@2} defined more transparently}%
\changes{4.1b}{2008/08/04}{Tally the height of the float}
\changes{4.1b}{2008/08/04}{Use \cs{document@inithook} instead of \cs{AtBeginDocument}}
\changes{4.1b}{2008/08/04}{Use \cs{trace@box} instead of \cs{showbox}}
\changes{4.1f}{2009/07/07}{(AO, 515) Prevent line numbering within a footnote}
\changes{4.1f}{2009/07/10}{(AO, 518) Tally register overflow when locument is long}
\changes{4.1f}{2009/07/14}{(AO, 519) \cs{footins} content must be preserved and reintegrated}
\changes{4.1f}{2009/07/15}{(AO, 519) Preserve footnotes that are in \cs{footsofar} across a page grid change}
\changes{4.1g}{2009/10/06}{(AO, 531) Fix package \classname{float} }
\changes{4.1n}{2009/12/02}{Restore the \cs{lastbox} if it is not a footnote}
\changes{4.1n}{2009/12/02}{More diagnostics of column balancing}
\changes{4.1n}{2009/12/18}{(AO, 571) Deconstruct balanced footnotes when needed}
\changes{4.1n}{2010/01/02}{(AO, 571) Interface \cs{set@footnotewidth} for determining the set width of footnotes}%
\changes{4.1n}{2010/01/02}{(AO, 571) Footnotes, when columns are balanced or when they are composed with their column}%
\changes{4.1n}{2010/01/02}{(AO, 571) Abandon \cs{recover@footins} in favor of \cs{recover@column}}%
\changes{4.1n}{2010/01/02}{(AO, 571) Use procedures \cs{output@do@prep} and \cs{output@column@do} as dispatchers}
\changes{4.1n}{2010/01/02}{(AO, 571) coding convention: use \cs{bgroup}, \cs{egroup} (instead of braces) when a box is being built}%
\changes{4.1n}{2010/01/02}{(AO, 571) calling sequence of \cs{combine@foot@inserts} and \cs{grid@column} to expose box registers; that of \cs{append@column@} its column counters}%
\changes{4.1n}{2010/01/02}{(AO, 571) footnote rule is leaders, so that it may be removed by \cs{vsplit}; mechanism of \cs{kern} signals to indicate footnote height}%
\changes{4.1o}{2010/02/02}{(AO, 576) Allow \classname{lscape} to act on \cs{@outputbox} at the right time}%
\changes{4.1p}{2010/02/24}{(AO, 583) Provide setup code also for footnotes in a one-column document}%
\changes{4.2a}{2018/12/12}{(MD) Updated name of README file and use standard fonts when typesetting}%

\end{filecontents*}
%    \end{macrocode}
%
% \subsection{The Document Body}
%
% Here is the document body, containing only a
% \cmd\DocInput\ directive---referring to this very file.
% This very cute self-reference is a common \classname{ltxdoc} idiom.
%    \begin{macrocode}
\begin{document}%
\def\revtex{REV\TeX}%
\expandafter\DocInput\expandafter{\jobname.dtx}%
\end{document}
%</driver>
%    \end{macrocode}
%
% \section{Using this package}
% Once this package is installed on your filesystem, you can employ it in
% adding functionality to \LaTeX\ by invoking it in your document or document class.
%
% \subsection{Invoking the package}
% In your document, you can simply call it up in your preamble:
% \begin{verbatim}
%\documentclass{book}%
%\usepackage{ltxgrid}%
%\begin{document}
%<your document here>
%\end{document}
% \end{verbatim}
% However, the preferred way is to invoke this package from within your
% customized document class:
% \begin{verbatim}
%\NeedsTeXFormat{LaTeX2e}[1995/12/01]%
%\ProvidesClass{myclass}%
%\LoadClass{book}%
%\RequirePackage{ltxgrid}%
%<class customization commands>
%\endinput
% \end{verbatim}
%
% Note that this package requires the features of the \classname{ltxutil} package,
% available at
% \url{publish.aps.org/revtex}.
%
% Once loaded, the package gives you acccess to certain procedures,
% usually to be invoked by a \LaTeX\ command or environment, but not at the document level.
% 
% \subsection{Changing the page grid}%
% This package provides two procedures, \cmd\onecolumngrid, \cmd\twocolumngrid,
% that change the page grid (it can be extended to more columns and to other page grids).
%
% They differ from standard \LaTeX's \cmd\onecolumn\ and \cmd\twocolumn\ commands in that
% they do not force a page break. Also, upon leaving a multiple-column grid, the columns are balanced.
% In other respects they work same.
%
% They differ from the grid-changing commands of Frank Mittelbach's \classname{multicol} package
% in that they allow floats of all types (single- and double column floats, that is) and
% preserve compatability with the \classname{longtable} package.
% 
% These commands must be issued in vertical mode (conceivably via a \cmd\vadjust) such that
% they are ultimately present in the MVL, where they can do their work.
% Because they do not work in \LaTeX's left-right mode, they are unsuitable at the 
% document level.
% Furthermore, packaging a grid command in a \cmd\vadjust, although possible, will probably 
% not acheive satisfactory page layout.
%
% Page grid commands are not intended to be issued unnecessarily: only the first of
% two successive \cmd\onecolumngrid\ commands is effective; the second will be silently ignored.
%
% \DescribeMacro\onecolumngrid
% You command \LaTeX\ to return to the one-column grid with the
% \cmd\onecolumngrid\ command. If you are already in the one-column grid, this
% is a no-op. The one-column grid is considered special of all page grids, in that
% no portion of the page is held back (in \cmd\pagesofar); all items that might go
% on the current page (with the exception of floats and footnotes) are on the MVL.
%
% \DescribeMacro\twocolumngrid
% You command \LaTeX\ to return to the two-column grid with the
% \cmd\twocolumngrid\ command. If you are already in the two-column grid, this
% is a no-op.
% 
% These two commands should be issued by a macro procedure that can ensure that 
% \TeX\ is in outer vertical mode. 
% 
% \subsection{Changing the MVL}%
%
% This package also provides commands to modify the main vertical list (MVL) in a safe way.
% The scheme here is to structure, insofar possible, \TeX's MVL as follows:
%\begin{quotation}
%box or boxes\\
%penalty\\
%glue
%\end{quotation}
% This should be a familiar sequence. It is the prototype sequence for a vertical list,
% and is followed when \TeX\ breaks paragraphs into lines, and when \TeX\ generates 
% a display math equation.
%
% If you (as a macro programmer) wish to modify the value of the penalty or glue item, 
% you can use one of the MVL-altering commands to do so. Certain operations are implemented
% here; you can make up your own.
% 
% Note that these commands must be issued in vertical mode, perhaps via a \cmd\vadjust\ or a \cmd\noalign.
% They can work directly if you are in inner mode (say within a parbox or a minipage).
%
% \DescribeMacro\removestuff
% You instruct \LaTeX\ to remove both the penalty and the glue item with this command.
%
% \DescribeMacro\addstuff
% You issue the \cmd\addstuff\arg{penalty}\arg{glue} command to add a penalty, glue, or both.
% If you do not wish to add one or the other, the corresponding argument should be nil.
% Note that the effect of \cmd\addstuff\ is to stack the penalties and glue items.
% Therefore, the lesser of the two penalties takes effect,
% and the two glue items add together. 
%
% \cmd\addstuff\ is limited because once applied, it cannot be applied again with correct results.
%
% \DescribeMacro\replacestuff
% The \cmd\replacestuff\ command is syntactically the same as \cmd\addstuff, but works
% differently: the existing penalty and glue are replaced or modified.
%
% The specified penalty is not inserted if the existing penalty is greater than 10000
% (that is, in case of a \cmd\nobreak), otherwise, the lower (non-zero) of the two penalties is inserted.
% 
% If the specified glue has a larger natural component than the existing glue, we replace the glue.
% However, if the specified glue's natural component is negative, then the existing glue's natural component is 
% changed by that amount.
%
% \cmd\replacestuff\ can be applied mutiple times bceause it retains the list structure in the canonical form.
%
% Note that we treat two penalties specially (as does \TeX): a penalty of 10000 is considered 
% a garbage value, to be replaced if found. This is the signal value that \TeX\ inserts on
% the MVL replacing the penalty that caused the page break (if the page break occurred at a penalty).
% Also, a penalty of zero is indistinguishable from no penalty at all, so it will always 
% be replaced by the given value.
%
% Therefore, it is highly recommended to never set any of \TeX's penalty parameters to 
% zero (a value of, say, 1, is practically the same), nor should a skip parameter be set to zero
% (instead, use, say, 1sp). Also, to prevent a pagebreak, do not use a penalty of 10000, use, say
% 10001 instead.
%
% You can define your own construct that modifies the MVL:
% Define a command, say, \cmd\myadjust, as follows:
%\begin{verbatim}
%\def\myadjust#1{\noexpand\do@main@vlist{\noexpand\@myadjust{#1}}\@tempa}%
%\end{verbatim}
% that is, \cmd\myadjust\ invokes \cmd\do@main@vlist, passing it the procedure name
% \cmd\@myadjust\ along with the arguments thereof pre-expanded.
% Next,  define the procedure \cmd\@myadjust:
%\begin{verbatim}
%\def\@myadjust#1{<meddle with the MVL>}%
%\end{verbatim}
% when \cmd\@myadjust\ executes, you will be in the output routine (in inner vertical mode)
% and the MVL will be that very vertical list.
%
%
%
%
% \section{Compatability with \LaTeX's Required Packages}
% Certain packages, usually ones written by members of the 
% \LaTeX\ Project itself, have been designated ``required'' and
% are distributed as part of standard \LaTeX.
% These packages have been placed in a priviledged position
% vis \'a vis the \LaTeX\ kernel in that they override the definitions of certain kernel macros.
%
% Compatability between \classname{ltxgrid} and these packages is complicated
% by a number of factors. First is that \classname{ltxgrid} alters the meaning of some of the same
% kernel macros as certain of the ``required'' packages.
% Second is that fact that certain of the ``required'' packages of \LaTeX\ are incompatible with 
% each other.
%
% Examples of the first kind are the \classname{ftnright}, \classname{multicol}, and \classname{longtable} 
% packages.
% The \classname{ltxgrid} package is not compatible with \classname{multicol},
% but if you are using \classname{ltxgrid}, you do not need to use \classname{ftnright} or \classname{multicol}
% anyway. The \classname{ltxgrid} package does however attempt to be compatible with \classname{longtable}.
%
% Among the ``required'' packages that are mutually incompatible are \classname{multicol} and \classname{longtable},
% the incompatibility arising because both packages replace \LaTeX's output routine:
% if one package is active, the other must not be so.
% This state of affairs has remained essentially unchanged since the introduction of the two as \LaTeX2.09 packages in the late 1980s.
%
% The reason that \classname{ltxgrid} can remain compatible with \classname{longtable} is due to the
% introduction of a more modern architecture, the ``output routine dispatcher'', which allows all macro packages access to the
% safe processing environment of the output routine, on an equal footing.
% The relevant portions of the \classname{longtable} package are reimplemented in \classname{ltxgrid}
% to take advantage of this mechanism.
%
% Timing is critical:
% the \classname{ltxgrid} package will be incompatible with any package that
% redefines any of the kernel macros that \classname{ltxgrid} patches---if that
% package is loaded \emph{after} \classname{ltxgrid}.
%
% Hereinafter follows some notes on specific \LaTeX\ packages.
%
% \subsection{ftnright}
% Frank Mittelbach's \classname{ftnright} package effects a change to \LaTeX's
% \cmd\twocolumn\ mode such that footnotes are set at the bottom of the right-hand
% column instead of at the foot of each of the two columns.
%
% Note that it overwrites three \LaTeX\ kernel macros: \cmd\@outputdblcol, \cmd\@startcolumn, and \cmd\@makecolumn.
% Fortunately none of the three are patched by \classname{ltxgrid}, so that compatability
% is not excluded on this basis.
%
% At the same time, it changes the meaning of \cmd\footnotesize, the macro that is automatically 
% invoked when setting a document's footnote into type.
% One might well argue that it is an error for the meaning of \cmd\footnotesize\ to be determined by 
% a package such as \classname{ftnright}, that indeed such a choice should be made in the 
% document class, or in a file such as \file{bk10.clo}.
%
% To avoid being tripped up by this misfeature in \classname{ftnright}, it is only necessary to 
% reassert our meaning for \cmd\footnotesize\ later on, after \classname{ftnright} has been loaded.
%
% Note that \classname{ftnright} inserts code that demands that \LaTeX's flag \cmd\if@twocolumn\
% is true, that is, it will complain if deployed in a \cmd\onecolumn\ document.
% It is therefore necessary for any other multicolumn package to assert that flag in order to
% avoid this package's complaint. It is an interesting question exactly why this package has
% this limitation. After all, a one-column page grid is just a degenerate case of the
% two column.
%
% \subsection{longtable}
% David Carlisle's \classname{longtable} package sets tables that can be so long as to break over pages.
% According to its author, it uses the same override of \LaTeX's output routine as
% Frank Mittelbach's \classname{multicol} package. By implication, then, it has a hard
% incompatability with the latter.
% 
% The \classname{longtable} package also performs a check of whether the document is in 
% \cmd\twocolumn\ mode, and declines to work if this is the case. It is not clear, however,
% that there is any true incompatability present if so. It's just that David did not see any reason
% anyone would want to set such long tables in a multicolumn document, hence the check.
%
% There does not appear to be any indication that \classname{longtable} would work less 
% well under \classname{ltxgrid} than under standard \LaTeX's \cmd\twocolumn\ mode.
% Therefore, this \classname{ltxgrid} patches \classname{longtable} (if loaded) so as to provide
% compatability. In the course of which, \classname{longtable} becomes more robust
% (\classname{longtable} has mumerous bugs and incompatabilities of long standing,
% some of which are repaired by \classname{ltxgrid}).
%
% One problem remains, namely that, if a \env{longtable} environment breaks over columns
% and thereby inserts its special headers and footers at that break, and those columms are then
% balanced (due to a return to the one-column page grid), then those inserted rows
% will remain, and may no longer fall at the column break. This will, of course look
% wrong.
%
% The only way to fix this problem is to avoid doing column balancing in the way 
% I have implemented here; such an enhancement to this package is possible.
%
% \subsection{multicol}
% Frank Mittelbach's \classname{multicol} package provides a page grid with many columns,
% albeit denies the placement of floats in individual columns.
%
% It establishes its own \cmd\output\ routine, which is the reason it runs afoul of 
% the \classname{longtable} package. On the other hand, \classname{ltxgrid} specifically
% allows for the case where a package installs its own \cmd\output\ routine, so 
% there is no incompatability on that basis.
%
% Still, it is pointless to use \classname{multicol} if you are using \classname{ltxgrid},
% since both packages provide multicolumn page layouts.
% Therefore, \classname{multicol} is not supported by \classname{ltxgrid}.
%
% \subsection{ltxgrid}
% It has been pointed out that one of the disadvantages of adopting the \classname{ltxgrid} package is that 
% it does alter the \LaTeX\ kernel. 
% Any package that itself alters the \LaTeX\ kernel may be incompatible with \classname{ltxgrid}, and new packages
% (destined perhaps to become part of the successor to \LaTeXe) may break \classname{ltxgrid}.
%
% The consequence is that packages introduced in future, and future changes to \LaTeX\ may be incompatible 
% with \classname{ltxgrid}. 
% This is, of course, true.
% The development plan for \classname{ltxgrid} is that when such packages and \LaTeX\ kernel changes come about,
% the burden will be on \classname{ltxgrid} to change in a way that provides for continued compatability with
% those packages and \LaTeX\ kernel changes.
% 
%
% \section{How \classname{ltxgrid} places footnotes}
%
% In conventional multicolumn layouts, a footnote will appear at the bottom of the column in which it is called out. 
% The \classname{ltxgrid} package implements this conventional layout choice by default.
% However, other choices are possible (a la \classname{ftnright}, whose compatability with \classname{ltxgrid} has not been tested).
%
% One unusual feature of \classname{ltxgrid}'s default implementation must be mentioned, though,
% namely the case in a two-column page grid, where a footnote is followed by a temporary change to the one-column page grid
% (e.g., for a wide equation).
% In such a case, the material above the wide material is split into two columns, and a footnote whose callout 
% appears in the right-hand column will nonetheless be set at the base of the left column.
%
% This arrangement was chosen because it ensures that the footnotes at the bottom of any page will appear in 
% numerical order. It can be argued that this choice is ``incorrect'', but be that as it may,
% the \classname{ltxgrid} package does not foreclose on other arrangements for the footnotes.
% The package can be adapted to accomodate any page design desired.
%
% \section{Limitations in \classname{ltxgrid}'s default column balancing method}%
%
% In a multicolumn page grid, when encountering a page that is not completely full,
% it is customary to set the material in balanced columns (typically with the last column no longer than any of the others).
% Such a case also crops up when temporarily interrupting the multicolumn grid to set material on the full width of the page:
% the material on the page above the break is customarily set in balanced columns.
%
% An awkward case arises when we have already set one or more complete columns of type before encountering the need to 
% balance columns. In this subset of cases, the default in \classname{ltxgrid} is to 
% do an operation I call ``re-balancing'':
% the material on the page so far is pasted back together into a single column, and new, balanced column breaks are
% calculated.
%
% This scheme typically works fine, but it has a significant vulnerability:
% any discardable items trimmed at the original column break are lost, never to be retrieved.
% Consequently, after re-balancing, an element like, say, a section head can fail to have the correct amount of whitespace above.
%
% This problem is due to an unfortunate optimization in \TeX, wherein a certain class of nodes is trimmed from the 
% top of main vertical list upon returning from the output routine:
% any penalty, glue, or leader node falls in to this class of discardable nodes,
% and trimming proceeds until a non-discardable node (such as a box, or rule) is encountered.
% It gets better: a third class of nodes is transparent to this trimming process;
% they are neither discarded nor do they halt the process of trimming:
% mark nodes and all whatsits fall into this class of transparent nodes;
% they are quietly passed over during trimming.
%
% An alternative approach for \TeX\ to take would have been, 
% rather than discarding the node entirely, to simply \emph{mark} it as discarded.
% (Implementors of extended \TeX, please note!)
% Then, upon shipping out, such nodes would not make it into the DVI.
% \TeX's optimization, driven by the small computer architectures current when it was developed,
% does save mem, but at the cost of revisiting page breaks in a reliable way.
%
% FIXME: how to fix a column break in the above case? Widetext?
%
%\StopEventually{}
%
% \section{Implementation of package}
%
% Special acknowledgment: this package uses concepts pioneered
% and first realized by William Baxter (mailto:web at superscript.com)
% in his SuperScript line of commercial typesetting tools, and
% which are used here with his permission. His thorough understanding
% of \TeX's output routine underpins the entire \classname{ltxgrid}
% package.
%
% \subsection{Beginning of the \file{ltxgrid} {\sc docstrip} module}
% Requires the underpinnings of the \classname{ltxkrnext} package.
%    \begin{macrocode}
%<*package>
\def\package@name{ltxgrid}%
\expandafter\PackageInfo\expandafter{\package@name}{%
 Page grid for \protect\LaTeXe,
 by A. Ogawa (arthur_ogawa at sbcglobal.net)%
}%
\RequirePackage{ltxutil}%
%</package>
%    \end{macrocode}
%
% \subsection{Banner}%
%    \begin{macrocode}
%<*kernel>
%    \end{macrocode}
%
% \subsection{Sundry}%
% Here are assorted macro definitions.
% \begin{macro}{\lineloop}
% \begin{macro}{\linefoot}
% The (document-level) command \cmd\lineloop\ sets numbered lines until the 
% specified count is reached. 
% The command \cmd\linefoot\ sets a single, automatically numbered line, but with a footnote (with the specified label);
% it automatically increments the line counter.
% These commands are typically used to construct test documents. 
%
% Because the counter is globally advanced and never reset, successive 
% calls to \cmd\lineloop\ should have an argument ever larger.
% The formatted output will have each line labeled with its ordinal number.
%    \begin{macrocode}
\newcounter{linecount}
\def\loop@line#1#2{%
 \par
 \hb@xt@\hsize{%
  \global\advance#1\@ne
  \edef\@tempa{\@ifnum{100>#1}{0}{}\@ifnum{10>#1}{0}{}\number#1}%
  \@tempa\edef\@tempa{\special{line:\@tempa}}\@tempa
  \vrule depth2.5\p@#2\leaders\hrule\hfil
 }%
}%
\def\lineloop#1{%
 \loopwhile{\loop@line\c@linecount{}\@ifnum{#1>\c@linecount}}%
}%
\def\linefoot#1{%
 \loop@line\c@linecount{%
  \footnote{%
   #1\special{foot:#1}\vrule depth2.5\p@\leaders\hrule\hfill
  }%
 }%
}%
%    \end{macrocode}
% \end{macro}
% \end{macro}
%
%
% \subsection{Mark Components}%
%
% Override LaTeX's mark macros to allow more components.
%
% We remain bound by the weakness of LaTeX's scheme in that
% one cannot emulate the action of \TeX\ whereby
% material with marks can be inserted in the middle of
% a vertical list such that the marks are reliably calculated.
% If we did that, \cmd\@themark\ would no longer be utilized.
%
% A more robust scheme involves placing all marks (component and value)
% into a list (using global scoping, i.e., \cmd\gdef),
% and using \cmd\@@mark to place an index on that list into the MVL.
% Then, e.g., \cmd\@@botmark signifies the place where that list is to be cut,
% and the \cmd\botmark\ of any component is
% the value of the last element of the cut
% list having the given component. The \cmd\firstmark\ and \cmd\topmark\
% can likewise be defined relative to \cmd\@@firstmark\ and \cmd\@@topmark,
% except in the latter case, we want the first following the cut instead of the last
% preceding the cut.
%
% The limitation of this scheme is its demands upon \TeX's mem.
% The list of marks would need to be trimmed back to, effectively,
% \cmd\topmark\ at the beginning of every page.
% 
% This approach is not yet part of the extended LaTeX kernel.
%
% \begin{macro}{\@@mark}
% \begin{macro}{\@@topmark}
% \begin{macro}{\@@firstmark}
% \begin{macro}{\@@botmark}
% \begin{macro}{\@@splitfirstmark}
% \begin{macro}{\@@splitbotmark}
% Remember primitives under a new set of names.
%    \begin{macrocode}
\let\@@mark\mark
\let\@@topmark\topmark
\let\@@firstmark\firstmark
\let\@@botmark\botmark
\let\@@splitfirstmark\splitfirstmark
\let\@@splitbotmark\splitbotmark
%    \end{macrocode}
% \end{macro}
% \end{macro}
% \end{macro}
% \end{macro}
% \end{macro}
% \end{macro}
%
% \subsubsection{Procedures that expose the component data structure}%
% This portion of the code exposes the internal representation of 
% the mark components. If we wish to add more components, we will have to revise
% these macro definitions:
% \cmd\@themark,
% \cmd\nul@mark,
% \cmd\set@mark@netw@,
% \cmd\set@marktw@,
% \cmd\set@markthr@@,
% \cmd\get@mark@@ne,
% \cmd\get@mark@tw@,
% \cmd\get@mark@thr@@,
% \cmd\get@mark@f@ur.
%
% \begin{macro}{\@themark}
%FIXME: is it safer to eliminate \cmd\@themark\ in favor of a message that evaluates \cmd\@@botmark?
%
% Note: these definitions expose the data structure of mark components.
%    \begin{macrocode}
\def\@themark{{}{}{}{}}%
\def\nul@mark{{}{}{}{}\@@nul}%
%    \end{macrocode}
% \end{macro}
%
% \begin{macro}{\set@mark@netw@}
% \begin{macro}{\set@marktw@}
% \begin{macro}{\set@markthr@@}
% These procedures insert the new value of a particular mark component into the given argument.
% They expose the data structure of mark components.
% 
%    \begin{macrocode}
\def\set@mark@netw@#1#2#3#4#5#6#7{\gdef#1{{#6}{#7}{#4}{#5}}\do@mark}%
\def\set@marktw@#1#2#3#4#5#6{\gdef#1{{#2}{#6}{#4}{#5}}\do@mark}%
\def\set@markthr@@#1#2#3#4#5#6{\gdef#1{{#2}{#3}{#6}{#5}}\do@mark}%
%    \end{macrocode}
% \end{macro}
% \end{macro}
% \end{macro}
%
% \begin{macro}{\get@mark@@ne}
% \begin{macro}{\get@mark@tw@}
% \begin{macro}{\get@mark@thr@@}
% \begin{macro}{\get@mark@f@ur}
% These procedures retreive the value of a particular mark component.
% They expose the data structure of mark components.
%    \begin{macrocode}
\def\get@mark@@ne#1#2#3#4#5\@@nul{#1}%
\def\get@mark@tw@#1#2#3#4#5\@@nul{#2}%
\def\get@mark@thr@@#1#2#3#4#5\@@nul{#3}%
\def\get@mark@f@ur#1#2#3#4#5\@@nul{#4}%
%    \end{macrocode}
% \end{macro}
% \end{macro}
% \end{macro}
% \end{macro}
%
%
% \subsubsection{Procedures that do not expose the component data structure}%
%
% \begin{macro}{\mark@netw@}
% \begin{macro}{\marktw@}
% \begin{macro}{\markthr@@}
% These procedures insert the new value of a particular mark component into \cmd\@themark,
% then execute \cmd\do@mark. 
% They constitute the implementation layer for mark components one, two, and three.
% An analogous procedure for component four could be defined; call it \cmd\markf@ur.
% 
%    \begin{macrocode}
\def\mark@netw@{\expandafter\set@mark@netw@\expandafter\@themark\@themark}%
\def\marktw@{\expandafter\set@marktw@\expandafter\@themark\@themark}%
\def\markthr@@{\expandafter\set@markthr@@\expandafter\@themark\@themark}%
%    \end{macrocode}
% \end{macro}
% \end{macro}
% \end{macro}
%
% \begin{macro}{\do@mark}
% \begin{macro}{\do@@mark}
% Access procedures \cmd\mark (AKA \cmd\@@mark).
% The \cmd\do@mark\ procedure is used when a mark is being put down into the MVL;
% \cmd\do@@mark\ when this happens in the output routine.
%    \begin{macrocode}
\def\do@mark{\do@@mark\@themark\nobreak@mark}%
\def\do@@mark#1{%
 \begingroup
  \let@mark
  \@@mark{#1}%
 \endgroup
}%
%    \end{macrocode}
% \end{macro}
% \end{macro}
%
% \begin{macro}{\let@mark}
% \begin{macro}{\nobreak@mark}%
% The procedure that makes \cmd\csname s robust within a mark.
% Use \cmd\appdef\ and \cmd\robust@\ to extend the list.
%    \begin{macrocode}
\def\let@mark{%
 \let\protect\@unexpandable@protect
 \let\label\relax
 \let\index\relax
 \let\glossary\relax
}%
\def\nobreak@mark{%
 \@if@sw\if@nobreak\fi{\@ifvmode{\nobreak}{}}{}%
}%
%    \end{macrocode}
% \end{macro}
% \end{macro}
%
%
% \subsubsection{Using mark components}%
%
% These procedures use the component mark mechanism to implement
% a mark component that remembers the current environment (used in page makeup)
% and the the two mark components left over from the original \LaTeX.
% The fourth component is presently unused.
%
% \begin{macro}{\mark@envir}
% The third mark component's access procedures.
% The \cmd\mark@envir\ and \cmd\bot@envir\ commands are a good model of how to write 
% access procedures for a new mark component.
%    \begin{macrocode}
\def\mark@envir{\markthr@@}%
\def\bot@envir{%
 \expandafter\expandafter
 \expandafter\get@mark@thr@@
 \expandafter\@@botmark
             \nul@mark
}%
%    \end{macrocode}
% \end{macro}
%
% \begin{macro}{\markboth}
% \begin{macro}{\markright}
% \begin{macro}{\leftmark}
% \begin{macro}{\rightmark}
% Set procedures for legacy components.
%    \begin{macrocode}
\def\markboth{\mark@netw@}%
\def\markright{\marktw@}%
%    \end{macrocode}
%
% Retrieval procedures for legacy mark components.
% The procedure for retrieving the first component from \cmd\botmark\ 
% and the second component from \cmd\firstmark have names in \LaTeX;
% they are called, respectively, \cmd\leftmark\ and \cmd\rightmark.
%
% It is possible to retrieve the components of \cmd\topmark\
% as well: use \cmd\saved@@topmark.
%    \begin{macrocode}
\def\leftmark{%
 \expandafter\expandafter
 \expandafter\get@mark@@ne
 \expandafter\saved@@botmark
             \nul@mark
}%
\def\rightmark{%
 \expandafter\expandafter
 \expandafter\get@mark@tw@
 \expandafter\saved@@firstmark
             \nul@mark
}%
%    \end{macrocode}
% \end{macro}
% \end{macro}
% \end{macro}
% \end{macro}
%
%
%
% \subsection{Output Super-routine}%
%
% We want to change \LaTeX's output routine, but do not wish to remain vulnerable
% to interference from such ``required'' packages as
% \classname{multicol} (authored by Frank Mittelbach)
% and \classname{longtable} (authored by David P. Carlisle), which 
% swap in their own output routines when the respective package is active.
%
% The better mechanism, used here, is due to William Baxter (web at superscript.com),
% who has allowed his several ideas to be used in this package.
%
% In what follows, we effectively wrap up the old \LaTeX\ output routine inside
% a new, more flexible ``super routine''. When the output routine is called,
% the ``super routine'' acts as a dispatcher. If the old routine is needed, it is called.
%
% If a package attempts to substitute in their own output routine, they will effectively
% be modifying a token register by the name of \cmd\output.
% The primitive \cmd\output\ is now known by a different name, which should no longer be
% necessary to use.
%
% Usage note: to make a visit to the output routine employing the dispatcher, enter
% with a value of \cmd\outputpenalty\ that corresponds to a macro. Defining as follows:
%\begin{verbatim}
%\@namedef{output@10000}{<your code here>}%
%\end{verbatim}
% by convention, your output routine should void out \cmd\box\cmd\@cclv.
%
% In rewriting \LaTeX's output dispatcher
% in a much simpler form, we also avoid the sin of multiple \cmd\shipout s
% within a single visit to the output routine.
%
% Conceptually, we divide visits to the output routine into two classes.
% The first involves natural page breaks
% (at a \cmd\newpage\ or when \cmd\pagetotal $>$ \cmd\pagegoal)
% and usually resulting in \cmd\box\cmd\@cclv\ either being shipped out or
% salted away (e.g., each column in a multicolumn layout).
% We might call this class the ``natural output routines''; the \cmd\outputpenalty\
% will never be less than $-10000$.
% Furthermore, we ensure that \cmd\holdinginserts\ is cleared when 
% calling such routines.
%
% The other class involves a forced visit to the output routine
% via a large negative penalty ($< -10000$). They do not generally
% result in a \cmd\shipout\ of \cmd\box\cmd\@cclv: they may be dead cycles.
% We provide a mechanism (call it a ``one-off'' output routine) that allows
% us to specify certain processing to be done when \TeX\ reaches
% the current position on the page.
%
% One-off output routines themselves fall into two divisions, ones
% that process \cmd\box\cmd\@cclv, and ones that work on the main vertical list (MVL).
% The former are typified by changes to the page grid, perhaps
% even column balancing.
% The latter involve the insertion of penalties or glue and the processing of floats.
% 
% The natural output routine is a single procedure. We have not introduced multiple
% natural output routines based on the \cmd\outputpenalty\ because \TeX\ does not
% support such a thing: \TeX\ sometimes lays down a penalty whose value is the sum 
% of other penalties. Because of this, we cannot depend on the value of \cmd\outputpenalty\
% in such areas.
%
% We do introduce flexibility in the form of a mechanism for patching into the
% natural output routine. Three hooks are offered, allowing a procedure
% to prepare for the upcoming visit to the output routine,
% access to \cmd\box\cmd\@cclv, and after doing \cmd\shipout\ (or otherwise 
% committing the material to the page).
%
% Environments, commands, and even packages can install their
% own procedures into these hooks.
% For instance, if the longtable package is loaded, it will install
% its procedures, but those procedures will punt if the page break 
% being processed does not actually fall within a longtable environment.
%
% \begin{macro}{\primitive@output}
% Here we remember the \TeX\ primitive \cmd\output\ and its value,
% and then proceed to take over the \cmd\csname\ of \cmd\output,
% making it a \cmd\toks\ register and installing the old value of
% the output routine.
%    \begin{macrocode}
\let\primitive@output\output
%    \end{macrocode}
% \end{macro}
%
% \begin{macro}{\output@latex}
% \begin{macro}{\output}
% Grab the tokens in \cmd\the\cmd\output\ (but without the extra set of braces).
% The value of \cmd\toks@\ must remain untouched until loaded into the appropriate token 
% register; this is done a few lines below.
%    \begin{macrocode}
\long\def\@tempa#1\@@nil{#1}%
            \toks@
\expandafter\expandafter
\expandafter{%
\expandafter \@tempa
             \the\primitive@output
             \@@nil
             }%
\newtoks\output@latex
\output@latex\expandafter{\the\toks@}%
\let\output\output@latex
%    \end{macrocode}
%
% A comment on compatibility with other packages that co-opt the output routine.
%
% Somewhere on the LaTeX-L list, David Kastrup has urged macro  writers to take over 
% the output routine in such a way that others can do likewise. How is this to be 
% accomplished?
%
% Consider what the \classname{lineno} package does when it loads.
% \begin{enumerate}
% \item
%  It does \cmd\let\ cmd\@tempa\ \cmd\output. 
%  This has the effect of identifying \cmd\@tempa\ with the 
%  \cmd\toks\ register we created above to hold the old output routine of \LaTeX.
%  Let us say that was \cmd\toks14. 
% \item
%  \classname{lineno} itself effectively does \cmd\newtoks\ \cmd\@LN@output, 
%  which assigns that \cmd\csname\ to \cmd\toks 15.
% \item
%  It loads \cmd\@LN@output\ with the contents of \cmd\@tempa 
%  (that is, \cmd\toks14, our copy of \LaTeX's output routine).
% \item
%  Then it loads \cmd\@tempa\ with its own desired procedure, 
%  to be executed at \cmd\output\ time, thereby taking over what it thinks is the output routine,
%  but which is in reality the procedure \revtex\ executes when it wants to pass control
%  to \LaTeX's original output routine.
% \item
%  It then does \cmd\let\ \cmd\output\ \cmd\@LN@output, 
%  which now identifies \cmd\output\ with \cmd\toks15, the output routine of \classname{lineno}.
% \item
%  When the \cmd\output\ routine is triggered, 
%  the primitive output routine \cmd\primitive@output\ is executed, and if 
%  appropriate, control is passed to \cmd\output@latex, which \revtex\ had loaded with the old
%  \LaTeX\ output routine, but which is presently loaded with that of \classname{lineno}.
% \item
%  The output routine of \classname{lineno} is executed, and if appropriate control is passed
%  to \cmd\@LN@output, the old output routine of \LaTeX.
% \item
%  Furthermore, the \cmd\csname\ \cmd\output\ now points to \cmd\@LN@output\ (\cmd\toks15).
%  This means that someone coming in after \classname{lineno} to take over the output routine
%  will actually get executed after that of \classname{lineno}, but before \LaTeX.
% \end{enumerate}
% As you can see, the process of taking over the output routine may continue until all of the 
% \cmd\toks\ registers have been allocated. 
% If, say, \classname{newpackage} would itself like to take over the output routine,
% and if it uses the above set of steps, then when the output routine is triggered, the order
% of execution is \revtex, then \classname{lineno}, then \classname{newpackage}, then \LaTeX.
% Each new package inserts itself on front of \LaTeX.
% \end{macro}
% \end{macro}
%
% \begin{macro}{\dispatch@output}
% \changes{4.1b}{2008/08/04}{(AO, 456) Compatibility with other packages that override the output routine, following suggestion by David Kastrup.}
% We now install our own output routine in place of the original 
% output routine of \LaTeX, which is still available as \cmd\the\ \cmd\output.
%
% The output routine is simply the procedure \cmd\dispatch@output.
% It either dispatches to a procedure based on a particular value of
% \cmd\outputpenalty\ or it executes \cmd\the\cmd\output@latex\ tokens.
%    \begin{macrocode}
\primitive@output{\dispatch@output}%
\def\dispatch@output{%
 \let\par\@@par
%    \end{macrocode}
% Try to interpret \cmd\outputpenalty\ as a dispatcher to a message handler, 
% its value is, e.g., \cmd\do@startpage@pen.
%    \begin{macrocode}
 \expandafter\let\expandafter\output@procedure\csname output@\the\outputpenalty\endcsname
%    \end{macrocode}
% If we have failed to find a dispatcher, then settle for \cmd\output@latex.
%    \begin{macrocode}
 \@ifnotrelax\output@procedure{}{%
  \expandafter\def\expandafter\output@procedure\expandafter{\the\output@latex}%
 }%
%    \end{macrocode}
% Now test if the dispatcher is the special case of \cmd\execute@message@pen, in which case execute the \cmd\@message@saved.
% \changes{4.1n}{2009/12/02}{More diagnostics of column balancing}
%    \begin{macrocode}
 \expandafter\@ifx\expandafter{\csname output@-\the\execute@message@pen\endcsname\output@procedure}{%
  \let\output@procedure\@message@saved
 }{}%
 \ltxgrid@info@sw{\class@info{\string\dispatch@output}\say\output@procedure\saythe\holdinginserts}{}%
 \outputdebug@sw{\output@debug}{}%
 \output@procedure
}%
%    \end{macrocode}
%
%    \begin{macrocode}
\def\set@output@procedure#1#2{%
 \count@\outputpenalty\advance\count@-#2%
 \expandafter\let\expandafter#1\csname output@\the\count@\endcsname
}%
%    \end{macrocode}
%
% \changes{4.1n}{2009/12/02}{More diagnostics of column balancing}
% The following procedure is executed at the beginning of each visit to the output routine, 
% contingent on the level of diagnostics specified. 
% However, it bails out when the visit is part of a tight sequence of visits to the output routine. 
%    \begin{macrocode}
\def\output@debug{%
 \def\@tempa{\save@message}%
 \@ifx{\output@procedure\@tempa}{%
  \true@sw
 }{%
  \@ifnum{\outputpenalty=-\save@column@insert@pen}{%
   \@ifnum{\holdinginserts>\z@}%
  }{%
   \false@sw
  }%
 }%
 {}{\output@debug@}%
}%
\def\output@debug@{%
%<ignore> 	\saythe\inputlineno
 	\saythe\outputpenalty
 	\saythe\interlinepenalty
 	\saythe\brokenpenalty
 	\saythe\clubpenalty
 	\saythe\widowpenalty
 	\saythe\displaywidowpenalty
 	\saythe\predisplaypenalty
 	\saythe\interdisplaylinepenalty
 	\saythe\postdisplaypenalty
 	\saythe\badness
 	\say\thepagegrid
 	\saythe\pagegrid@col
 	\saythe\pagegrid@cur
%<ignore>  \say\bot@envir
 	\saythe\insertpenalties
%<ignore>  \say\@@topmark
%<ignore>  \say\saved@@topmark
%<ignore>  \say\@@firstmark
%<ignore>  \say\saved@@firstmark
 	\say\@@botmark
%<ignore>  \say\saved@@botmark
 	\saythe\pagegoal
 	\saythe\pagetotal
 	\saythe{\badness\@cclv}%
 	\say\@toplist
 	\say\@botlist
 	\say\@dbltoplist
 	\say\@deferlist
 	\trace@scroll{%
 		\showbox\@cclv
 		\showbox\@cclv@saved
 		\showbox\pagesofar
%    \end{macrocode}
% Klootch! The following line provides only for two-column page grid; if debugging more columns, you must
% add more statements here. 
%    \begin{macrocode}
 		\showbox\csname col@1\endcsname
 		\showbox\footsofar
 		\showbox\footins
 		\showbox\footins@saved
 		\showlists
 	}%
}%
\@ifxundefined{\outputdebug@sw}{%
 \@booleanfalse\outputdebug@sw
}{}%
\def\trace@scroll#1{\begingroup\showboxbreadth\maxdimen\showboxdepth\maxdimen\scrollmode#1\endgroup}%
\def\trace@box#1{\trace@scroll{\showbox#1}}%
%    \end{macrocode}
% \end{macro}
%
% \begin{macro}{\@outputpage}
% \begin{macro}{\@outputpage@head}
% \begin{macro}{\@outputpage@tail}
% \changes{4.1b}{2008/08/04}{Procedure \cs{@outputpage@tail} tailpatches \cs{@outputpage}}%
% \changes{4.1b}{2008/08/04}{Procedure \cs{@outputpage@head} headpatches \cs{@outputpage}}%
% The procedure \cmd\@outputpage\ of standard \LaTeX\ is the sole place where a \cmd\shipout\ is carried out.
% The procedures that build \cmd\@outputbox\ just before a page is shipped out by \cmd\@outputpage\ are:
% \cmd\@makecolumn,
% \cmd\@combinepage, and
% \cmd\@combinedblfloats.
%
% We need to head- and tailpatch this procedure, so we perform here
% the only modifications to that procedure that are essential.
% Elsewhere, we will build up the meanings of \cmd\@outputpage@head\ and \cmd\@outputpage@tail.
%    \begin{macrocode}
\prepdef\@outputpage{\@outputpage@head}%
\let\@outputpage@head\@empty
\appdef\@outputpage{\@outputpage@tail}%
\let\@outputpage@tail\@empty
%    \end{macrocode}
% \end{macro}
% \end{macro}
% \end{macro}
%
% \begin{macro}{\show@box@size}
% \begin{macro}{\show@text@box@size}
% \begin{macro}{\show@pagesofar@size}
% \begin{macro}{\show@box@size@sw}
% \begin{macro}{\total@text}
% \changes{4.1b}{2008/08/04}{(AO, 452) Support length checking: show size of shipped out text.}
% Procedure \cmd\show@box@size\ is a diagnostic for the sizes of boxes;
% the boolean \cmd\show@box@size@sw\ turns it on and off.
%    \begin{macrocode}
\def\show@box@size#1#2{%
 \show@box@size@sw{%
  \begingroup
   \setbox\z@\vbox{\unvcopy#2\hrule}%
   \class@info{Show box size: #1^^J%
    (\the\ht\z@\space X \the\wd\z@) 
    \the\c@page\space\space\the\pagegrid@cur\space\the\pagegrid@col
   }%
  \endgroup
 }{}%
}%
%    \end{macrocode}
% Procedure \cmd\show@text@box@size\ tallies the size of the indicated column.
% If \cmd\box\ \cmd\pagesofar\ is a factor, then its height has been memorized
% in the depth of the tally box.
% 
% \changes{4.1f}{2009/07/10}{(AO, 518) Tally register overflow when locument is long}
%    \begin{macrocode}
\def\show@text@box@size{%
 \show@box@size{Text column}\@outputbox
 \tally@box@size@sw{%
  \@ifdim{\wd\@outputbox>\z@}{%
   \dimen@\ht\@outputbox\divide\dimen@\@twopowerfourteen
   \advance\dimen@-\dp\csname box@size@\the\pagegrid@col\endcsname
   \@ifdim{\dimen@>\z@}{%
    \advance\dimen@ \ht\csname box@size@\the\pagegrid@col\endcsname
    \global\ht\csname box@size@\the\pagegrid@col\endcsname\dimen@
    \show@box@size@sw{%
     \class@info{Column: \the\dimen@}%
    }{}%
   }{}%
  }{}%
  \global\dp\csname box@size@\the\pagegrid@col\endcsname\z@
 }{}%
}%
%    \end{macrocode}
% Take the height of \cmd\box\ \cmd\pagesofar\ into account.
%    \begin{macrocode}
\def\show@pagesofar@size{%
 \show@box@size{Page so far}\pagesofar
 \dimen@\ht\pagesofar\divide\dimen@\@twopowerfourteen
 \global\dp\csname box@size@1\endcsname\dimen@
 \show@box@size@sw{%
  \class@info{Pagesofar: \the\dimen@}%
 }{}%
}%
\@booleanfalse\tally@box@size@sw
\@booleanfalse\show@box@size@sw
\expandafter\newbox\csname box@size@1\endcsname
\expandafter\setbox\csname box@size@1\endcsname\hbox{}%
\expandafter\newbox\csname box@size@2\endcsname
\expandafter\setbox\csname box@size@2\endcsname\hbox{}%
\def\total@text{%
 \@tempdima\the\ht\csname box@size@2\endcsname\divide\@tempdima\@twopowertwo\@tempcnta\@tempdima
 \@tempdimb\the\ht\csname box@size@1\endcsname\divide\@tempdimb\@twopowertwo\@tempcntb\@tempdimb
 \class@info{Total text: Column(\the\@tempcnta pt), Page(\the\@tempcntb pt)}%
}%
%    \end{macrocode}
% \end{macro}
% \end{macro}
% \end{macro}
% \end{macro}
% \end{macro}
%
%
% \subsection{Further thoughts about inserts}
%
% The only safe way to deal with inserts is to either set \cmd\holdininserts\ or
% to commit to using whatever insert comes your way: you cannot change your mind
% once you see a non-void \cmd\box\cmd\footins, say.
%
% Therefore all output routine processing must proceed with \cmd\holdinginserts\ set
% until you are sure of the material to be committed to the page. At that point, you
% can clear \cmd\holdinginserts, spew \cmd\box\cmd\@cclv, put down the appropriate penalty, 
% and exit, with the knowledge that \TeX\ will re-find the same pagebreak, this time 
% visiting the output routine with everything, including inserts, in their proper 
% place.
% This technique applies to split elements (screens, longtable, index) as well as to 
% manufactured pages (float pages and clearpage pages). 
%
% Therefore, the output routine must not make assumptions about whether \cmd\holdinginserts\
% should be cleared; instead this must be left to the one-off output routines or the natural output routine.
%
% If we are manufacturing pages (``float page processing''), and if \cmd\pagegoal\ is not equal to
% \cmd\vsize, then inserts are at hand, and our criterion should take into account the insert
% material, even though we cannot measure its height based on the size of \cmd\box\cmd\footins\
% (because \cmd\holdinginserts\ is set, you see).
%
% It would be better to take the complement of \cmd\floatpagefraction\ and use that
% as a standard for the looseness of the page. Since \cmd\pagegoal\ reflects the inserted material,
% the criterion becomes the difference of the aggregate height of the floats and the \cmd\pagegoal\
% versus this "page looseness" standard.
%
% As a check, consider what happens if we bail out: \cmd\@deferlist\ has never been touched, so it 
% requires no attention. Also, \cmd\holdinginserts\ has never been cleared, so inserts require
% no attention. So we only have to ensure that marks are preserved, which is already taken
% care of by the message handler mechanism.
%
% If we are doing ordinary page cutting, then the scheme would be to detect whether we are within
% a screen (or longtable as may be), do the adjustment to the page height, and return, but this time
% with \cmd\holdinginserts\ cleared. Upon reentering the output routine, we may or may not be within
% the screen environment, but we are now sure to have a final page break, and we can commit this
% material (by shipping out or by saving it out as a full column).
%
% In the above, the first of the two visits to the output routine is a dead cycle and requires
% propagation of marks, but nothing else.
%
% \subsection{The difference between inserts and floats}
%
% While revisiting this package in 2008, I needed to clarify under what circumstances inserts
% would be added to the \cmd\pagesofar. My conclusion is that I had been treating them
% exactly the same as floats, but that was a mistake. 
%
% Floats can be committed at the top of a column, in the middle, or at the bottom. 
% Footnotes (the only \cmd\insert\ that is used in \LaTeX) may only be committed at the bottom
% of a column. So, it was necessary to provide two versions of \cmd\@combinepage, one 
% that committed \cmd\insert s, and the other that did not, the former used only when 
% a column of text was committed. Note that even after a column is committed, we could change our minds:
% for instance if in multicolumn grid and we decide to balance the columns. 
%
%
% \subsection{The natural output routine}
%
% Here is the portion of the output routine that fields cases not handled by
% the dispatcher. 
%
% The default is to ship out a page and then look around for more material
% that might constitute a ``float page''. However, because \cmd\holdinginserts\
% is normally set, this output routine must first have a dead cycle and
% come back again with \cmd\holdinginserts\ cleared.
% Then, after shipping out, it puts down a message that 
% will manufacture zero or more float pages, finally terminating
% with a procedure that commits floats to a new unfinished page.
%
% To accomodate special processing, we execute hooks whose name is based
% on the value of the "envir" mark component. The default is "document",
% ensured by an initial mark of that value; the associated procedures
% are all nil. Any unknown envir value will "\cmd\relax\ out".
%
% The test made by \cmd\toggle@insert\ tells whether we are on our first visit to
% the output routine (with \cmd\holdinginserts\ still positive), or our second
% (with \cmd\holdinginserts\ zeroed). The output routine will toggle the setting. 
%
% The commands \cmd\hold@insertions\ and \cmd\move@insertions\ respectively
% clear and set \cmd\holdinginserts, so this procedure effectively
% clears \cmd\holdinginserts\ just long enough to pick up the insertions.
% Important: any output routine that clears \cmd\holdinginserts\
% must guarentee that it is restored on the subsequent visit to the output routine.
% Or, to put it another way, if an output routine detects that \cmd\holdinginserts\ 
% is cleared, it should take it upon itself to restore it to a positive value before exiting.
%
% The branch with \cmd\holdinginserts\ set is executed first; the other
% branch follows on practically immediately thereafter. In the first branch,
% we simply execute the appropriate hook and then execute a dead cycle.
%
% In the branch with \cmd\holdinginserts\ cleared, the procedure
% builds up the current column, which is now complete, with \cmd\@makecolumn, then
% dispatches to the shipout routine associated with the current page grid, \cmd\output@column@.
% At the end, it triggers the execution of an output routine to prepare the next column (or page).
%
% \subsection{Natural output routine}%
%
% \begin{macro}{\natural@output}
% \begin{macro}{\output}
% Here is the output routine that handles natural pagebreaks:
% we now have page that needs to be shipped out or a portion of a page that is ready to be committed to the page grid.
% Processing is of necessity divided into phases, \cmd\output@holding\ is executed upon first encountering the natural page-breaking point, while inserts are being held.
% The second phase, \cmd\output@moving, is set in motion by the first: here the same material (in most cases) will be processed with \cmd\holdinginserts\ cleared,
% and the insertions (e.g., footnotes) are split off into their assigned box registers. 
%    \begin{macrocode}
\def\natural@output{\toggle@insert{\output@holding}{\output@moving}}%
\output@latex{\natural@output}%
%    \end{macrocode}
%
% In accordance with the scheme suggested by David Kastrup for allowing another output routine
% to slip itself into ours, we use a token register called \cmd\output. 
% However, we reserve the ability to restore things if we so desire.
% This we must do in the case of the \filename{lineno.sty} package, because its functionality
% is best served by being integrated into our own dispatcher-based output routine.
%
% To restore our own output routine, we can repeat the above assignment,
% \begin{verbatim}
%\output@latex{\natural@output}%
% \end{verbatim}
% some time before the document begins.
% \end{macro}
% \end{macro}
%
% \begin{macro}{\output@holding}
% \begin{macro}{\@if@exceed@pagegoal}
% The procedure \cmd\output@holding\
% is our first cycle through the output routine; \cmd\holdinginserts\ is still set.
% We give the current environment a heads up
% (it is through this means that \classname{longtable} sets its running header and footer),
% then we execute a dead cycle, which should propagate marks. 
%
% One corner case that can crop up is the presence of a single unbreakable chunk whose size is larger
% than \cmd\vsize.
% Doing a dead cycle under such circumstances will not find the same breakpoint as this time
% (remember we threw in a \cmd\mark\ node).
% Instead, we attempt to remove the excess height of the material, so we can continue to propagate marks.
%
% The corner case is at hand if the natural size of \cmd\box\cmd\@cclv\ exceeds \cmd\pagegoal\ and
% the contents cannot be shrunk to fit.
%
% \changes{4.1b}{2008/08/04}{New procedure for showing a box contents, \cs{trace@box}}
%    \begin{macrocode}
\def\output@holding{%
 \csname output@init@\bot@envir\endcsname
 \@if@exceed@pagegoal{\unvcopy\@cclv}{%
  \setbox\z@\vbox{\unvcopy\@cclv}%
  \outputdebug@sw{\trace@box\z@}{}%
  \dimen@\ht\@cclv\advance\dimen@-\ht\z@
  \dead@cycle@repair\dimen@
 }{%
  \dead@cycle
 }%
}%
\def\@if@exceed@pagegoal#1{%
 \begingroup
  \setbox\z@\vbox{#1}%
  \dimen@\ht\z@\advance\dimen@\dp\z@
  \outputdebug@sw{\saythe\dimen@}{}%
  \@ifdim{\dimen@>\pagegoal}{%
   \setbox\z@\vbox{\@@mark{}\unvbox\z@}%
   \splittopskip\topskip
   \splitmaxdepth\maxdepth
   \vbadness\@M
   \vfuzz\maxdimen
   \setbox\tw@\vsplit\z@ to\pagegoal
   \outputdebug@sw{\trace@scroll{\showbox\tw@\showbox\z@}}{}%
   \setbox\tw@\vbox{\unvbox\tw@}%
   \@ifdim{\ht\tw@=\z@}{%
    \ltxgrid@info{Found overly large chunk while preparing to move insertions. Attempting repairs}%
    \aftergroup\true@sw
   }{%
    \aftergroup\false@sw
   }%
 }{%
  \aftergroup\false@sw
 }%
 \endgroup
}%
%    \end{macrocode}
% \end{macro}
% \end{macro}
%
% \begin{macro}{\output@moving}
% \begin{macro}{\@cclv@nontrivial@sw}
% The procedure \cmd\output@moving\ 
% is our second cycle through the output routine; \cmd\holdinginserts\ is now cleared,
% and \cmd\insert s will have been split off into their respective box registers, like \cmd\footins.
%
% \begin{enumerate}
% \item
%  Set the values of \cmd\topmark\ and \cmd\firstmark.
% \item
%  If we got here because of a \cmd\clearpage\ command, remove the protection box that this mechanism has left on the MVL.
% \item
%  If the contents of \cmd\box\cmd\@cclv\ are non-trivial, commit it to the current page (as a column) or ship it out, as the case may call for.
% \item
%  If not, discard it (we are at the end of \cmd\clearpage\ processing).
% \item
%  Set various values, including the available space for setting type on the next column (\cmd\@colroom).
% \end{enumerate}
%
% The processing for a non-trivial \cmd\box\cmd\@cclv\ are:
% \begin{enumerate}
% \item
%  Execute the head procedure for the current environment.
% \item
%  Make up a column and ship it out (or commit it to the current page) via a procedure keyed to the current page grid.
% \item
%  Put down an interrupt for \cmd\do@startcolumn@pen: this will force a visit to the output routine for the 
%  purpose of committing floats to the next column.
% \item
%  Possibly put down an interrupt to continue \cmd\clearpage\ proccessing.
% \item
%  Execute the tail procedure for the current environment.
% \end{enumerate}
%
%
% The processing for a trivial \cmd\box\cmd\@cclv\ are:
% \begin{enumerate}
% \item
%  Void out \cmd\box\cmd\@cclv\ and give appropriate warning messages and diagnostics.
% \item
%  Put down the same interrupts as for the non-trivial case above.
% \end{enumerate}
%
% This instance of \cmd\@makecolumn\ is followed by \cmd\output@column@, 
% that is, it builds a column for \cmd\shipout\ rather than for adding to \cmd\pagesofar. 
%
% We need to handle cases where the \cmd\output@pre@, \cmd\output@column@, or \cmd\output@post@\
% dispatchers come up \cmd\relax ed out: the default is to execute the corresponding 
% procedures from the \env{docuemnt} environment and the one-column grid respectively.
%
% One such case comes up with frequency: at the end of the document, where the \cmd\botmark\ is
% now empty.
% \changes{4.1b}{2008/08/04}{Change \cs{@makecol} to \cs{@makecolumn} with argument}
% \changes{4.1b}{2008/08/04}{Change \cs{set@colroom} to \cs{set@colht}}
% \changes{4.1n}{2010/01/02}{(AO, 571) Use procedures \cs{output@do@prep} and \cs{output@column@do} as dispatchers}
%    \begin{macrocode}
\def\output@moving{%
 \set@top@firstmark
 \@ifnum{\outputpenalty=\do@newpage@pen}{%
  \setbox\@cclv\vbox{%
   \unvbox\@cclv
   \remove@lastbox
   \@ifdim{\ht\z@=\ht\@protection@box}{\box\lastbox}{\unskip}%
  }%
 }{}%
 \@cclv@nontrivial@sw{%
  \expandafter\output@do@prep\csname output@prep@\bot@envir \endcsname
  \@makecolumn\true@sw
  \expandafter\output@column@do\csname output@column@\thepagegrid\endcsname
  \protect@penalty\do@startcolumn@pen
  \clearpage@sw{%
   \protect@penalty\do@endpage@pen
  }{}%
  \expandafter\let\expandafter\output@post@\csname output@post@\bot@envir \endcsname
  \outputdebug@sw{\say\output@post@}{}%
  \@ifx{\output@post@\relax}{\output@post@document}{\output@post@}%
 }{%
  \void@cclv
 }%
 \set@colht
 \global\@mparbottom\z@
 \global\@textfloatsheight\z@
}%
%    \end{macrocode}
% Procedure \cmd\output@do@prep\ dispatches to the proper procedure to prepare page.
%    \begin{macrocode}
\def\output@do@prep#1{%
 \outputdebug@sw{\class@info{Prep: \string#1}}{}%
 \@ifx{#1\relax}{\output@prep@document}{#1}%
}%
%    \end{macrocode}
% Procedure \cmd\output@column@do\ dispatches to the proper procedure to output column or page.
%    \begin{macrocode}
\def\output@column@do#1{%
  \outputdebug@sw{\class@info{Output column: \string#1}}{}%
  \@ifx{#1\relax}{\output@column@one}{#1}%
}%
\def\void@cclv{\begingroup\setbox\z@\box\@cclv\endgroup}%
\def\remove@lastbox{\setbox\z@\lastbox}%
%    \end{macrocode}
%
% The procedure \cmd\@cclv@nontrivial@sw\ determines if this visit to \cmd\output@moving\
% is a trivial one, which happens at the end of \cmd\clearpage\ processing and under some pathological circumstances.
% It emits a Boolean, so it is syntactically like \cmd\true@sw, albeit does not execute solely via expansion.
%
% Note: the case where \cmd\box\cmd\@cclv\ is void comes up at the very beginning of the job, when 
% typesetting a (full-page-width) title block in a two-column layout.
%
% Note: the code that removes the last box and skip from the output is intended to detect the case
% where the output has whatsit nodes followed by topskip and a protection box.
% This is what happens under normal circumstances at the end of \cmd\clearpage\ processing.
% \changes{4.1b}{2008/08/04}{New procedure for showing a box contents, \cs{trace@box}}
%    \begin{macrocode}
\def\@cclv@nontrivial@sw{%
 \@ifx@empty\@toplist{%
  \@ifx@empty\@botlist{%
   \@ifvoid\footins{%
    \@ifvoid\@cclv{%
     \false@sw
    }{%
     \setbox\z@\vbox{\unvcopy\@cclv}%
     \@ifdim{\ht\z@=\topskip}{%
      \setbox\z@\vbox\bgroup
       \unvbox\z@
       \remove@lastbox
       \dimen@\lastskip\unskip
       \@ifdim{\ht\z@=\ht\@protection@box}{%
        \advance\dimen@\ht\z@
        \@ifdim{\dimen@=\topskip}{%
         \aftergroup\true@sw
        }{%
         \aftergroup\false@sw
        }%
       }{%
        \aftergroup\false@sw
       }%
%    \end{macrocode}
% End of \cmd\box\cmd\z@.
%    \begin{macrocode}
      \egroup
      {%
%    \end{macrocode}
% Normal for \clearpage
%    \begin{macrocode}
       \false@sw
      }{%
       \true@sw
      }%
     }{%
      \@ifdim{\ht\z@=\z@}{%
       \ltxgrid@info{Found trivial column. Discarding it}%
       \outputdebug@sw{\trace@box\@cclv}{}%
       \false@sw
      }{%
       \true@sw
      }%
     }%
    }%
   }{%
    \true@sw
   }%
  }{%
   \true@sw
  }%
 }{%
  \true@sw
 }%
}%
%    \end{macrocode}
%
% 
% \end{macro}
% \end{macro}
%
% \begin{macro}{\protect@penalty}
% The procedure \cmd\protect@penalty\ is the utility procedure for invoking a 
% one-off output routine. Such a routine can expect to find the protection box
% above it in \cmd\box\cmd\@cclv: it should remove that box.
%
% Note that \cmd\execute@message\ does the same thing as \cmd\protect@penalty, but
% in a slightly different way.
%
% We create a specially formulated box that will be universally used when a protection box is needed.
% In this way, we can always recognize when \cmd\box\cmd\@cclv\ is trivial:
% it will consist of whatsits followed by \cmd\topskip\ glue and the \cmd\@protection@box.
%    \begin{macrocode}
\def\protect@penalty#1{\protection@box\penalty-#1\relax}%
\newbox\@protection@box
\setbox\@protection@box\vbox to1986sp{\vfil}%
\def\protection@box{\nointerlineskip\copy\@protection@box}%
%    \end{macrocode}
% \end{macro}
%
% \begin{macro}{\dead@cycle}
% \begin{macro}{\dead@cycle@repair}
% The procedure \cmd\dead@cycle\ is defined separately as a utility which can be used by
% any output processing routine to emulate what takes place in the standard output routine.
%
% Here, we have entered the output routine with \cmd\holdinginserts\ enabled, which means that we 
% are not yet ready to ship out material, because the \cmd\insert\ registers are being held.
% We want to clear \cmd\holdinginserts\ and come back here with the same page break as before, whereupon
% we may properly proceed with page makeup.
%
% To do this, we
% propagate marks, then spew the contents of \cmd\box\cmd\@cclv\ followed by the 
% original output penalty that landed us here (but only if it is not 10000,
% the flag value for a pagebreak not at a penalty).
%
% However, the natural output routine should do this only if \cmd\box\cmd\@cclv\ is nontrivial.
% A pathological case exists wherein a box of height greater than \cmd\textheight\ would cause an infinite loop involving the output routine.
% The procedure \cmd\dead@cycle@repair, attempts to catch this case and avoid the loop.
%
% The test of the height of \cmd\box\cmd\@cclv\ is not the correct one, because this test will run afoul in
% the case where \cmd\box\cmd\@cclv\ contains nothing but an \cmd\insert\ node. What to do?
%
% It is possible that the pathological case can be detected by looking at \cmd\pagetotal. If that quantity is
% zero, then \cmd\box\cmd\@cclv\ really is trivial.
%
% In the procedure \cmd\dead@cycle@repair, if \cmd\box\cmd\@cclv\ is nontrivial, we execute \cmd\dead@cycle,
% otherwise it contains nothing but a mark, so we dispense with propagating marks
% and we simply spew out \cmd\box\cmd\@cclv\ without an accompanying mark.
% This has the effect of failing to propagate marks, but this problem is preferrable to the infinite loop,
% which in principle could crash even a robust operating system by filling up the file system.
%
% If a document has such a large chunk, it should be fixed, so we give a message in the log.
%
% You ask, ``In what way does this infinite loop come about?'' Good question!
%
% The setup is a chunk in the MVL that is taller than \cmd\textheight. 
% (Yes, it's that simple.)
% As soon as the previous page ships out, the MVL will contain a mark (propagated from the previous page) followed
% by that large chunk (call it the `big bad box', albeit does not need to be a single box).
% The next visit to the output routine will be a natural page break, but
% \TeX\ will select the juncture between the mark and the big bad box as the least-cost page break.
% Unless the test in \cmd\dead@cycle\ is done, the cycle is perpetuated when the macro 
% reinserts the mark.
%
% The crux matter is achieving, in a robust way, the goal of going from a \cmd\holdinginserts\ state to one
% where the insertions are moving.
%
%    \begin{macrocode}
\def\dead@cycle@repair#1{%
 \expandafter\do@@mark
 \expandafter{%
              \@@botmark
             }%
 \unvbox\@cclv
 \nointerlineskip
 \vbox to#1{\vss}%
 \@ifnum{\outputpenalty<\@M}{\penalty\outputpenalty}{}%
}%
\def\dead@cycle@repair@protected#1{%
 \expandafter\do@@mark
 \expandafter{%
              \@@botmark
             }%
 \begingroup
  \unvbox\@cclv
%    \end{macrocode}
% Remove the protection box
%    \begin{macrocode}
  \remove@lastbox
  \nointerlineskip
  \advance#1-\ht\@protection@box
  \vbox to#1{\vss}%
  \protection@box % Reinsert protection box
  \@ifnum{\outputpenalty<\@M}{\penalty\outputpenalty}{}%
 \endgroup
}%
\def\dead@cycle{%
 \expandafter\do@@mark
 \expandafter{%
              \@@botmark
             }%
 \unvbox\@cclv
 \@ifnum{\outputpenalty<\@M}{\penalty\outputpenalty}{}%
}%
%    \end{macrocode}
% \end{macro}
% \end{macro}
%
% \begin{macro}{\output@init@document}
% \begin{macro}{\output@prep@document}
% \begin{macro}{\output@post@document}
% The default processing simply provides for insertion of held-over footnotes.
% At a natural page break, we are either at the bottom of a column or at the bottom 
% of a page. In either case, the \cmd\output@init@\ processing adjusts for the height
% of the held-over footnotes and bails out.
% Upon our return, at \cmd\output@prep@\ time, the page break will accomodate the material;
% it is now actually inserted by concatenating it with the contents of \cmd\footins.
% The default processing for \cmd\output@post@\ is nil.
% \changes{4.1b}{2008/08/04}{Box \cs{footbox} changed to box \cs{footsofar}}
% \changes{4.1n}{2009/12/02}{More diagnostics of column balancing}
%    \begin{macrocode}
\def\output@init@document{%
 \ltxgrid@info@sw{\class@info{\string\output@init@document}}{}%
 \global\vsize\vsize
}%
%    \end{macrocode}
% QUERY: the following procedure is very like \cmd\combine@foot@inserts. Should it be the same?
% Answer: no, the two differ: this procedure makes a local assignment of \cmd\footins;
% the latter makes a global assignment of \cmd\footsofar.
%
% Note: In a multicolumn document, footnotes must \emph{not} be balanced at this point.
% \changes{4.1n}{2009/12/02}{More diagnostics of column balancing}
%    \begin{macrocode}
\def\output@prep@document{%
 \ltxgrid@foot@info@sw{\class@info{\string\output@prep@document}\trace@scroll{\showbox\footins\showbox\footsofar}}{}%
 \@ifvoid\footsofar{%
 }{%
  \global\setbox\footins\vbox\bgroup
   \unvbox\footsofar
   \@ifvoid\footins{}{%
    \marry@baselines
    \unvbox\footins
   }%
  \egroup
  \ltxgrid@foot@info@sw{\trace@box\footins}{}%
 }%
}%
\def\output@post@document{}%
%    \end{macrocode}
% \end{macro}
% \end{macro}
% \end{macro}
%
% \begin{macro}{\@opcol}
% The standard \LaTeX\ procedure \cmd\@opcol\ is now completely obsoleted.
%    \begin{macrocode}
\let\@opcol\@undefined
%    \end{macrocode}
% \end{macro}
%
% \begin{macro}{\@makecolumn}
% \changes{4.1b}{2008/08/04}{Change \cs{@makecol} to \cs{@makecolumn} with argument}
% The procedure \cmd\@makecolumn\ packages up a page along with all its insertions and floats.
% Therefore it is essential that it be executed with \cmd\holdininserts\ cleared.
%
% Note that there is a corner case when in a multi-column grid, where the change back to 
% one-column grid occurs just after a complete page ships out. We want to detect when
% \cmd\@cclv\ contains nothing but a \cmd\mark, but this is a \TeX\ impossibility.
%
% Note on \cmd\@kludgeins: we have removed this mechanism from \LaTeX, because the implementation 
% of \cmd\enlargethispage\ no longer requires it. 
% Here, for consistency sake, we remove \cmd\@makespecialcolbox.
% 
% The argument of \cmd\@makecolumn\ is a Boolean and determines if we combine the footnote material
% into the present column. If the procedure is building a column for shipping out, then we will
% combine the footnote material, if not, we return with the \cmd\footins\ box unchanged.
%
% \changes{4.1f}{2009/07/14}{(AO, 519) \cs{footins} content must be preserved and reintegrated}
% \changes{4.1n}{2009/12/02}{More diagnostics of column balancing}
% \changes{4.1n}{2010/01/02}{(AO, 571) coding convention: use \cs{bgroup}, \cs{egroup} (instead of braces) when a box is being built}%
% \changes{4.1n}{2010/01/02}{(AO, 571) change calling sequence of \cs{combine@foot@inserts} to expose box registers}%
% I changed the behavior of this procedure in the case where the argument is \cmd\false@sw:
% send the unused footnote material to \cmd\footsofar.
%    \begin{macrocode}
\def\@makecolumn#1{%
 \ltxgrid@foot@info@sw{\class@info{\string\@makecolumn\string#1}}{}%
 \setbox\@outputbox\vbox\bgroup
  \boxmaxdepth\@maxdepth
  \@tempdima\dp\@cclv
  \unvbox\@cclv
  \vskip-\@tempdima
 \egroup
 \xdef\@freelist{\@freelist\@midlist}\global\let\@midlist\@empty
 \show@text@box@size
 \@combinefloats
 #1{%
  \@combineinserts\@outputbox\footins
 }{%
  \combine@foot@inserts\footsofar\footins
 }%
 \set@adj@colht\dimen@
 \count@\vbadness
 \vbadness\@M
 \setbox\@outputbox\vbox to\dimen@\bgroup
  \@texttop
  \dimen@\dp\@outputbox
  \unvbox\@outputbox
  \vskip-\dimen@
  \@textbottom
 \egroup
 \vbadness\count@
 \global\maxdepth\@maxdepth
}%
\let\@makespecialcolbox\@undefined
%    \end{macrocode}
% \end{macro}
%
% \begin{macro}{\@combineinserts}
% The procedure to add the specified insertions to the packaged-up page. 
% All other classes of insertions should also be dealt with at this time. 
%
% Note that the second argument must be a \cmd\newinsert\ register: we access the \cmd\box\ along with the \cmd\skip.
% \changes{4.1n}{2009/12/02}{More diagnostics of column balancing}
% \changes{4.1n}{2010/01/02}{(AO, 571) coding convention: use \cs{bgroup}, \cs{egroup} (instead of braces) when a box is being built}%
% \changes{4.1n}{2010/01/02}{(AO, 571) footnote rule is leaders, so that it may be removed by \cs{vsplit}; mechanism of \cs{kern} signals to indicate footnote height}%
%    \begin{macrocode}
\def\@combineinserts#1#2{%
 \ltxgrid@foot@info@sw{\class@info{\string\@combineinserts\string#1\string#2}\trace@box#2}{}%
 \setbox#1\vbox\bgroup
  \unvbox#1%
  \@ifvoid{#2}{}{%
   \dimen@\ht#2\advance\dimen@\dp#2\advance\dimen@\skip#2%
   \show@box@size{Combining inserts}#2%
   \vskip\skip#2%
%    \end{macrocode}
% The footnote rule is created as leaders, so that it may be removed automatically (via \cmd\vsplit) 
% in the event the footnote is recovered from this column. 
% Note that if \cmd\color@begingroup\ or \cmd\normalcolor\ produce marks, this technique will be confounded. 
%    \begin{macrocode}
   \setbox\z@\vbox{\footnoterule}\dimen@i\ht\z@
   \color@begingroup
   \normalcolor
   \cleaders\box\z@\vskip\dimen@i\kern-\dimen@i
   \csname combine@insert@\the\pagegrid@col\endcsname#2%
   \color@endgroup
%    \end{macrocode}
% The following tells \cmd\recover@column\ the size of the footnotes added here, including the skip glue above. 
%    \begin{macrocode}
   \kern-\dimen@\kern\dimen@
  }%
 \egroup
 \ltxgrid@foot@info@sw{\trace@box#1}{}%
}%
%    \end{macrocode}
% We provide for a layer of abstraction for the laying down of footnotes at the bottom of this column or page. 
% \end{macro}
%
% \begin{macro}{\combine@insert@tw@}
% \begin{macro}{\combine@insert@@ne}
% \begin{macro}{\twocolumn@grid@setup}
% \begin{macro}{\onecolumn@grid@setup}
% \begin{macro}{\columngrid@setup}
% \changes{4.1p}{2010/02/24}{(AO, 583) Provide setup code also for footnotes in a one-column document}%
% The following two definitions cover the cases of a two-column document (with footnotes set on a single-column width),
% and a one-column document.
% However, the case of a two-column document with footnotes set on full text width is not covered. 
% 
% For a document in an overall two-column page grid, execute the commands
% \cmd\twocolumn@grid@setup\ followed by \cmd\open@twocolumn;
% if on the full page width (one-column grid), the command \cmd\onecolumn@grid@setup. 
%
% The following is the way REVTeX does the initialization.
% The procedure \cmd\select@column@grid\ is executed at \cmd\AtBeginDocument\ time;
% the boolean \cmd\twocolumn@sw\ selects between the two alternatives.
% \begin{verbatim}
%\def\select@column@grid{%
% \twocolumn@sw{%
%  \twocolumn@grid@setup
%  \open@twocolumn
% }{%
%  \onecolumn@grid@setup
% }%
%}%
%\appdef\class@documenthook{%
% \select@column@grid
%}%
% \end{verbatim}
%
%    \begin{macrocode}
\def\combine@insert@tw@#1{%
 \compose@footnotes@two#1\@ifvbox{#1}{\unvbox}{\box}#1%
}%
\def\combine@insert@@ne#1{%
 \compose@footnotes@one#1\@ifvbox{#1}{\unvbox}{\box}#1%
}%
\def\twocolumn@grid@setup{%
 \expandafter\let\csname combine@insert@1\endcsname\combine@insert@tw@
 \expandafter\let\csname combine@insert@2\endcsname\combine@insert@@ne
}%
\def\onecolumn@grid@setup{%
 \expandafter\let\csname combine@insert@1\endcsname\combine@insert@@ne
 \expandafter\let\csname combine@insert@2\endcsname\combine@insert@@ne
}%
\let\columngrid@setup\onecolumn@grid@setup
\columngrid@setup
%    \end{macrocode}
% \end{macro}
% \end{macro}
% \end{macro}
% \end{macro}
% \end{macro}
%
% \begin{macro}{\@floatplacement}
% In standard \LaTeX, someone (DPC?) makes the assumption that \cmd\@fpmin\ can be assigned
% locally. This is no longer true now that we ship no more than one page per visit to the output routine.
% We apply a bandaid.
%    \begin{macrocode}
\appdef\@floatplacement{%
 \global\@fpmin\@fpmin
}%
%    \end{macrocode}
% \end{macro}
%
% \begin{macro}{\pagebreak@pen}
% While we are in the way of registering certain penalty values,
% let us register the smallest one that will force a visit to the output routine. 
% However, this penalty will not have an assciated macro: we wish to execute the 
% natural output routine instead.
%
% Note that this penalty is invoked by \cmd\clearpage\ and \cmd\newpage.
%    \begin{macrocode}
\mathchardef\pagebreak@pen=\@M
\expandafter\let\csname output@-\the\pagebreak@pen\endcsname\relax
%    \end{macrocode}
% \end{macro}
%
%
% \subsection{Float placement}%
%
% \begin{macro}{\do@startcolumn@pen}
% The procedure \cmd\do@startcolumn@pen\ is executed as a one-off output routine
% just after a page is shipped out (or, in a multicolumn page grid, a column is salted away).
%
% Its job is to either generate a ``float page'' (in reality a column) for shipping out,
% or to commit deferred floats to the fresh column, concluding with a dead cycle.
% In the former case, we accomodate split footnotes and other insertions (by comparing \cmd\vsize\ and \cmd\pagegoal):
% the floats are spewed onto the page, whereupon \LaTeX's output routine will place the footnotes and ship out,
% iterating the process once again.
%
% Note that when this procedure is invoked, \cmd\box\cmd\@cclv\ still has within it the protection box, so we
% start by removing it. Note also that if there was a split insertion held over from the previous page, the
% insert node will be present in \cmd\box\cmd\@cclv, \emph{prior to} the protection box. For this reason, we cannot 
% just throw away that box, as we might be tempted to do. 
%
% FIXME: where else do we possibly inappropriately discard \cmd\box\cmd\@cclv?
%
% Note that, because a column or page page had previously just been completed, 
% we can assume that there is nothing of importance on the page,
% and because no message is being passed, we can preserve marks in a simple way.
%
% A Note on terminology:
% In a single-column page grid, you might expect that we would execute the procedure \cmd\do@startpage.
% But this is not so.
% \LaTeX\ has a confustion of long standing,
% in which the procedures that handle full-page width floats in a two-column page grid all have in their names
% the string `dbl', which erroneously suggests having something to do with ``double''. It does not:
% when you see `dbl', think ``full page width''.
% \changes{4.1b}{2008/08/04}{Change \cs{set@colroom} to \cs{set@colht}}
%    \begin{macrocode}
\mathchardef\do@startcolumn@pen=10005
\@namedef{output@-\the\do@startcolumn@pen}{\do@startcolumn}%
\def\do@startcolumn{%
 \setbox\@cclv\vbox{\unvbox\@cclv\remove@lastbox\unskip}%
 \clearpage@sw{\@clearfloatplacement}{\@floatplacement}%
 \set@colht
 \@booleanfalse\pfloat@avail@sw
 \begingroup
  \@colht\@colroom
  \@booleanfalse\float@avail@sw
  \@tryfcolumn\test@colfloat
  \float@avail@sw{\aftergroup\@booleantrue\aftergroup\pfloat@avail@sw}{}%
 \endgroup
 \fcolmade@sw{%
  \setbox\@cclv\vbox{\unvbox\@outputbox\unvbox\@cclv}%
%    \end{macrocode}
% Now ask for a return visit, this time with insertions and all.
%    \begin{macrocode}
  \outputpenalty-\pagebreak@pen
  \dead@cycle
 }{%
  \begingroup
   \let\@elt\@scolelt
   \let\reserved@b\@deferlist\global\let\@deferlist\@empty\reserved@b
  \endgroup
  \clearpage@sw{%
   \outputpenalty\@M
  }{%
   \outputpenalty\do@newpage@pen
  }%
  \dead@cycle
 }%
 \check@deferlist@stuck\do@startcolumn
 \set@vsize
}%
\def\@scolelt#1{\def\@currbox{#1}\@addtonextcol}%
\def\test@colfloat#1{%
 \csname @floatselect@sw@\thepagegrid\endcsname#1{}{\@testtrue}%
 \@if@sw\if@test\fi{}{\aftergroup\@booleantrue\aftergroup\float@avail@sw}%
}%
%    \end{macrocode}
% \end{macro}
%
% \begin{macro}{\@addtonextcol}
% We must adjust \cmd\@addtonextcol\ to take held-over inserts into account.
% Now that all deferred floats are queued up together (in order), we must have a way of 
% differentiating them; this is done by the page grid-dependent procedure \cmd\@floatselect@sw@.
%    \begin{macrocode}
\def\@addtonextcol{%
 \begingroup
  \@insertfalse
  \@setfloattypecounts
  \csname @floatselect@sw@\thepagegrid\endcsname\@currbox{%
   \@ifnum{\@fpstype=8 }{}{%
     \@ifnum{\@fpstype=24 }{}{%
       \@flsettextmin
       \@reqcolroom \ht\@currbox
       \advance \@reqcolroom \@textmin
       \advance \@reqcolroom \vsize % take into account split insertions
       \advance \@reqcolroom -\pagegoal
       \@ifdim{\@colroom>\@reqcolroom}{%
         \@flsetnum \@colnum
         \@ifnum{\@colnum>\z@}{%
            \@bitor\@currtype\@deferlist
            \@if@sw\if@test\fi{}{%
              \@addtotoporbot
            }%
         }{}%
       }{}%
     }%
   }%
  }{}%
  \@if@sw\if@insert\fi{}{%
    \@cons\@deferlist\@currbox
  }%
 \endgroup
}%
%    \end{macrocode}
% \end{macro}
%
% \begin{macro}{\do@startpage@pen}
% \begin{macro}{\forcefloats@sw}
% \begin{macro}{\@output@combined@page}
% \begin{macro}{\@sdblcolelt}
% \begin{macro}{\test@dblfloat}
% \begin{macro}{\@if@notdblfloat}
% Similar to \cmd\do@startcolumn,
% the procedure \cmd\do@startpage\ starts up a new page (not column) in a multi-column page grid. 
% It is invoked after a page is shipped out in a multi-column page grid, and
% it commits full-page-width floats to the fresh page, possibly resulting in a float page.
% In implementation, it is similar to \cmd\do@startcolumn, except that
% it commits effectively via \cmd\@addtodblcol\ instead of \cmd\@addtonextcol.
% Note that this procedure will inevitably be followed by \cmd\do@startcolumn.
%
% Some details of the procedure:
%
% We begin by removing the protection box from \cmd\box\cmd\@cclv, then setting the values of the
% float placement parameters appropriately, and resetting \cmd\@colht, \cmd\@colroom, and \cmd\vsize\ to base values.
%
% Next we attempt to compose a float page, a page consisting entirely of floats. If successful,
% we ship out the float page and lay down an interrupt that will send us back here for another try.
%
% If no float page is formed, we attempt to commit full-page-width floats to the text page, and return with a dead cycle.
% We are now ready to compose columns of text.
%
% Note that all floats (both column floats and full-page-width floats) move through a single queue.
% To differentiate between the two, the width of the float is compared to \cmd\textwidth.
% This comparison is encapsulated in the macro \cmd\@if@notdblfloat, which should be used whenever 
% such a determination must be made. This procedure returns a Boolean.
%
% \changes{4.1b}{2008/08/04}{Change \cs{set@colroom} to \cs{set@colht}}
%    \begin{macrocode}
\mathchardef\do@startpage@pen=10006
\@namedef{output@-\the\do@startpage@pen}{\do@startpage}%
\def\do@startpage{%
 \setbox\@cclv\vbox{\unvbox\@cclv\remove@lastbox\unskip}%
 \clearpage@sw{\@clearfloatplacement}{\@dblfloatplacement}%
 \set@colht
 \@booleanfalse\pfloat@avail@sw
 \begingroup
  \@booleanfalse\float@avail@sw
  \@tryfcolumn\test@dblfloat
  \float@avail@sw{\aftergroup\@booleantrue\aftergroup\pfloat@avail@sw}{}%
 \endgroup
 \fcolmade@sw{%
  \global\setbox\pagesofar\vbox{\unvbox\pagesofar\unvbox\@outputbox}%
  \@output@combined@page
 }{%
  \begingroup
   \@booleanfalse\float@avail@sw
   \let\@elt\@sdblcolelt
   \let\reserved@b\@deferlist\global\let\@deferlist\@empty\reserved@b
  \endgroup
  \@ifdim{\@colht=\textheight}{% No luck...
   \pfloat@avail@sw{% ...but a float *was* available!
    \forcefloats@sw{%
     \ltxgrid@warn{Forced dequeueing of floats stalled}%
    }{%
     \ltxgrid@warn{Dequeueing of floats stalled}%
    }%
   }{}%
  }{}%
  \outputpenalty\@M
  \dead@cycle
 }%
 \check@deferlist@stuck\do@startpage
 \set@colht
}%
%    \end{macrocode}
%
% Procedure \cmd\@output@combined@page\ is a utility that ships out a page consisting of the
% result of \cmd\@combinepage\ and \cmd\@combinedblfloats, after which it prepares for 
% the process to repeat.
%
% It is coincidentally identical to what needs to happen with 
% a float page that has been built by \cmd\@tryfcolumn, in the multi-column page grid, 
% and also handles the case where a page needs to be shipped out when in multicolumn mode.
% \changes{4.1b}{2008/08/04}{New procedure \cs{@output@combined@page}}
% \changes{4.1b}{2008/08/04}{Change \cs{@combinepage} to \cs{@combinepage} with argument}
%    \begin{macrocode}
\def\@output@combined@page{%
 \@combinepage\true@sw
 \@combinedblfloats
 \@outputpage
 \global\pagegrid@cur\@ne
 \protect@penalty\do@startpage@pen
}%
\def\@sdblcolelt#1{\def\@currbox{#1}\@addtodblcol}%
\def\test@dblfloat#1{%
 \@if@notdblfloat{#1}{\@testtrue}{}%
 \@if@sw\if@test\fi{}{\aftergroup\@booleantrue\aftergroup\float@avail@sw}%
}%
\def\@if@notdblfloat#1{\@ifdim{\wd#1<\textwidth}}%
\@booleanfalse\forcefloats@sw
%    \end{macrocode}
% \end{macro}
% \end{macro}
% \end{macro}
% \end{macro}
% \end{macro}
% \end{macro}
%
% \begin{macro}{\@addtodblcol}
%
% The procedure \cmd\@addtodblcol\ is called into play at the beginning of each fresh page
% and operates on each deferred float, in the hopes of placing one or more such floats
% at the top of the current page.
% 
% We alter the procedure of standard \LaTeX\ by putting failed floats into 
% \cmd\@deferlist\ instead of \cmd\@dbldeferlist. Having done so, we must have a means
% of differentiating full-page-width floats from column-width floats.
% We assume that the latter will always be narrower than \cmd\textwidth.
%
% In aid of detecting a stalled float flushing process, we set a Boolean if we encounter
% a qualified full-page-width float here. Any that qualify but fail the rest of the tests
% might still pass when reconsidered on an otherwise blank page.
%    \begin{macrocode}
\def\@addtodblcol{%
 \begingroup
  \@if@notdblfloat{\@currbox}{%
   \false@sw
  }{%
   \@setfloattypecounts
   \@getfpsbit \tw@
   \@bitor \@currtype \@deferlist
   \@if@sw\if@test\fi{%
    \false@sw
   }{%
    \@ifodd\@tempcnta{%
     \aftergroup\@booleantrue\aftergroup\float@avail@sw
     \@flsetnum \@dbltopnum
     \@ifnum{\@dbltopnum>\z@}{%
       \@ifdim{\@dbltoproom>\ht\@currbox}{%
        \true@sw
       }{%
        \@ifnum{\@fpstype<\sixt@@n}{%
         \begingroup
          \advance \@dbltoproom \@textmin
          \@ifdim{\@dbltoproom>\ht\@currbox}{%
           \endgroup\true@sw
          }{%
           \endgroup\false@sw
          }%
        }{%
         \false@sw
        }%
       }%
     }{%
      \false@sw
     }%
    }{%
     \false@sw
    }%
   }%
  }%
  {%
   \@tempdima -\ht\@currbox
   \advance\@tempdima
    -\@ifx{\@dbltoplist\@empty}{\dbltextfloatsep}{\dblfloatsep}%
   \global \advance \@dbltoproom \@tempdima
   \global \advance \@colht \@tempdima
   \global \advance \@dbltopnum \m@ne
   \@cons \@dbltoplist \@currbox
  }{%
   \@cons \@deferlist \@currbox
  }%
 \endgroup
}%
%    \end{macrocode}
% \end{macro}
%
% \begin{macro}{\@tryfcolumn}
% \begin{macro}{\@wtryfc}
% \begin{macro}{\@xtryfc}
% \begin{macro}{\@ztryfc}
% Whenever a page is shipped out, \LaTeX\ automatically tries out a float column:
% a page containing nothing but floats (and, as we have added here, split footnotes).
%
% The following four procedures employ certain macros to communicate between each other:
%
% \cmd\fcolmade@sw, a boolean, says whether we were successful in making a float column.
%
% \cmd\if@test, a \cmd\newif\ switch, says a float has failed some test.
% 
% \cmd\@deferlist, is the input to the process, a list, of deferred floats.
% 
% \cmd\@trylist, a list, stores the deferred floats to be tried out on the float column.
%
% \cmd\@failedlist, a list of floats that have failed the selection for the float column.
%
% \cmd\@flfail, a list of floats that have failed the second selection for the float column.
%
% \cmd\@flsucceed, a list, the floats that have been successfully placed on the float column.
%
% \cmd\@freelist, a list, receives any freed floats.
%
% \cmd\@colht, a dimen, the available space for the column, including column floats and insertions (footnotes).
%
% \cmd\@fpmin, a dimen, the required minimum height for the float column.
%
% \cmd\@outputbox, a box, the output of the process.
%
% \cmd\@fptop, \cmd\@fpsep, \cmd\@fpbot, glue, placed above, between, and below floats on the float column.
%
% \cmd\@currtype, a count, used temporarily for the float's bits.
%
% \cmd\@tempcnta, a count, used temporarily for the float's bits.
%
% In \cmd\@tryfcolumn, we alter the criterion for a float page, because if footnotes are present at this point
% (presumably due to a split insertion) then \cmd\@fpmin is no longer the right threshold to apply.
% 
% Note that we have changed \cmd\@tryfcolumn, \cmd\@xtryfc, and \cmd\@ztryfc\ syntactically so that the procedure
% to test for the float's being a column float versus a full-page-width float is passed in as an 
% argument.
%
%    \begin{macrocode}
\def\@tryfcolumn#1{%
  \global\@booleanfalse\fcolmade@sw
  \@ifx@empty\@deferlist{}{%
    \global\let\@trylist\@deferlist
    \global\let\@failedlist\@empty
    \begingroup
      \dimen@\vsize\advance\dimen@-\pagegoal\@ifdim{\dimen@>\z@}{%
       \advance\@fpmin-\dimen@
      }{}%
      \def\@elt{\@xtryfc#1}\@trylist
    \endgroup
    \fcolmade@sw{%
      \global\setbox\@outputbox\vbox{\vskip \@fptop}%
      \let \@elt \@wtryfc \@flsucceed
      \global\setbox\@outputbox\vbox{\unvbox\@outputbox
        \unskip \vskip \@fpbot
      }%
      \let \@elt \relax
      \xdef\@deferlist{\@failedlist\@flfail}%
      \xdef\@freelist{\@freelist\@flsucceed}%
    }{}%
  }%
}%
\def\@wtryfc #1{%
  \global\setbox\@outputbox\vbox{\unvbox\@outputbox
    \box #1\vskip\@fpsep
  }%
}%
\def\@xtryfc#1#2{%
  \@next\reserved@a\@trylist{}{}% trim \@trylist. Ugly!
  \@currtype \count #2%
  \divide\@currtype\@xxxii\multiply\@currtype\@xxxii
  \@bitor \@currtype \@failedlist
  \@testfp #2%
  #1#2%
  \@ifdim{\ht #2>\@colht   }{\@testtrue}{}%
  \@if@sw\if@test\fi{%
   \@cons\@failedlist #2%
  }{%
   \begingroup
     \gdef\@flsucceed{\@elt #2}%
     \global\let\@flfail\@empty
     \@tempdima\ht #2%
     \def \@elt {\@ztryfc#1}\@trylist
     \@ifdim{\@tempdima >\@fpmin}{%
       \global\@booleantrue\fcolmade@sw
     }{%
       \@cons\@failedlist #2%
     }%
   \endgroup
   \fcolmade@sw{%
     \let \@elt \@gobble
   }{}%
  }%
}%
\def\@ztryfc #1#2{%
  \@tempcnta \count#2%
  \divide\@tempcnta\@xxxii\multiply\@tempcnta\@xxxii
  \@bitor \@tempcnta {\@failedlist \@flfail}%
  \@testfp #2%
  #1#2%
  \@tempdimb\@tempdima
  \advance\@tempdimb \ht#2\advance\@tempdimb\@fpsep
  \@ifdim{\@tempdimb >\@colht}{%
    \@testtrue
  }{}%
  \@if@sw\if@test\fi{%
    \@cons\@flfail #2%
  }{%
    \@cons\@flsucceed #2%
    \@tempdima\@tempdimb
  }%
}%
%    \end{macrocode}
% \end{macro}
% \end{macro}
% \end{macro}
% \end{macro}
%
%
% \subsection{Clearing pages}%
% 
% Clearing the page is an elaboration of ending the page: it entails flushing all floats.
%
% This package might make number of float flushing algorithms available,
% a very simple one that does not try to produce excellent pages,
% another that tries to make the best use of space,
% and a more complex one that tries to balance columns.
% 
% At the beginning of the page-clearing process, by definition all of the paragraph text involved is on the MVL and all floats have been encountered.
% There may be material in \cmd\pagesofar, and (in a multi-column page grid) any number of columns of the page have been composed.
% Also, there might be footnote material saved up in \cmd\footsofar.
% 
% Because we did not want to perform multiple \cmd\shipout s per visit to the output routine,
% our multi-column page makeup will not compose multiple columns per visit.
% This implementation detail may not require alteration, but it is not a limitation that is truly necessary:
% it is only multiple \cmd\shipout s per visit that must be avoided.
% 
% The crux matter is how to continue with flushing floats even after the material in the MVL is exhausted.
% At that point, we must, upon completion of the output routine, 
% insert into the MVL an interrupt that triggers the next step in the processing.
%
% Therefore, after processing a \cmd\do@startcolumn\ interrupt, we must somehow force the completion of that column.
% This could be done by inserting a \cmd\do@newpage@pen\ interrupt.
% 
% And after processing a \cmd\do@startpage@pen\ interrupt, that results in \cmd\@dbltopinsert s,
% we must ensure that the multiple columns on the page get completed, so that the page itself finally gets shipped out.
% This part will proceed automatically given that \cmd\do@startcolumn\ processing completes successfully.
% 
% The process will not be complete until all deferred floats have been placed and shipped out, and all saved-up footnotes have been inserted.
% 
% Full-page-width floats can get out of order of column floats. This problem can be remedied by holding them all in the same list.
% We therefore stop using \cmd\@dbldeferlist\ entirely, and all of the procedures that formerly used it have been rewritten to
% use \cmd\@deferlist\ instead. When traversing the list, we apply a selector on the given box that determines whether it is a column-width or page-width float.
% This selector is different depending on the page grid.
%
% When the \cmd\@deferlist\ is processed (by any means), we have to take care of the case where a float of one category is passed over but we are looking for a float of the other category.
% Here, we must terminate processing, to avoid disordering the floats. This we do by the usual means.
% 
% The system has a Boolean that says we are clearing pages: \cmd\clearpage@sw; if it is true,
% then at the tail of \cmd\do@startcolumn\ processing, we should put down a (\cmd\vfil?) \cmd\do@newpage@pen\ interrupt.
% This is because the MVL is now empty, so we have to force the columns to complete.
% 
% One potential very pathological case would be where there is one or more deferred floats that never successfully get placed:
% placing floats has stalled, and we will ship out blank pages indefinitely. How to detect this case?
% 
% First, \cmd\do@startpage\ will evidently be stalled if the following are all true:
% a) \cmd\@tryfcolumn\ and \cmd\@sdblcolelt\ both fail,
% b) there are deferred floats available for page placement, and
% c) the \cmd\@colht=\cmd\textheight, that is, the full page height is available for placement of column floats.
% 
% Second, \cmd\do@startcolumn\ will evidently be stalled if the following are all true:
% a) tryfcolumn fails,
% b) there are deferred floats available for column placement, and
% a) the \cmd\@colroom=\cmd\textheight, that is, the full page height is available for placement of column floats.
% 
%
% \begin{macro}{\cleardoublepage}
% \begin{macro}{\clearpage}
% \begin{macro}{\newpage}
% \begin{macro}{\newpage@prep}
% The function of \cmd\clearpage\ is to end the current page with \cmd\newpage\ and then
% ship out additional pages until (\footins) inserts and (deferred) floats are exhausted.
%
% The method involves setting the float placement parameters to completely permissive values
% and kicking out the current page (using a non-discardable penalty).
% A possibly short page will be shipped out, followed
% by any number of float pages. However these float pages, because using permissive float placement,
% will exhaust all inserts and deferred floats.
%
% Bug Note: in the code for \cmd\clearpage, the first penalty we output is an unprotected \cmd\pagebreak@pen.
% I tried using a protected \cmd\do@newpage@pen, but that gave rise to a corner case where a blank page
% was output.
%
% At present, the \cmd\clearpage\ procedure does the same as \cmd\newpage,
% except that \cmd\clearpage@sw\ is turned on,
% and the (discardable) \cmd\newpage\ is inevitably followed by the same procedures
% that are executed if a page is shipped out.
%
% FIXME: it seems that better than \cmd\pagebreak@pen\ would be an unprotected penalty of a special value that would
% entail output routine processing consisting of the following steps:
% 3) \cmd\unvbox\cmd\@cclv,
% 1) set \cmd\clearpage@sw\ to \cmd\true@sw,
% 2) put down a protected \cmd\do@startcolumn@pen,
% 4) take a dead cycle.
%
% The effect would be to liberalize float placement options for the current column as well as further columns that may be output as part of \cmd\clearpage\ processing.
% Of course, it would still be necessary to set \cmd\clearpage@sw\ again via an interrupt.
%
% An optimization might be to clear \cmd\clearpage@sw\ as part of the same interrupt,
% but that would actually not work properly, because it is necessary for \cmd\do@endpage\ to
% possibly invoke further visits to the output routine before clearpage processing ceases.
%    \begin{macrocode}
\def\newpage@prep{%
  \if@noskipsec
    \ifx \@nodocument\relax
      \leavevmode
      \global \@noskipsecfalse
    \fi
  \fi
  \if@inlabel
    \leavevmode
    \global \@inlabelfalse
  \fi
  \if@nobreak \@nobreakfalse \everypar{}\fi
  \par
}%
\def \newpage {%
 \newpage@prep
 \do@output@MVL{%
  \vfil
  \penalty-\pagebreak@pen
 }%
}%
\def\clearpage{%
 \newpage@prep
 \do@output@MVL{%
  \vfil
  \penalty-\pagebreak@pen
  \global\@booleantrue\clearpage@sw
  \protect@penalty\do@startcolumn@pen
  \protect@penalty\do@endpage@pen
 }%
 \do@output@MVL{%
  \global\@booleanfalse\clearpage@sw
 }%
}%
\def\cleardoublepage{%
 \clearpage
 \@if@sw\if@twoside\fi{%
  \@ifodd\c@page{}{%
   \null\clearpage
  }%
 }{}%
}%
\@booleanfalse\clearpage@sw
%    \end{macrocode}
% \end{macro}
% \end{macro}
% \end{macro}
% \end{macro}
%
% \begin{macro}{\do@endpage@pen}
% The penalty \cmd\do@endpage@pen\ simply dispatches to the page grid procedure that forces an end page.
% That procedure should test whether there is anything to ship out (say committed floats), then act accordingly.
% Note that as part of this work, it should \cmd\unvbox\cmd\@cclv, which has been left boxed up so it can be 
% measured.
%    \begin{macrocode}
\mathchardef\do@endpage@pen=10007
\@namedef{output@-\the\do@endpage@pen}{\csname end@column@\thepagegrid\endcsname}%
%    \end{macrocode}
% \end{macro}
%
% \begin{macro}{\do@newpage@pen}
% The penalty \cmd\do@newpage@pen\ allows a ``non-discardable \cmd\newpage'' command:
% a \cmd\newpage\ command that will not disappear at a pagebreak.
% This visit to the output routine will not be dispatched to an interrupt,
% rather the natural output routine will be executed, where it
% will remove the protection box.
% 
% Call this routine by executing \cmd\protect@penalty\cmd\do@newpage@pen.
%    \begin{macrocode}
\mathchardef\do@newpage@pen=10001
\expandafter\let\csname output@-\the\do@newpage@pen\endcsname\relax
%    \end{macrocode}
% \end{macro}
%
% \begin{macro}{\@clearfloatplacement}
% The procedure \cmd\@clearfloatplacement\ sets the float placement parameters
% to completely permissive values (except for \cmd\@fpmin). The standard values are:
%
% \begin{tabular}{ll}
%\cmd\@topnum      &\cmd\c@topnumber\\
%\cmd\@toproom     &\cmd\topfraction\cmd\@colht\\
%\cmd\@botnum      &\cmd\c@bottomnumber\\
%\cmd\@botroom     &\cmd\bottomfraction\cmd\@colht\\
%\cmd\@colnum      &\cmd\c@totalnumber\\
%\cmd\@fpmin       &\cmd\floatpagefraction\cmd\@colht\\
%\cmd\@dbltopnum   &\cmd\c@dbltopnumber\\
%\cmd\@dbltoproom  &\cmd\dbltopfraction\cmd\@colht\\
%\cmd\@textmin     &\cmd\@colht\cmd\advance\cmd\@textmin -\cmd\@dbltoproom\\
%\cmd\@fpmin       &\cmd\dblfloatpagefraction\cmd\textheight
% \end{tabular}
%    \begin{macrocode}
\def\@clearfloatplacement{%
 \global\@topnum     \maxdimen
 \global\@toproom    \maxdimen
 \global\@botnum     \maxdimen
 \global\@botroom    \maxdimen
 \global\@colnum     \maxdimen
 \global\@dbltopnum  \maxdimen
 \global\@dbltoproom \maxdimen
 \global\@textmin    \z@      
 \global\@fpmin      \z@      
 \let\@testfp\@gobble
 \appdef\@setfloattypecounts{\@fpstype16\advance\@fpstype\m@ne}%
}%
%    \end{macrocode}
% \end{macro}
%
% \begin{macro}{\@doclearpage}
% \begin{macro}{\@makefcolumn}
% \begin{macro}{\@makecol}
% The \cmd\@doclearpage\ procedure is now obsoleted, as is \cmd\@makefcolumn, which it invoked.
% We also completely avoid using \cmd\@makecol\ (in favor of \cmd\@makecolumn).
%    \begin{macrocode}
\let\@doclearpage\@undefined
\let\@makefcolumn\@undefined
\let\@makecol\@undefined
%    \end{macrocode}
% \end{macro}
% \end{macro}
% \end{macro}
%
% \begin{macro}{\clr@top@firstmark}
% \begin{macro}{\set@top@firstmark}
% \begin{macro}{\@outputpage@tail}
% We want accurate values of \cmd\topmark\ and \cmd\firstmark, but we must deal with 
% the fact that there are many different ways of contributing material
% to the page. Only upon the first contribution to the page is the value of \cmd\topmark\
% accurate. However, with \cmd\firstmark\ we must potentially examine each contribution
% because the first mark on the page may happen to fall in the last piece of material contributed.
%
% To begin, we define the procedure that initializes the macros to appropriate flag values.
%    \begin{macrocode}
\def\clr@top@firstmark{%
 \global\let\saved@@topmark\@undefined
 \global\let\saved@@firstmark\@empty
 \global\let\saved@@botmark\@empty
}%
\clr@top@firstmark
%    \end{macrocode}
%
% Note that the flag value for \cmd\saved@@topmark\ is \cmd\@undefined, just as one would
% expect. But that for \cmd\saved@@firstmark\ and \cmd\saved@@botmark\ is \cmd\@empty.
%
% Next, we define procedure \cmd\set@top@firstmark; it will be exercised everywhere material is contributed,
% capturing the mark values if appropriate.
%    \begin{macrocode}
\def\set@top@firstmark{%
 \@ifxundefined\saved@@topmark{\expandafter\gdef\expandafter\saved@@topmark\expandafter{\@@topmark}}{}%
 \@if@empty\saved@@firstmark{\expandafter\gdef\expandafter\saved@@firstmark\expandafter{\@@firstmark}}{}%
 \@if@empty\@@botmark{}{\expandafter\gdef\expandafter\saved@@botmark\expandafter{\@@botmark}}%
}%
%    \end{macrocode}
% When should \cmd\set@top@firstmark\ be called?
% A good candidate for a universal procedure for handling contributed material is
% the natural output routine; are any other calls needed?
%
% Yes, in \cmd\save@column\ we must execute \cmd\set@top@firstmark\ because we are about to 
% save away \cmd\box\cmd\@cclv, and we will never see its marks again (unless it is unboxed into the MVL),
% because \TeX\ lets one access a box's marks only within an output routine that has put that box into \cmd\box\cmd\@cclv.
%
% As soon as a page is shipped out, we initialize the two macros that
% hold the values of \cmd\topmark\ and \cmd\firstmark, respectively.
% \changes{4.1b}{2008/08/04}{Procedure \cs{@outputpage@tail} tailpatches \cs{@outputpage}}%
%    \begin{macrocode}
\appdef\@outputpage@tail{%
 \clr@top@firstmark
}%
%    \end{macrocode}
% \end{macro}
% \end{macro}
% \end{macro}
%
%
% \subsection{Other interfaces to \LaTeX}%
%
% \begin{macro}{\@float}
% \begin{macro}{\@dblfloat}
% \begin{macro}{\@yfloat}
% \begin{macro}{\fps@}
% \begin{macro}{\fpsd@}
% The \LaTeX\ kernel procedures \cmd\@float\ and \cmd\@dblfloat\ are treated on an equal footing.
% Each now takes environment-specific float placement defaults.
% If none are defined for the calling environment, we apply a default.
%
% A parameter is passed that will set the width of text within the float, normally \cmd\columnwidth,
% and in the "dbl" version, \cmd\textwidth. However, an environment such as \env{turnpage}
% may change the meanings of these macros to allow turnpage floats.
%
% \changes{4.0a}{2008/06/26 }{\cs{@yfloat}: de-fang \cs{set@footnotewidth} (see ltxutil.dtx): we have already done its job. }
%
% Note on \cmd\@xfloat: the optional argument must come to it fully expanded, because the macro
% does a weird procedure on this argument, involving \cmd\@onelevel@sanitize, which I do not
% understand, and which does not work if not so expanded. 
% \changes{4.1b}{2008/08/04}{Get rid of the \cs{reserved@a} idiom}
%    \begin{macrocode}
\def\@float#1{%
 \@ifnextchar[{%
%    \end{macrocode}
% \verb+}]{+Brace-matching klootch
%    \begin{macrocode}
  \@yfloat\width@float{#1}%
 }{%
  \@ifxundefined@cs{fps@#1}{}{\expandafter\let\expandafter\fps@\csname fps@#1\endcsname}%
  \expandafter\@argswap\expandafter{\expandafter[\fps@]}{\@yfloat\width@float{#1}}%
 }%
}%
\def\@dblfloat#1{%
 \@ifnum{\pagegrid@col=\@ne}{%
  \@float{#1}%
 }{%
  \@ifnextchar[{%
%    \end{macrocode}
% \verb+}]{+Brace-matching klootch
%    \begin{macrocode}
   \@yfloat\widthd@float{#1}%
  }{%
   \@ifxundefined@cs{fpsd@#1}{}{\expandafter\let\expandafter\fpsd@\csname fpsd@#1\endcsname}%
   \expandafter\@argswap\expandafter{\expandafter[\fpsd@]}{\@yfloat\widthd@float{#1}}%
  }%
 }%
}%
%    \end{macrocode}
% \changes{4.1b}{2008/08/04}{Turn off the \cs{set@footnotewidth} mechanism; a float `knows' its proper typesetting context}
% \changes{4.1n}{2010/01/02}{(AO, 571) Interface \cs{set@footnotewidth} for determining the set width of footnotes}%
% \cmd\@yfloat\ is the go-to procdure for creating the proper environment for the content of a float.
% Argument \verb+#1+ is the width of the float environment (we disable \cmd\set@footnotewidth), and 
% we establish a self-contained (minipage) environment for footnotes. 
%    \begin{macrocode}
\def\@yfloat#1#2[#3]{%
 \@xfloat{#2}[#3]%
 \hsize#1\linewidth\hsize
 \let\set@footnotewidth\@empty
 \minipagefootnote@init
}%
\def\fps@{tbp}%
\def\fpsd@{tp}%
\def\width@float{\columnwidth}%
\def\widthd@float{\textwidth}%
%    \end{macrocode}
% \end{macro}
% \end{macro}
% \end{macro}
% \end{macro}
% \end{macro}
%
% \begin{macro}{\end@float}
% \begin{macro}{\end@dblfloat}
% \begin{macro}{\end@@float}
% \begin{macro}{\check@currbox@count}
% \begin{macro}{\minipagefootnote@init}
% \begin{macro}{\minipagefootnote@here}
% \LaTeX\ kernel procedures \cmd\end@float\ and \cmd\end@dblfloat\
% have been changed to work alike; in particular, floats of both classes
% are deferred into the same queue.
% This measure ensures that they will be placed in their original order, 
% an aspect in which \LaTeX\ is broken.
%
% Note: when retrieving floats from the queues, we can differentiate those of the two categories
% by the width of the box.
%
% Floats are processed via an output routine message, and are checked for 
% sanity in re the float placement options. In the case of full-page-width floats,
% we ensure that the h and b float placement options are never asserted, because they
% make no sense.
%
% Note that if we get to the end of the float box and still have pending
% footnotes, we put then out.
%
% LaTeX Bug note: if a user types \cmd\begin{table*}[h], the float will never succeed in being placed!
% we try to catch such cases.
%
% Note that the macro \cmd\check@currbox@count\ tries to catch cases where the float placement options
% are such that the float can never be placed.
%
% The calls to \cmd\@iffpsbit\ are part of a procedure to deny certain of the float placement
% parameters: ``h'' and ``b'' are not possible, the former because the \cmd\marginpar\ mechanism
% cannot place a full-page-width float within a multicolumn page grid, the latter because nobody
% has yet written the code to do so (pretty bad reason, I know).
% \changes{4.1b}{2008/08/04}{New procedure \cs{@iffpsbit} replaces \cs{@getfpsbit}}
%    \begin{macrocode}
\def\end@float{%
 \end@@float{%
  \check@currbox@count
 }%
}%
\def\end@dblfloat{%
 \@ifnum{\pagegrid@col=\@ne}{%
  \end@float
 }{%
  \end@@float{%
   \@iffpsbit\@ne{\global\advance\count\@currbox\m@ne}{}%
   \@iffpsbit\f@ur{\global\advance\count\@currbox-4\relax}{}%
   \global\wd\@currbox\textwidth % Klootch
   \check@currbox@count
  }%
 }%
}%
\def\end@@float#1{%
 \minipagefootnote@here
 \@endfloatbox
 #1%
 \@ifnum{\@floatpenalty <\z@}{%
  \@largefloatcheck
  \@cons\@currlist\@currbox
  \@ifnum{\@floatpenalty <-\@Mii}{%
   \do@output@cclv{\@add@float}%
  }{%
   \vadjust{\do@output@cclv{\@add@float}}%
   \@Esphack
  }%
 }{}%
}%
%    \end{macrocode}
% The \classname{float} package of Anselm Lingnau fails when used under \classname{ltxgrid}, but
% we can fix things. 
% We also repair a bug in that package. 
%
% \changes{4.1g}{2009/10/06}{(AO, 531) Fix package \classname{float} }
%    \begin{macrocode}
\newcommand\float@end@float{%
 \@endfloatbox
 \global\setbox\@currbox\float@makebox\columnwidth
 \let\@endfloatbox\relax
 \end@float
}%
\newcommand\float@end@ltx{%
 \end@@float{%
  \global\setbox\@currbox\float@makebox\columnwidth
  \check@currbox@count
 }%
}%
\newcommand\newfloat@float[3]{%
 \@namedef{ext@#1}{#3} %!
 \let\float@do=\relax
 \xdef\@tempa{\noexpand\float@exts{\the\float@exts \float@do{#3}}}%
 \@tempa
 \floatplacement{#1}{#2}%
 \@ifundefined{fname@#1}{\floatname{#1}{#1}}{} %!
 \expandafter\edef\csname ftype@#1\endcsname{\value{float@type}}%
 \addtocounter{float@type}{\value{float@type}} %!
 \restylefloat{#1}%
 \expandafter\edef\csname fnum@#1\endcsname{%
  \expandafter\noexpand\csname fname@#1\endcsname{} %!
  \expandafter\noexpand\csname the#1\endcsname
 }
 \@ifnextchar[%]
  {%
   \float@newx{#1}%
  }{%
   \@ifundefined{c@#1}{\newcounter{#1}\@namedef{the#1}{\arabic{#1}}}{}%
  }%
}%
\newcommand\newfloat@ltx[3]{%
 \@namedef{ext@#1}{#3}%
 \let\float@do=\relax
 \xdef\@tempa{\noexpand\float@exts{\the\float@exts \float@do{#3}}}%
 \@tempa
 \floatplacement{#1}{#2}%
 \@ifundefined{fname@#1}{\floatname{#1}{#1}}{}%
 \expandafter\edef\csname ftype@#1\expandafter\endcsname\expandafter{\the\c@float@type}%
 \addtocounter{float@type}{\value{float@type}}%
 \restylefloat{#1}%
 \expandafter\edef\csname fnum@#1\endcsname{%
  \expandafter\noexpand\csname fname@#1\endcsname{}%
  \expandafter\noexpand\csname the#1\endcsname
 }
 \@ifnextchar[%]
  {%
   \float@newx{#1}%
  }{%
   \@ifundefined{c@#1}{\newcounter{#1}\@namedef{the#1}{\arabic{#1}}}{}%
  }%
}%
\appdef\document@inithook{%
 \@ifxundefined\newfloat{}{%
  \@ifx{\float@end\float@end@float}{%
   \@ifx{\newfloat\newfloat@float}{\true@sw}{\false@sw}%
   }{\false@sw}% 
   {%
    \class@warn{Repair the float package}%
    \let\float@end\float@end@ltx
    \let\newfloat\newfloat@ltx
   }{%
    \class@warn{Failed to patch the float package}%
   }%
 }% 
}%
%    \end{macrocode}
%
% Boolean procedure \cmd\@iffpsbit\ is similar to the \cmd\@getfpsbit\ of \LaTeX, except
% that we do not expose the scratch count register or even change its value.
%    \begin{macrocode}
\def\@iffpsbit#1{%
 \begingroup
  \@tempcnta\count\@currbox
  \divide\@tempcnta#1\relax
  \@ifodd\@tempcnta{\aftergroup\true@sw}{\aftergroup\false@sw}%
 \endgroup
}%
%    \end{macrocode}
%
% In procedure \cmd\check@currbox@count, we calculate the net float placement
% directive (encoded into \cmd\count\ \cmd\@currbox's least significant four bits).
% If zero, issue a warning.
% \changes{4.1b}{2008/08/04}{Tally the height of the float}
%    \begin{macrocode}
\def\check@currbox@count{%
 \@ifnum{\count\@currbox>\z@}{%
  \count@\count\@currbox\divide\count@\sixt@@n\multiply\count@\sixt@@n
  \@tempcnta\count\@currbox\advance\@tempcnta-\count@
  \@ifnum{\@tempcnta=\z@}{%
   \ltxgrid@warn{Float cannot be placed}%
  }{}%
  \expandafter\tally@float\expandafter{\@captype}%
 }{%
%    \end{macrocode}
% In this case, the float is a \cmd\marginpar.
%    \begin{macrocode}
 }%
}%
\providecommand\minipagefootnote@init{}%
\providecommand\minipagefootnote@here{}%
\providecommand\tally@float[1]{}%
%    \end{macrocode}
% \end{macro}
% \end{macro}
% \end{macro}
% \end{macro}
% \end{macro}
% \end{macro}
%
% \begin{macro}{\@specialoutput}
% The \cmd\@add@float\ procedure used to reside in standard \LaTeX's \cmd\@specialoutput,
% which is no more.
%
% Historical Note: \cmd\@specialoutput\ and Lamport's method of an output routine dispatcher
% is the genesis of our more powerful and refined way of using \TeX's output routine to 
% safely accomplish page makeup tasks. To it and to him we owe acknowledgement and thanks.
%    \begin{macrocode}
\let\@specialoutput\@undefined
%    \end{macrocode}
% \end{macro}
%
% \begin{macro}{\@add@float}
% In the following, we do not need to execute \cmd\@reinserts, which was wrong anyway, as you cannot
% reliably recover insertions when they split (unless you have a way of reinserting the captured insertion
% ahead of the split-off part).
%
% Now that full-page-width floats are being processed the same as column floats, we
% have to nip in here and cause them always to be deferred.
%
% At the very end, the \cmd\vsize\ is adjusted for any newly committed float.
%    \begin{macrocode}
\def\@add@float{%
 \@pageht\ht\@cclv\@pagedp\dp\@cclv
 \unvbox\@cclv
 \@next\@currbox\@currlist{%
  \csname @floatselect@sw@\thepagegrid\endcsname\@currbox{%
   \@ifnum{\count\@currbox>\z@}{%
    \advance \@pageht \@pagedp
%    \end{macrocode}
% Do not assume \cmd\holdinginserts is cleared:
%    \begin{macrocode}
    \advance \@pageht \vsize \advance \@pageht -\pagegoal
%    \end{macrocode}
% Commit an `h' float:
%    \begin{macrocode}
    \@addtocurcol
   }{%
    \@addmarginpar
   }%
  }{%
   \@resethfps
   \@cons\@deferlist\@currbox
  }%
 }{\@latexbug}%
 \@ifnum{\outputpenalty<\z@}{%
  \@if@sw\if@nobreak\fi{%
   \nobreak
  }{%
   \addpenalty \interlinepenalty
  }%
 }{}%
 \set@vsize
}%
%    \end{macrocode}
% \end{macro}
%
% \begin{macro}{\@reinserts}
% The \cmd\@reinserts\ procedure of standard \LaTeX\ is now obsoleted (it had been erroneous anyway).
%    \begin{macrocode}
\let\@reinserts\@undefined
%    \end{macrocode}
% \end{macro}
%
% \begin{macro}{\@addtocurcol}
% We modify the \cmd\@addtocurcol\ procedure of standard \LaTeX\ 
% so that a float placed ``here'' may break over pages.
%    \begin{macrocode}
\def \@addtocurcol {%
   \@insertfalse
   \@setfloattypecounts
   \ifnum \@fpstype=8
   \else
     \ifnum \@fpstype=24
     \else
       \@flsettextmin
       \advance \@textmin \@textfloatsheight
       \@reqcolroom \@pageht
       \ifdim \@textmin>\@reqcolroom
         \@reqcolroom \@textmin
       \fi
       \advance \@reqcolroom \ht\@currbox
       \ifdim \@colroom>\@reqcolroom
         \@flsetnum \@colnum
         \ifnum \@colnum>\z@
           \@bitor\@currtype\@deferlist
           \if@test
           \else
             \@bitor\@currtype\@botlist
             \if@test
               \@addtobot
             \else
               \ifodd \count\@currbox
                 \advance \@reqcolroom \intextsep
                 \ifdim \@colroom>\@reqcolroom
                   \global \advance \@colnum \m@ne
                   \global \advance \@textfloatsheight \ht\@currbox
                   \global \advance \@textfloatsheight 2\intextsep
                   \@cons \@midlist \@currbox
                   \if@nobreak
                     \nobreak
                     \@nobreakfalse
                     \everypar{}%
                   \else
                     \addpenalty \interlinepenalty
                   \fi
                   \vskip \intextsep
                   \unvbox\@currbox %AO
                   \penalty\interlinepenalty
                   \vskip\intextsep
                   \ifnum\outputpenalty <-\@Mii \vskip -\parskip\fi
                   \outputpenalty \z@
                   \@inserttrue
                 \fi
               \fi
               \if@insert
               \else
                 \@addtotoporbot
               \fi
             \fi
           \fi
         \fi
       \fi
     \fi
   \fi
   \if@insert
   \else
     \@resethfps
     \@cons\@deferlist\@currbox
   \fi
}%
%    \end{macrocode}
% \end{macro}
%
% \begin{macro}{\if@twocolumn}
% The \cmd\newif\ switch \cmd\if@twocolumn\ is entirely unused. However its access words are invoked by
% \LaTeX's \cmd\document\ procedure, so we de-fang it.
%    \begin{macrocode}
\@twocolumnfalse
\let\@twocolumntrue\@twocolumnfalse
%    \end{macrocode}
% \end{macro}
%
% \begin{macro}{\@addmarginpar}
% The procedure \cmd\@addmarginpar\ used to access \cmd\if@twocolumn, but that switch is not reliable;
% the better way is to use \cmd\thepagegrid. We establish a convention for a page-grid-oriented
% procedure, e.g., \cmd\@addmarginpar@one, that emits a boolean, telling this procedure
% whether to set the marginpar on the left or right.
%    \begin{macrocode}
\def\@addmarginpar{%
 \@next\@marbox\@currlist{%
  \@cons\@freelist\@marbox\@cons\@freelist\@currbox
 }\@latexbug
 \setbox\@marbox\hb@xt@\columnwidth{%
  \csname @addmarginpar@\thepagegrid\endcsname{%
   \hskip-\marginparsep\hskip-\marginparwidth
   \box\@currbox
  }{%
   \hskip\columnwidth\hskip\marginparsep
   \box\@marbox
  }%
  \hss
 }%
 \setbox\z@\box\@currbox
    \@tempdima\@mparbottom
    \advance\@tempdima -\@pageht
    \advance\@tempdima\ht\@marbox
 \@ifdim{\@tempdima >\z@}{%
   \@latex@warning@no@line {Marginpar on page \thepage\space moved}%
 }{%
   \@tempdima\z@
 }%
    \global\@mparbottom\@pageht
    \global\advance\@mparbottom\@tempdima
    \global\advance\@mparbottom\dp\@marbox
    \global\advance\@mparbottom\marginparpush
    \advance\@tempdima -\ht\@marbox
    \global\setbox \@marbox
                   \vbox {\vskip \@tempdima
                          \box \@marbox}%
    \global \ht\@marbox \z@
    \global \dp\@marbox \z@
    \kern -\@pagedp
    \nointerlineskip
  \box\@marbox
    \nointerlineskip
    \hbox{\vrule \@height\z@ \@width\z@ \@depth\@pagedp}%
}%
%    \end{macrocode}
% \end{macro}
%
% \begin{environment}{turnpage}
% Any float (viz., \env{figure} or \env{table}) within the scope of this environment
% will be a turnpage float: It will be assumed to occupy an entire page (constitute a float page),
% the width will be \cmd\textheight, the height \cmd\textwidth, and the entire float will be presented
% rotated 90 degrees. 
%
% The implementation requires the services of the \cmd\rotatebox\ command, so we supply a dummy definition
% that explains things to the user.
%    \begin{macrocode}
\newenvironment{turnpage}{%
 \def\width@float{\textheight}%
 \def\widthd@float{\textheight}%
 \appdef\@endfloatbox{%
  \@ifxundefined\@currbox{%
   \ltxgrid@warn{Cannot rotate! Not a float}%
  }{%
   \setbox\@currbox\vbox to\textwidth{\vfil\unvbox\@currbox\vfil}%
   \global\setbox\@currbox\vbox{\rotatebox{90}{\box\@currbox}}%
  }%
 }%
}{%
}%
\def\rotatebox@dummy#1#2{%
 \ltxgrid@warn{You must load the graphics or graphicx package in order to use the turnpage environment}%
 #2%
}%
%    \end{macrocode}
% \changes{4.1b}{2008/08/04}{Use \cs{document@inithook} instead of \cs{AtBeginDocument}}
%    \begin{macrocode}
\appdef\document@inithook{%
 \@ifxundefined\rotatebox{\let\rotatebox\rotatebox@dummy}{}%
}%
%    \end{macrocode}
% \end{environment}
%
%
% \subsection{One-off output routines}
%
% These procedures are executed in lieu of \cmd\the\cmd\output\ when the output penalty has the associated flag value.
%
% \begin{macro}{output@-1073741824}
% The first one-off output routine handles the end of the job, wherein
% \LaTeX\ executes \cmd\@@end, and breaks to the output with a penalty of
% $"40000000 = 2^{32}/4 = 1073741824$. 
% We simply discard \cmd\box\cmd\@cclv\ and leave.
% This means that \LaTeX\ is obligated to do \cmd\clearpage\ as part of
% its \enve{document} processing, otherwise material will be lost.
%    \begin{macrocode}
\@namedef{output@-1073741824}{%
 \deadcycles\z@
%    \end{macrocode}
%    \begin{verbatim}
%\showbox\@cclv
%    \end{verbatim}
%    \begin{macrocode}
 \void@cclv
}%
%    \end{macrocode}
% \end{macro}
%
% \begin{macro}{\save@column@pen}
% The one-off output routine associated with \cmd\penalty\cmd\save@column@pen\
% will be called within a sequence of three such routines by \cmd\execute@message
% or its companion routine \cmd\execute@message@insert.
% This procedure must save away any the current page and preserve marks.
%    \begin{macrocode}
\mathchardef\save@column@pen=10016
\@namedef{output@-\the\save@column@pen}{\save@column}%
%    \end{macrocode}
% \end{macro}
%
% \begin{macro}{\@cclv@saved}
% We take over the \cmd\@holdpg\ box register. Hereafter,
% we no longer use the \cmd\@holdpg\ box register, so let the world know.
% This should decisively break packages that assume standard \LaTeX.
% Breaking decisively is preferred to quietly proceeding erroneously.
%    \begin{macrocode}
\let \@cclv@saved \@holdpg
\let \@holdpg \@undefined
%    \end{macrocode}
% \end{macro}
%
% \begin{macro}{\save@column}
% The procedure \cmd\save@column\ does the actual work of saving away the material
% on the page. It is invoked both by \cmd\save@column@pen\ and by \cmd\save@column@insert@pen.
% We save \cmd\box\cmd\@cclv\ and the primitive \cmd\@@topmark.
%    \begin{macrocode}
\def\save@column{%
 \@ifvoid\@cclv@saved{%
  \set@top@firstmark
  \global\@topmark@saved\expandafter{\@@topmark}%
 }{}%
 \global\setbox\@cclv@saved\vbox{%
  \@ifvoid\@cclv@saved{}{%
   \unvbox\@cclv@saved
   \marry@baselines
  }%
  \unvbox\@cclv
  \lose@breaks
  \remove@lastbox
 }%
}%
\newtoks\@topmark@saved
%    \end{macrocode}
% \end{macro}
%
% \begin{macro}{\prep@cclv}
% The procedure \cmd\prep@cclv\ is used by message handlers to set up their environment
% to ape that of the usual output routine, with the boxed-up page in \cmd\box\cmd\@cclv.
% Here, we retrieve the material from \cmd\@cclv@saved, where it was saved away by
% the one-off output routine associated with \cmd\save@column@pen.
%    \begin{macrocode}
\def\prep@cclv{%
 \void@cclv
 \setbox\@cclv\box\@cclv@saved
 \vbadness\@M
}%
%    \end{macrocode}
% \end{macro}
%
% \begin{macro}{\save@column@insert@pen}
% The one-off output routine associated with \cmd\penalty\cmd\save@column@insert@pen\
% is similar to that of \cmd\save@column@pen\ augmented with the processing of insertions.
% It is called by \cmd\execute@message@insert\  (i.e., at a grid change)
% and saves away the current page and preserves marks.
% In addition, it saves away any insertions that fall on the current page.
% As with the natural output routine, it executes in two phases,
% first with \cmd\holdinginserts\ set, then cleared. 
%    \begin{macrocode}
\mathchardef\save@column@insert@pen=10017
\@namedef{output@-\the\save@column@insert@pen}{\toggle@insert{\savecolumn@holding}{\savecolumn@moving}}%
%    \end{macrocode}
% The procedure \cmd\savecolumn@holding\ is the first phase of saving a column with its inserts.
% This phase must detect and remedy the one circumstance that will confound our efforts to propagate marks.
% It is similar to \cmd\output@holding, except that we have to deal with the protection box, which must
% remain, because the messaging mechanism is being used.
%
% If it appears that we have the pathological ``Big Bad Box'' case at hand, we use the \cmd\dead@cycle@repair@protected\
% procedure instead of \cmd\dead@cycle\ to do our dead cycle.
% \changes{4.1b}{2008/08/04}{Use \cs{trace@box} instead of \cs{showbox}}
%    \begin{macrocode}
\def\savecolumn@holding{%
 \@if@exceed@pagegoal{\unvcopy\@cclv\remove@lastbox}{%
  \setbox\z@\vbox{\unvcopy\@cclv\remove@lastbox}%
  \outputdebug@sw{\trace@box\z@}{}%
  \dimen@\ht\@cclv\advance\dimen@-\ht\z@
  \dead@cycle@repair@protected\dimen@
 }{%
  \dead@cycle
 }%
}%
%    \end{macrocode}
% \changes{4.1b}{2008/08/04}{New procedure for showing a box contents, \cs{trace@box}}
% \changes{4.1n}{2009/12/02}{More diagnostics of column balancing}
% The procedure \cmd\save@column@moving\ is the second phase of saving a column with its inserts. 
% Now that \cmd\holdinginserts\ is cleared, we can look in the various \cmd\insert\ registers for 
% our inserts (at present there is only one, \cmd\footins, along with \cmd\footins@saved). 
% if anything is there, we save it away and ask for another cycle (because it may have split). 
% 
% Note that the message that is about to be executed had better deal properly with the contents of 
% the \cmd\footins@saved\ box.
%
%    \begin{macrocode}
\def\savecolumn@moving{%
 \ltxgrid@info@sw{\class@info{\string\savecolumn@moving}}{}%
 \@cclv@nontrivial@sw{%
  \save@column
 }{%
  \void@cclv
 }%
 \@ifvoid\footins{}{%
  \ltxgrid@foot@info@sw{\class@info{\string\savecolumn@moving}\trace@scroll{\showbox\footins@saved\showbox\footins}}{}%
%    \end{macrocode}
% Save all away in \cmd\footins@saved. Note that if \cmd\footins\ is void, then \cmd\footins@saved\ remains untouched. 
%    \begin{macrocode}
  \@ifvoid\footins@saved{%
   \global\setbox\footins@saved\box\footins
  }{%
   \global\setbox\footins@saved\vbox\bgroup
    \unvbox\footins@saved
    \marry@baselines
    \unvbox\footins
   \egroup
  }%
  \ltxgrid@foot@info@sw{\trace@box\footins@saved}{}%
  \protect@penalty\save@column@insert@pen
 }%
}%
\newbox\footins@saved
\newbox\footins@recovered
\newbox\column@recovered
%    \end{macrocode}
% \end{macro}
%
% \begin{macro}{\save@message@pen}
% The one-off output routine associated with \cmd\penalty\cmd\save@message@pen\
% saves away the message that has been passed.
% This procedure is penultimate in a sequence of one-off output routine calls;
% earlier ones have saved away the MVL and preserved marks, the last executes the message.
%
% Note that we are passing tokens to \TeX's primitive \cmd\mark\ mechanism, so we must ensure
% that they are not inappropriately expanded. We use the same mechanism for all such cases,
% namely \cmd\let@mark.
%
% Note: we expect that \cmd\box\cmd\@cclv's contents are well known:
% \cmd\topskip, protection box, and a \cmd\mark, the latter containing the message.
% But if we came here via \cmd\penalty 10017, there might be an \cmd\insert\ node present as well, 
% because a footnote may have split. 
% Because this procedure simply voids out \cmd\box\cmd\@cclv, such material would be lost.
% Perhaps we can repair things by manipulating the \cmd\insert \ mechanism temporarily.
%    \begin{macrocode}
\mathchardef\save@message@pen=10018
\@namedef{output@-\the\save@message@pen}{\save@message}%
\def\save@message{%
 \void@cclv 
%    \end{macrocode}
%FIXME: what if \cmd\box\cmd\@cclv is not empty?
%    \begin{macrocode}
 \toks@\expandafter{\@@firstmark}%
 \expandafter\gdef\expandafter\@message@saved\expandafter{\the\toks@}%
 \expandafter\do@@mark\expandafter{\the\@topmark@saved}%
}%
\gdef\@message@saved{}%
%    \end{macrocode}
% \end{macro}
%
% \begin{macro}{\execute@message@pen}
% The one-off output routine associated with \cmd\execute@message@pen\ 
% simply executes the given message. It is last in a sequence of one-off output routine calls;
% earlier ones have saved all that require saving.
%    \begin{macrocode}
\mathchardef\execute@message@pen=10019
\@namedef{output@-\the\execute@message@pen}{\@message@saved}%
%    \end{macrocode}
% \end{macro}
%
%
% \subsection{Output messages}
%
% Message handlers are procedures that execute output messages, tokens that are passed to
% the output routine for execution in an environment appropriate to page makeup.
%
% How it works. We put down three large negative penalties, each of which will be handled by
% the output dispatcher (\emph{not} the natural output routine), each penalty being protected by
% a removable, non-discardable item (i.e., a box).
% Either three or four invocations of one-off output routines are involved per message.
%
% We make the last of the three protection boxes have a depth equal to the value of \cmd\prevdepth\
% that was current when the procedure is called. This effectively restores \cmd\prevdepth.
%
% In each case, the one-off output routine will remove the extraneous box we have inserted.
% And the second and third one-off routines will simply void \cmd\box\cmd\@cclv, because its contents
% are entirely artificial.
%
% FIXME: not so! If \cmd\holdinginserts\ is cleared, that box may have an insert node; it must be preserved, too.
%
% The first routine saves away the current column contents and remembers the \cmd\topmark\ for later
% use. There is a variant routine that first clears \cmd\holdinginserts, so that the message can
% handle any inserts present in the boxed-up page; this of course entails yet another
% visit to the output routine.
%
% The penultimate routine saves away the tokens transmitted in via the \cmd\@@mark: the argument of
% the macro. These tokens are of course the very thing we wish to execute within the safety
% of the output routine. It also puts down a mark containing the \cmd\topmark\ tokens saved
% by the first routine. By this means, the mark, which we have clobbered, is restored.
%
% The last routine simply executes the given tokens.
% In the course of doing this, it must take care of \cmd\box\cmd\@cclv, either by shipping it out,
% or by \cmd\unvbox ing it onto the MVL.
%
% \begin{macro}{\execute@message}
% The procedure \cmd\execute@message\ simply calls the utility procedure \cmd\@execute@message\
% with a penalty value for the standard treatment.
%    \begin{macrocode}
\def\execute@message{%
 \@execute@message\save@column@pen
%    \end{macrocode}
% Implicit second argument
%    \begin{macrocode}
}%
%    \end{macrocode}
% \end{macro}
%
% \begin{macro}{\execute@message@insert}
% The procedure \cmd\execute@message@insert\ is like \cmd\execute@message\ in all respects
% except that the penalty value is \cmd\save@column@insert@pen, which arranges for the 
% message handler involved to deal with the page's insertions.
% At the same time, we prepare the \cmd\footins\ box so that these insertions can be dealt with.
%
% Note:
% If more insertions are added to \LaTeX\ (presumably via \cmd\newinsert), then they
% must be dealt with in a way entirely analogous to \cmd\footins.
% \changes{4.1n}{2009/12/02}{More diagnostics of column balancing}
%    \begin{macrocode}
\def\execute@message@insert#1{%
 \@execute@message\save@column@insert@pen{%
  \setbox \footins \box \footins@saved
  \ltxgrid@foot@info@sw{\class@info{\string\execute@message@insert}\trace@box\footins}{}%
  #1%
 }%
}%
%    \end{macrocode}
% \end{macro}
%
% \begin{macro}{\@execute@message}
% The utility procedure \cmd\@execute@message\ is called by \cmd\execute@message\ and \cmd\execute@message@insert.
% We prepare by creating a \cmd\vbox\ containing all the needed nodes and proceed by simply
% \cmd\unvbox ing that box onto the MVL.
% We ensure that \cmd\box\cmd\@cclv\ is properly set up for the output message handler
% by always inserting \cmd\prep@cclv\ in advance of the argument.
%
% Note that each one-off output routine is invoked effectively the same as
% \cmd\protect@penalty, except that the second invocation involves an additional
% \cmd\mark\ node, and the third a specially prepared protection box.
%
% Note also that \TeX's primitive \cmd\mark\ is called here without any expansion protection.
% This is the only place where it is called that way, but it's OK because those tokens
% have have been pre-expanded by procedures that call \cmd\execute@message.
% FIXME: all procedures calling \cmd\execute@message\ must pre-expand their tokens!
%    \begin{macrocode}
\long\def\@execute@message#1#2{%
 \begingroup
  \dimen@\prevdepth\@ifdim{\dimen@<\z@}{\dimen@\z@}{}%
  \setbox\z@\vbox{%
   \protect@penalty#1%
   \protection@box
   \toks@{\prep@cclv#2}%
   \@@mark{\the\toks@}%
   \penalty-\save@message@pen
%    \end{macrocode}
%\begin{verbatim}
%  \hbox{\vrule\@height\z@\@width\z@\@depth\dimen@}%
%\end{verbatim}
%    \begin{macrocode}
   \setbox\z@\null\dp\z@\dimen@\ht\z@-\dimen@
   \nointerlineskip\box\z@
   \penalty-\execute@message@pen
  }\unvbox\z@
 \endgroup
}%
%    \end{macrocode}
% \end{macro}
%
% \begin{macro}{\do@output@cclv}
% The procedure \cmd\do@output@cclv\ provides access to message handlers at their simplest.
% The message will execute in the usual environment of the output routine, with
% the boxed-up page in \cmd\box\cmd\@cclv, and we assume that \cmd\holdinginserts\ remains set.
% This procedure must be invoked within main vertical mode;
% it is the obligation of the macro writer to ensure that this is the case.
%    \begin{macrocode}
\def\do@output@cclv{\execute@message}%
%    \end{macrocode}
% \end{macro}
%
% \begin{macro}{\do@output@MVL}
% The procedure \cmd\do@output@MVL, like \cmd\do@output@cclv, is an interface for messages,
% but provides two additional services: the command may also be invoked in horizontal mode, and
% the message handler will execute with the MVL unboxed. 
%    \begin{macrocode}
\def\do@output@MVL#1{%
 \@ifvmode{%
  \begingroup\execute@message{\unvbox\@cclv#1}\endgroup
 }{%
  \@ifhmode{%
   \vadjust{\execute@message{\unvbox\@cclv#1}}%
  }{%
   \@latexerr{\string\do@output@MVL\space cannot be executed in this mode!}\@eha
  }%
 }%
}%
%    \end{macrocode}
% \end{macro}
%
% \begin{macro}{\lose@breaks}
% The purpose of this procedure is to get rid of all the extraneous
% \cmd\penalty\cmd\@M\ nodes that tend to build up in the MVL.
%    \begin{macrocode}
\def\lose@breaks{%
 \loopwhile{%
  \count@\lastpenalty
  \@ifnum{\count@=\@M}{%
%    \end{macrocode}
% Note: 10000 is a TeX magic number!
%    \begin{macrocode}
   \unpenalty\true@sw
  }{%
   \false@sw
  }%
 }%
}%
%    \end{macrocode}
% \end{macro}
%
% \begin{macro}{\removestuff}
% \cmd\removestuff\ is a document-level command that removes the bottom skip glue item
% from the MVL.
%    \begin{macrocode}
\def\removestuff{\do@output@MVL{\unskip\unpenalty}}%
%    \end{macrocode}
% \end{macro}
%
% \begin{macro}{\removephantombox}
% The procedure \cmd\removephantombox\ is a special-purpose message handler exclusively for 
% preventing incorrect spacing above display math. It must be issued in
% horizontal mode within the phantom paragraph generated when display math starts up in
% vertical mode.
%    \begin{macrocode}
\def\removephantombox{%
 \vadjust{%
  \execute@message{%
   \unvbox\@cclv
   \remove@lastbox
   \unskip
   \unskip
   \unpenalty
   \penalty\predisplaypenalty
   \vskip\abovedisplayskip
  }%
 }%
}%
%    \end{macrocode}
% \end{macro}
%
% \begin{macro}{\addstuff}
% \cmd\addstuff\ is a document-level command that adds penalty, glue, or both to the 
% MVL. The penalty and glue items are rearranged so that all penalties nodes precede all the glue nodes,
% which is the canonical arrangement.
%    \begin{macrocode}
\def\addstuff#1#2{\edef\@tempa{\noexpand\do@output@MVL{\noexpand\@addstuff{#1}{#2}}}\@tempa}%
\def\@addstuff#1#2{%
 \skip@\lastskip\unskip
 \count@\lastpenalty\unpenalty
 \@if@empty{#1}{}{\penalty#1\relax}%
 \@ifnum{\count@=\z@}{}{\penalty\count@}%
 \vskip\skip@
 \@if@empty{#2}{}{\vskip#2\relax}%
}%
%    \end{macrocode}
% \end{macro}
%
% \begin{macro}{\replacestuff}
% \cmd\replacestuff\ is a document-level command similar to \cmd\addstuff;
% but it replaces penalty, glue, or both in the MVL.
% The penalty and glue items are rearranged so that all penalties nodes precede all the glue nodes,
% which is the canonical arrangement.
%    \begin{macrocode}
\def\replacestuff#1#2{\edef\@tempa{\noexpand\do@output@MVL{\noexpand\@replacestuff{#1}{#2}}}\@tempa}%
\def\@replacestuff#1#2{%
 \skip@\lastskip\unskip
 \count@\lastpenalty\unpenalty
 \@if@empty{#1}{}{%
 \@ifnum{\count@>\@M}{}{%
   \@ifnum{\count@=\z@}{\count@=#1\relax}{%
    \@ifnum{\count@<#1\relax}{}{%
     \count@=#1\relax
    }%
   }%
 }%
 }%
 \@ifnum{\count@=\z@}{}{\penalty\count@}%
 \@if@empty{#2}{}{%
  \@tempskipa#2\relax
  \@ifdim{\z@>\@tempskipa}{%
   \advance\skip@-\@tempskipa
  }{%
   \@ifdim{\skip@>\@tempskipa}{}{%
    \skip@\@tempskipa
   }%
  }%
 }%
 \vskip\skip@
}%
%    \end{macrocode}
% \end{macro}
%
% \begin{macro}{\move@insertions}
% \begin{macro}{\hold@insertions}
% In order to avoid bolluxing up \cmd\insert\ registers by our one-off 
% output routines, we set \cmd\holdinginserts\ to zero by default and only 
% clear it (briefly) while we handle cases where we want inserts to show up.
%    \begin{macrocode}
\def\move@insertions{\global\holdinginserts\z@}%
\def\hold@insertions{\global\holdinginserts\@ne}%
\hold@insertions
\def\toggle@insert#1#2{%
 \@ifnum{\holdinginserts>\z@}{\move@insertions#1}{\hold@insertions#2}%
}%
%    \end{macrocode}
% \end{macro}
% \end{macro}
%
%
% \subsection{Messages to alter the page grid}%
%
% Here is the implementation of the grid-switching procedures.
% We perform two checks when changing the page grid; first to ensure that 
% the target page grid is known (defensive programming), second to ensure that
% the switch is a non-trivial one. The latter check must be performed within
% the safety of the output routine, so requires using an output message.
% Thus, a grid change requires two messages, for a total of six visits to the output routine.
%
% \begin{macro}{\do@columngrid}
% Utility procedure \cmd\do@columngrid\ changes the page grid.
% Note that this command forces an end to the current paragraph. This is necessary, because
% a page grid change makes no sense unless we can alter the \cmd\hsize\ before commencing to
% typeset the following paragraph. So the command should never be executed in horizontal mode
% anyway.
%    \begin{macrocode}
\def\do@columngrid#1#2{%
 \par
 \expandafter\let\expandafter\@tempa\csname open@column@#1\endcsname
 \@ifx{\relax\@tempa}{%
  \ltxgrid@warn{Unknown page grid #1. No action taken}%
 }{%
  \do@output@MVL{\start@column{#1}{#2}}%
 }%
}%
%    \end{macrocode}
% \end{macro}
%
% \begin{macro}{\start@column}
% Procedure \cmd\start@column\ lays down the interrupts to switch the page grid.
% If the change to the page grid would have been trivial, it bails out.
% It seems a reasonable tradeoff of processing versus security: once we commit
% to changing the page grid, we clear \cmd\holdinginserts, so there is no turning back.
% 
% Note that the second argument to the macro allows us to pass an argument to the 
% page grid that is starting up. This can be handy, because a single procedure can 
% handle multiple page grids, differing only by the value of a parameter.
%
% FIXME: this means that you cannot switch between mlt page grids in a single step.
% But do we want to do this, at all, at all?
%    \begin{macrocode}
\def\start@column#1#2{%
 \def\@tempa{#1}\@ifx{\@tempa\thepagegrid}{%
  \ltxgrid@info{Already in page grid \thepagegrid. No action taken}%
 }{%
  \expandafter\execute@message@insert
  \expandafter{%
               \csname shut@column@\thepagegrid\expandafter\endcsname
               \csname open@column@#1\endcsname{#2}%
               \set@vsize
             }%
 }%
}%
%    \end{macrocode}
% \end{macro}
%
% \begin{macro}{\thepagegrid}
% The macro \cmd\thepagegrid\ tracks what kind of page grid we are in.
%
% Note: Access \cmd\thepagegrid\ only within the safety of the output routine.
%
% Warning: The page grid should be changed only within the safety of the output routine.
% People who write multicol page grid mechanisms appear not to understand the matter, so they
% should particularly heed this warning. Think about it: obviously Lamport did so, which is why
% his \cmd\twocolumn\ command forced a pagebreak, which is limiting, but safe.
%    \begin{macrocode}
\def\thepagegrid{one}%
%    \end{macrocode}
% \end{macro}
%
%
% \subsection{Application Note: implementing a page grid}
%
% If you want to create a new page grid for \LaTeX,
% you must define five procedures with specific names:
% \cmd\open@column@ name, \cmd\shut@column@ name, \cmd\end@column@ name,
% \cmd\output@column@ name, and \cmd\@addmarginpar@ name, where ``name'' is the 
% name of your page grid.
%
% The procedure \cmd\open@column@ name starts the new page grid. It should define \cmd\thepagegrid,
% deal with \cmd\box\cmd\pagesofar\ and \cmd\box\cmd\footsofar\ (perhaps by leaving them alone),
% and it should set the values of \LaTeX's page layout parameters for the column size and height.
%
% The procedure \cmd\shut@column@ name should expect to be called with \cmd\holdinginserts\
% cleared (it can assume that \cmd\holdinginserts\ will automatically be restored).
% It should properly deal with insertions (like footnotes); calling \cmd\@makecolumn\
% with an argument of \cmd\false@sw\ will do this.
% It should know that the page grid is being terminated
% in the middle of a page, so it should make arrangements to carry the footnotes down to the 
% bottom of the column or page, and it should possibly salt away the material for later
% incorporation into the page. The box registers \cmd\footsofar\ and \cmd\pagesofar\ are customarily used 
% for this purpose.
%
% The procedure \cmd\end@column@ name should kick out a possibly short page containing all the 
% floats committed to the page. It will be invoked during \cmd\clearpage\ processing.
% After that, it should \cmd\unvbox\cmd\@cclv.
%
% The procedure \cmd\output@column@ name should ship out or commit the current \cmd\@outputbox.
% In a one-column layout, you ship out; in a multicolumn layout, you commit the box as the contents
% of a particular column, and if that column is the last, you ship out.
%
% The procedure \cmd\@addmarginpar@ name should return a boolean (either \cmd\true@sw\ or \cmd\false@sw\ or an equivalent)
% to tell the marginpar mechanism to place the marginal material to the right or left, respectively.
%
% You can use the existing page grids ``one'' and ``mlt'' as a point of departure for creating others.
% The former can be the basis for, say, a single-column page grid with a side column.
%
% \begin{macro}{\pagesofar}
% \begin{macro}{\footsofar}
% The box register \cmd\pagesofar\ holds the portion of the (full-width) page that is already composed into columns. 
% This, plus the finished columns, each with its floats, plus \cmd\box255 constitute the full galley.
%
% The box register \cmd\footsofar\ holds all of the footnotes associated with \cmd\pagesofar.
%    \begin{macrocode}
\newbox\pagesofar
\newbox\footsofar
%    \end{macrocode}
% \end{macro}
% \end{macro}
%
% \begin{macro}{\combine@foot@inserts}
% \changes{4.1n}{2010/01/02}{(AO, 571) calling sequence of \cs{combine@foot@inserts} and \cs{grid@column} to expose box registers; that of \cs{append@column@} its column counters}%
% \changes{4.1n}{2010/01/02}{(AO, 571) coding convention: use \cs{bgroup}, \cs{egroup} (instead of braces) when a box is being built}%
% The procedure \cmd\combine@foot@inserts\ is for the purpose of merging 
% the recently contributed footnotes (usually \cmd\box\cmd\footins)
% with those saved from earlier on the page (usually \cmd\box\cmd\footsofar). 
%
% It is employed in a number of circumstances. 
%
% \cmd\@makecolumn (when its argument is \cmd\false@sw): 
% we are not shipping out, so we need to salt away any footnotes there may be. 
% 
% \cmd\shut@column@one: we are leaving the one-column page grid, so recover the footnotes from that material
% and combine them with those of \cmd\pagesofar. 
%
% \expandafter\cmd\csname balance@2\endcsname: two columns of type have been balanced, so now balance the footnotes. 
% The \cmd\combine@foot@inserts\ procedure is first used to gather footnotes from the columns balanced with those of \cmd\pagesofar. 
%
% Bug 571 note: if balancing a two-column page grid, and there had been footnotes in the \cmd\pagesofar, 
% those footnotes will have been balanced into a page-width box, \cmd\box\cmd\footsofar. 
% We need to now re-cast them into a single, column-width galley, and only then 
% combine them with those in \cmd\box\cmd\footins. 
% 
%    \begin{macrocode}
\def\combine@foot@inserts#1#2{%
  \ltxgrid@info@sw{\class@info{\string\combine@foot@inserts\string#1\string#2}}{}%
  \@ifvoid#1{%
    \ltxgrid@foot@info@sw{\trace@box#2}{}\global\setbox#1\box#2%
  }{%
   \global\setbox#1\vbox\bgroup
    \ltxgrid@foot@info@sw{\trace@box#1}{}\unvbox#1%
    \@ifvoid#2{}{%
     \marry@baselines
     \ltxgrid@foot@info@sw{\trace@box#2}{}\unvbox#2%
    }%
   \egroup
  }%
  \ltxgrid@foot@info@sw{\trace@scroll{\showbox#1\showbox#2}}{}%
}%
%    \end{macrocode}
% \end{macro}
%
%
% \subsubsection{One-column page grid}
%
% \begin{macro}{\onecolumngrid}
% \begin{macro}{\open@column@one}
% \begin{macro}{\shut@column@one}
% \begin{macro}{\float@column@one}
% \begin{macro}{\end@column@one}
% \begin{macro}{\output@column@one}
% \begin{macro}{\@addmarginpar@one}
% Here are all the procedures necessary for the standard page grid named ``one'':
% a single column layout. It is, of course, \LaTeX's familiar \cmd\onecolumn\ layout.
% We begin with the procedure exposed to the style writer.
% This is, however, not a \LaTeX\ command; users should not change the page grid.
%    \begin{macrocode}
\newcommand\onecolumngrid{\do@columngrid{one}{\@ne}}%
%    \end{macrocode}
%
% Note that a document class that issues the command \cmd\onecolumn\ will break. This includes
% \LaTeX's standard classes.dtx-based classes: if your class descends from one of these, you
% must expunge it of all such commands.
%    \begin{macrocode}
\let\onecolumn\@undefined
%    \end{macrocode}
%
% The procedure \cmd\open@column@one\ takes advantage of the special nature of the one-column
% page grid to deal with \cmd\box\cmd\pagesofar, therefore it must also reset \cmd\@colroom.
% \changes{4.1b}{2008/08/04}{Change \cs{set@colroom} to \cs{set@colht}}
% \changes{4.1n}{2009/12/02}{More diagnostics of column balancing}
%    \begin{macrocode}
\def\open@column@one#1{%
 \ltxgrid@info@sw{\class@info{\string\open@column@one\string#1}}{}%
%    \end{macrocode}
% Throw the \cmd\pagesofar\ back onto the Main Vertical List. 
% At this point, we must also \cmd\insert\ the footnotes back into the MVL. 
% \changes{4.1n}{2010/01/02}{(AO, 571) coding convention: use \cs{bgroup}, \cs{egroup} (instead of braces) when a box is being built}%
%    \begin{macrocode}
 \unvbox\pagesofar
 \@ifvoid{\footsofar}{}{%
  \insert\footins\bgroup\unvbox\footsofar\egroup
  \penalty\z@
 }%
%    \end{macrocode}
% Record which page grid we are using. Then calculate the set width (\cmd\hsize) and the goal height (\cmd\vsize).
%
% Klootch: we set the \cmd\count\cmd\footins\ to a magic number. This is only correct in the case of a two-column document.
%    \begin{macrocode}
 \gdef\thepagegrid{one}%
 \global\pagegrid@col#1%
 \global\pagegrid@cur\@ne
 \global\count\footins\@m
 \global\divide\count\footins\tw@
 \set@column@hsize\pagegrid@col
 \set@colht
}%
%    \end{macrocode}
%
% The procedure \cmd\shut@column@one\ saves away the one-column material into the box register \cmd\pagesofar.
% Because it is called from a message handler, we are assured that marks are properly taken care of.
%
% This instance of \cmd\@makecolumn\ is building a column for saving into \cmd\pagesofar.
% \changes{4.1b}{2008/08/04}{Change \cs{@makecol} to \cs{@makecolumn} with argument}
% \changes{4.1b}{2008/08/04}{Change \cs{set@colroom} to \cs{set@colht}}
% \changes{4.1f}{2009/07/15}{(AO, 519) Preserve footnotes that are in \cs{footsofar} across a page grid change}
% \changes{4.1n}{2009/12/02}{More diagnostics of column balancing}
% \changes{4.1n}{2010/01/02}{(AO, 571) calling sequence of \cs{combine@foot@inserts} and \cs{grid@column} to expose box registers; that of \cs{append@column@} its column counters}%
% We recover the footnotes into \cmd\footsofar\ (globally) and the column into \cmd\pagesofar\ (also globally), voiding \cmd\@outputbox\ by side effect. 
%    \begin{macrocode}
\def\shut@column@one{%
 \ltxgrid@info@sw{\class@info{\string\shut@column@one}}{}%
 \@makecolumn\false@sw
%    \end{macrocode}
% Split text portion of \cmd\@outputbox\ into \cmd\pagesofar,
% and add its footnote portion to \cmd\footsofar.
% Then void out \cmd\@outputbox.
%    \begin{macrocode}
 \global\setbox\pagesofar\vbox\bgroup
  \recover@column\@outputbox\footsofar\column@recovered\footins@recovered
 \egroup
 \begingroup\setbox\z@\box\@outputbox\endgroup
%    \end{macrocode}
% FIXME: is \cmd\combine@foot@inserts\ needed?
% Also: if this procedure is immediately followed by \cmd\open@column@ \texttt{grid}, then \cmd\set@colht\ will be unneeded.
%    \begin{macrocode}
 \combine@foot@inserts\footsofar\footins
 \set@colht
}%
%    \end{macrocode}
%
%FIXME: the first line of a footnote should have an up-strut, and the last line a down-strut, so that they can marry baselines.
% The latter is the case; how about the former?
%
% The procedure \cmd\float@column@one\ takes care of a float column that has been built by 
% \cmd\@tryfcolumn, in the single-column page grid.
%
% This instance of \cmd\@makecolumn\ is followed by \cmd\@outputpage: it is building 
% a column for \cmd\shipout, rather than for saving into \cmd\pagesofar.
% \changes{4.1b}{2008/08/04}{Change \cs{@makecol} to \cs{@makecolumn} with argument}
%    \begin{macrocode}
\def\float@column@one{%
 \@makecolumn\true@sw
 \@outputpage
}%
%    \end{macrocode}
%
% The procedure \cmd\end@column@one\ is executed at the end of \cmd\clearpage\ processing,
% if we were in a one-column page grid, once all permissive float pages have been shipped out.
% At this point, one could perhaps
% assume that nothing more need be done, but let us anyway test for committed floats and force a shipout.
%
% FIXME: this procedure does the same as \cmd\end@column@mlt 
% (except for the test of \cmd\@ifx@empty\cmd\@dbltoplist):
% the two could almost be the same procedure.
%
% I have changed this procedure to avoid the testing it once did: it simply puts down interrupts,
% upon which it relies to correctly do what \cmd\clearpage\ requires.
%    \begin{macrocode}
\def\end@column@one{%
 \unvbox\@cclv\remove@lastbox
 \protect@penalty\do@newpage@pen
}%
%    \end{macrocode}
%
% The procedure \cmd\output@column@one\ is dispatched from the output routine when
% we have completed a page (that is, a column in a one-column page grid);
% it ships out the page using the \cmd\@outputpage.
% It will be followed up with an output routine message to prepare a new column.
%
% Query: by what mechanism do the footnotes get placed onto such a page?
%    \begin{macrocode}
\def\output@column@one{%
 \@outputpage
}%
%    \end{macrocode}
%
% The following procedure determines which side of the page a marginpar will appear.
% It reproduces the behavior of standard \LaTeX.
%    \begin{macrocode}
\def\@addmarginpar@one{%
 \@if@sw\if@mparswitch\fi{%
  \@ifodd\c@page{\false@sw}{\true@sw}%
 }{\false@sw}{%
  \@if@sw\if@reversemargin\fi{\false@sw}{\true@sw}%
 }{%
  \@if@sw\if@reversemargin\fi{\true@sw}{\false@sw}%
 }%
}%
%    \end{macrocode}
%
% The following procedure yields a Boolean value; it determines whether a float in the deferred queue
% is appropriate for placing. In the one-column grid, all floats are so.
%    \begin{macrocode}
\def\@floatselect@sw@one#1{\true@sw}%
%    \end{macrocode}
%
%    \begin{macrocode}
\def\onecolumngrid@push{%
 \do@output@MVL{%
  \@ifnum{\pagegrid@col=\@ne}{%
   \global\let\restorecolumngrid\@empty
  }{%
   \xdef\restorecolumngrid{%
    \noexpand\start@column{\thepagegrid}{\the\pagegrid@col}%
   }%
   \start@column{one}{\@ne}%
  }%
 }%
}%
\def\onecolumngrid@pop{%
 \do@output@MVL{\restorecolumngrid}%
}%
%    \end{macrocode}
% \end{macro}
% \end{macro}
% \end{macro}
% \end{macro}
% \end{macro}
% \end{macro}
% \end{macro}
%
%
% \subsubsection{Two-column page grid}
%
% \begin{macro}{\twocolumngrid}
% \begin{macro}{\open@column@mlt}
% \begin{macro}{\shut@column@mlt}
% \begin{macro}{\end@column@mlt}
% \begin{macro}{\output@column@mlt}
% \begin{macro}{\@addmarginpar@mlt}
% \begin{macro}{\set@footnotewidth@mlt}
% \begin{macro}{\set@footnotewidth@two}
% \begin{macro}{\compose@footnotes@two}
% Here are all the procedures necessary for the standard page grid named ``mlt'':
% the multi-column page grid. With an argument of "2", it is,
% of course, \LaTeX's familiar \cmd\twocolumn\ layout.
%
% We start with the procedure to switch to the two-column page grid.
%    \begin{macrocode}
\newcommand\twocolumngrid{\do@columngrid{mlt}{\tw@}}%
%    \end{macrocode}
%
% The corresponding command of \LaTeX\ is obsolete.
%    \begin{macrocode}
\let\twocolumn\@undefined
%    \end{macrocode}
%
% Of course, \cmd\@topnewpage\ is also obsolete. Just do 
%\begin{quote}
% \cmd\clearpage\cmd\onecolumngrid<vertical mode material>\cmd\twocolumngrid.
%\end{quote}
%    \begin{macrocode}
\let\@topnewpage\@undefined
%    \end{macrocode}
%
% If your document class descends from one of \LaTeX's standard classes.dtx-derived
% classes, it will break. You must expunge from it all such commands.
% 
% \changes{4.1b}{2008/08/04}{Change \cs{set@colroom} to \cs{set@colht}}
%
% Bug 571 note: it is not enough to have the \cmd\pagesofar, we must also deal with the \cmd\footsofar.
% At this juncture, we should treat the case where the document has an essentially two-column page grid, 
% with occasional excursions into the one-column grid. 
% If a footnote is set within the latter grid, its set width should be that of the two-column grid. 
%
% When a page is shipped out, if we are currently in a one-column grid, we will compose the footnotes onto the page
% in the form of balanced columns. This is only one way to handle footnotes: \classname{multicol}
% appears to set footnotes on the full text width.
%    \begin{macrocode}
\def\open@column@mlt#1{%
 \ltxgrid@info@sw{\class@info{\string\open@column@mlt\string#1}}{}%
%    \end{macrocode}
% At this point, we must \cmd\insert\ the footnotes back into the Main Vertical List.
% \changes{4.1n}{2010/01/02}{(AO, 571) coding convention: use \cs{bgroup}, \cs{egroup} (instead of braces) when a box is being built}%
%    \begin{macrocode}
 \@ifvoid{\footsofar}{}{%
  \insert\footins\bgroup\unvbox\footsofar\egroup
 }%
%    \end{macrocode}
% Record which page grid we are using. Then calculate the set width (\cmd\hsize) and the goal height (\cmd\vsize).
%
% Klootch: we set the \cmd\count\cmd\footins\ to a magic number. This value is valid whether footnotes are being
% set on the column width or the full text width. 
%    \begin{macrocode}
 \gdef\thepagegrid{mlt}%
 \global\pagegrid@col#1%
 \global\pagegrid@cur\@ne
 \global\count\footins\@m
 \set@column@hsize\pagegrid@col
 \set@colht
}%
%    \end{macrocode}
%
% The procedure \cmd\shut@column@mlt\ ends the current column, balances the columns, and
% salts away all in \cmd\pagesofar. Because it is called in a message handler,
% we are assured that marks are handled properly.
% Attention: because this procedure balances columns, all footnotes are
% held aside in \cmd\footsofar\ for placement at the bottom of the page.
%
% Bug note: the last macro executed by this procedure is \cmd\set@colht, 
% but had been erroneously \cmd\set@colroom.
% I now believe that the latter should be changed pretty much everywhere to the former.
%
% This instance of \cmd\@makecolumn\ is building material for \cmd\pagesofar, rather than 
% for \cmd\shipout.
% \changes{4.1b}{2008/08/04}{Change \cs{@makecol} to \cs{@makecolumn} with argument}
% \changes{4.1b}{2008/08/04}{Change \cs{@combinepage} to \cs{@combinepage} with argument}
% \changes{4.1b}{2008/08/04}{Change \cs{set@colroom} to \cs{set@colht}}
% \changes{4.1n}{2009/12/02}{More diagnostics of column balancing}
% \changes{4.1n}{2010/01/02}{(AO, 571) calling sequence of \cs{combine@foot@inserts} and \cs{grid@column} to expose box registers; that of \cs{append@column@} its column counters}%
%    \begin{macrocode}
\def\shut@column@mlt{%
 \ltxgrid@info@sw{\class@info{\string\shut@column@mlt}}{}%
 \@cclv@nontrivial@sw{%
  \@makecolumn\false@sw
  \@ifnum{\pagegrid@cur<\pagegrid@col}{%
   \expandafter\global\expandafter\setbox\csname col@\the\pagegrid@cur\endcsname\box\@outputbox
   \global\advance\pagegrid@cur\@ne
  }{}%
 }{%
  \void@cclv
 }%
 \@ifnum{\pagegrid@cur>\@ne}{%
  \csname balance@\the\pagegrid@col\endcsname
  \grid@column\@outputbox{}%
  \@combinepage\false@sw
  \@combinedblfloats
  \global\setbox\pagesofar\box\@outputbox
  \show@pagesofar@size
 }{}%
 \set@colht
}%
%    \end{macrocode}
%
% The procedure \cmd\float@column@mlt\ takes care of a float page that has been built by \cmd\@tryfcolumn,
% in the multi-column page grid. It is coincidentally identical to what happens in 
% \cmd\do@startpage\ when a page needs to be shipped out.
% \changes{4.1b}{2008/08/04}{New procedure \cs{@output@combined@page}}
%    \begin{macrocode}
\def\float@column@mlt{%
  \@output@combined@page
}%
%    \end{macrocode}
%
% The procedure \cmd\end@column@mlt\ is executed at the end of \cmd\clearpage\ processing,
% if we were in a multi-column page grid, once all permissive float pages have been shipped out.
% If no floats are committed and if no columns are yet filled, we have nothing to do.
% Otherwise, we kick out a column and try again.
%
% Note that in our code to kick out a column, we must deal properly with the case where the column
% is trivial: it will have nothing but \cmd\topskip\ glue plus a protection box. We substitute an ordinary
% \cmd\null\ for the protection box.
% 
%    \begin{macrocode}
\def\end@column@mlt{%
 \@ifx@empty\@toplist{%
  \@ifx@empty\@botlist{%
   \@ifx@empty\@dbltoplist{%
    \@ifx@empty\@deferlist{%
     \@ifnum{\pagegrid@cur=\@ne}{%
      \false@sw
     }{%
      \true@sw
     }%
    }{%
     \true@sw
    }%
   }{%
    \true@sw
   }%
  }{%
   \true@sw
  }%
 }{%
  \true@sw
 }%
 % true = kick out a column and try again
 {%
  \@cclv@nontrivial@sw{%
   \unvbox\@cclv\remove@lastbox
  }{%
   \unvbox\@cclv\remove@lastbox\unskip\null
  }%
  \protect@penalty\do@newpage@pen
  \protect@penalty\do@endpage@pen
 }{%
  \unvbox\@cclv\remove@lastbox
 }%
}%
%    \end{macrocode}
%
% The procedure \cmd\output@column@mlt (cf. \cmd\output@column@one)
% is dispatched from the output routine when
% we have completed a column in a multi-column page grid).
% (It replaces the \cmd\@outputdblcol\ of standard \LaTeX.)
% If a complete set of columns is at hand, it ships out the page and 
% lays down an interrupt for \cmd\do@startpage@pen, which will commit the 
% full-page-width floats to the next page.
% Like \cmd\output@column@mlt, this is followed by
% an output routine message to prepare a new column.
%
% If a page needs to be shipped out, it uses the same mechanism as \cmd\do@startpage.
% \changes{4.1b}{2008/08/04}{New procedure \cs{@output@combined@page}}
% \changes{4.1n}{2010/01/02}{(AO, 571) calling sequence of \cs{combine@foot@inserts} and \cs{grid@column} to expose box registers; that of \cs{append@column@} its column counters}%
%    \begin{macrocode}
\def\output@column@mlt{%
 \@ifnum{\pagegrid@cur<\pagegrid@col}{%
  \expandafter\global\expandafter\setbox\csname col@\the\pagegrid@cur\endcsname\box\@outputbox
  \global\advance\pagegrid@cur\@ne
 }{%
  \set@adj@colht\dimen@
  \grid@column\@outputbox{}%
  \@output@combined@page
 }%
}%
%    \end{macrocode}
% The procedure \cmd\output@column@mlt\ obsoletes \LaTeX's \cmd\@outputdblcol
%    \begin{macrocode}
\let\@outputdblcol\@undefined
%    \end{macrocode}
%
% The following procedure yields a Boolean value; it determines whether a float in the deferred queue
% is appropriate for placement in the column. In the multi-column grid, only those narrower than \cmd\textwidth\ are so.
%    \begin{macrocode}
\def\@floatselect@sw@mlt#1{\@if@notdblfloat{#1}}%
%    \end{macrocode}
%
% The following procedure determines which side of the page a marginpar will appear.
% It reproduces the behavior of standard \LaTeX.
%    \begin{macrocode}
\def\@addmarginpar@mlt{% emits a boolean
 \@ifnum{\pagegrid@cur=\@ne}%
}%
%    \end{macrocode}
%
% \cmd\set@footnotewidth@one\ sets the width of type within footnotes to span the full text width; 
% \cmd\set@footnotewidth@two\ to span a single column of the two-column grid, and 
% more generally \cmd\set@footnotewidth@mlt\ for a multi-column page grid. 
%    \begin{macrocode}
\def\set@footnotewidth@one{%
 \hsize\columnwidth
 \linewidth\hsize
}%
\def\set@footnotewidth@two{\set@footnotewidth@mlt\tw@}%
\def\set@footnotewidth@mlt#1{%
 \hsize\textwidth
 \advance\hsize\columnsep
 \divide\hsize#1%
 \advance\hsize-\columnsep
 \linewidth\hsize
}%
%    \end{macrocode}
%
% \cmd\compose@footnotes\ is the procedure for arranging the footnotes for
% placement at the bottom of the page or column. 
% In the former case, the material will be shipped out; in the latter, we must allow
% the column to possibly be balanced later on. 
%
% \cmd\compose@footnotes@one\ is a no-op, because the footnotes require no rearrangement. 
% In a scheme where footnotes are set on the full text width, this would be the procedure called. 
%
% \cmd\compose@footnotes@two\ implements the case where a two-column document has been interrupted
% with full-page-width text (e.g., the widetext environment or the end of the document), 
% and a natural page break appears. 
%
% In either case, we assume that argument \verb+#1+ is an \cmd\insert\  register and must be assigned globally,
% so that when it is accessed with \cmd\box\ or \cmd\unvbox, it will be voided globally as well. 
% 
% To extend this scheme to a three-column page grid \cmd\compose@footnotes@thr@@\ would be created:
% it would balance the saved up footnotes into three columns. 
%
%    \begin{macrocode}
\def\compose@footnotes@one#1{%
 \ltxgrid@foot@info@sw{\class@info{\string\compose@footnotes@one\string#1}\trace@box#1}{}%
}%
\let\compose@footnotes\compose@footnotes@one
\def\compose@footnotes@two#1{%
 \ltxgrid@foot@info@sw{\class@info{\string\compose@footnotes@two\string#1}\trace@box#1}{}%
 \setbox\z@\box\@tempboxa
 \let\recover@column\recover@column@null
 \let\marry@baselines\@empty
 \balance@two#1\@tempboxa
 \global\setbox#1\hbox to\textwidth{\box#1\hfil\box\@tempboxa}%
 \ltxgrid@foot@info@sw{\trace@box#1}{}%
}%
%    \end{macrocode}
% \end{macro}
% \end{macro}
% \end{macro}
% \end{macro}
% \end{macro}
% \end{macro}
% \end{macro}
% \end{macro}
%
%
% \subsubsection{Page grid utility procedures}
%
% \begin{macro}{\pagegrid@cur}
% \begin{macro}{\pagegrid@col}
% \begin{macro}{\pagegrid@init}
% We take over \LaTeX's \cmd\col@number, and \cmd\@leftcolumn, which are obsolete (\cmd\@holdpg\ could also be taken over). 
% We create two counters to hold the columns in the page grid and the current column within.
% We also create the first of a set of box registers to hold the committted columns.
%    \begin{macrocode}
\let\pagegrid@cur\col@number
\let\col@number\@undefined
\newcount\pagegrid@col
\pagegrid@cur\@ne
\expandafter\let\csname col@\the\pagegrid@cur\endcsname\@leftcolumn
\let\@leftcolumn\@undefined
%    \end{macrocode}
%
% The default is for maximum two columns.
% If your class will require more columns, assign that number to \cmd\pagegrid@col\
% before \envb{document} time.
%    \begin{macrocode}
\pagegrid@col\tw@
%    \end{macrocode}
%
% The procedure \cmd\pagegrid@init\ is a loop, exercising \cmd\newbox\ sufficiently to create the 
% boxes for holding the columns in the page grid; these have names like \expandafter\cmd\csname col@1\endcsname, etc. 
%    \begin{macrocode}
\def\pagegrid@init{%
 \advance\pagegrid@cur\@ne
 \@ifnum{\pagegrid@cur<\pagegrid@col}{%
  \csname newbox\expandafter\endcsname\csname  col@\the\pagegrid@cur\endcsname
  \pagegrid@init
 }{%
 }%
}%
\appdef\class@documenthook{%
 \pagegrid@init
}%
%    \end{macrocode}
% \end{macro}
% \end{macro}
% \end{macro}
%
% \begin{macro}{\grid@column}
% \changes{4.1n}{2010/01/02}{(AO, 571) calling sequence of \cs{combine@foot@inserts} and \cs{grid@column} to expose box registers; that of \cs{append@column@} its column counters}%
% The procedure \cmd\grid@column\ knows how to lay up the columns in a multi-column page grid.
% It uses utility procedures \cmd\append@column@\ and \cmd\box@column.
%
% The first argument is the box register to create, usually \cmd\@outputbox, and provides both input and output. 
% The second argument a dimension, allowing us to strut down the depth of the box we create. 
%
%    \begin{macrocode}
\def\grid@column#1#2{%
 \ltxgrid@info@sw{\class@info{\string\grid@column\string#1}}{}%
 \global\setbox#1\vbox\bgroup
  \hb@xt@\textwidth\bgroup
   \vrule\@height\z@\@width\z@\@if@empty{#2}{}{\@depth#2}%
   \pagegrid@cur\@ne
   \@ifnum{\pagegrid@cur<\pagegrid@col}{\loopwhile{\append@column@\pagegrid@cur\pagegrid@col}}{}%
   \box@column#1%
  \egroup
%    \end{macrocode}
% FIXME: page depth!
%    \begin{macrocode}
  \vskip\z@skip
 \egroup
}%
%    \end{macrocode}
% \end{macro}
%
% \begin{macro}{\append@column@}
% \begin{macro}{\box@column}
% \begin{macro}{\marry@baselines}
% The procedure \cmd\append@column@\ appends columns for \cmd\grid@column,
% \cmd\box@column\ builds the columns for \cmd\append@column@,
% and \cmd\marry@baselines\ pastes vertical things back together.
% \changes{4.0a}{2001/06/18}{Introduce \cs{marry@height} }
%
% Note that \cmd\box@column\ makes an attempt to prevent excessive \cmd\topskip\
% or \cmd\baselineskip\ glue
% from being applied by \TeX\ when \cmd\@outputbox\ is contributed to the MVL.
% If this is not done, it is possible to get into an infinite loop in the corner case,
% wherein the page grid is changed to one column and the balanced-up columns are
% already sufficient to fill the page.
%
% Note (AO 0920):  I have changed the dimension involved with \cmd\box@column\ from
% \cmd\vsize\ to \cmd\textheight, because the former is certainly not the correct value
% to use: it will change if floats have been placed in the last column of the page.
% I believe \cmd\textheight\ is the correct parameter to use here.
%
% A REVTeX4 user, Sergey Strelkov (strelkov@maik.rssi.ru), wants the option
% of ragged-bottom columns. Implementing this feature properly means reboxing the 
% columns to their natural height only if \cmd\raggedcolumn@sw\ is true.
% Otherwise, they get reboxed to their common height (\cmd\@colht?).
%
% Note that the default has hereby changed from ragged to flush. 
% It's not clear that anyone but Sergey will notice.
%
% The macro \cmd\marry@skip\  addresses (in a limited way)
% the fact that neither the value of \cmd\baselineskip\ nor that of \cmd\topskip\
% can be relied upon for the purpose of marrying the baselines of two split columns.
% (Because there might have been a local change to their values at the point where
% the output routine got triggered.)
%
% For best results, your document class should call for grid changes only when in basal text
% settings. The \cmd\marry@baselines\ procedure will use the values appropriate to that point
% when attempting to put the columns back together.
%
% In any case, we are not attempting to solve the more general problem of how to marry baselines
% where the leading can change arbitrarily within the galley or where glue could have been trimmed
% at a page top.
%
% \changes{4.1n}{2010/01/02}{(AO, 571) calling sequence of \cs{combine@foot@inserts} and \cs{grid@column} to expose box registers; that of \cs{append@column@} its column counters}%
% Procedure \cmd\append@column@\ composes a column onto the horizontal list along with its \cmd\columnseprule. 
% Its arguments are: \verb+#1+---\cmd\pagegrid@cur, and \verb+#2+---\cmd\pagegrid@col
%    \begin{macrocode}
\def\append@column@#1#2{%
 \expandafter\box@column\csname col@\the#1\endcsname
 \hfil\vrule\@width\columnseprule\hfil
 \advance#1\@ne
%    \end{macrocode}
% This procedure is the argument of \cmd\loopwhile, so it must leave a Boolean (e.g., \cmd\true@sw) in \TeX's scanner. 
%    \begin{macrocode}
 \@ifnum{#1<#2}%
}%
%    \end{macrocode}
% \changes{4.1n}{2010/01/02}{(AO, 571) calling sequence of \cs{combine@foot@inserts} and \cs{grid@column} to expose box registers; that of \cs{append@column@} its column counters}%
% \changes{4.1n}{2009/12/02}{More diagnostics of column balancing}
% \changes{4.1n}{2010/01/02}{(AO, 571) coding convention: use \cs{bgroup}, \cs{egroup} (instead of braces) when a box is being built}%
% Procdure \cmd\box@column, used by \cmd\append@column@, puts down a box containing the specified column.
% Its height is adjusted down to \cmd\@colht, if needed; likewise, the width is set to \cmd\columnwidth. 
% The rag at the bottom is controlled by \cmd\raggedcolumn@skip. 
%    \begin{macrocode}
\def\box@column#1{%
 \ltxgrid@info@sw{\class@info{\string\box@column\string#1}}{}%
 \raise\topskip
 \hb@xt@\columnwidth\bgroup
  \dimen@\ht#1\@ifdim{\dimen@>\@colht}{\dimen@\@colht}{}%
  \count@\vbadness\vbadness\@M
  \dimen@ii\vfuzz\vfuzz\maxdimen
  \ltxgrid@info@sw{\saythe\@colht\saythe\dimen@}{}%
  \vtop to\dimen@\bgroup
   \hrule\@height\z@
   \unvbox#1%
   \raggedcolumn@skip
  \egroup
  \vfuzz\dimen@ii
  \vbadness\count@
  \hss
 \egroup
}%
%    \end{macrocode}
% The purpose of procedure \cmd\marry@baselines\ is to ensure that the baseline spacing is correct; 
% it does this by making adjustments to the previous line, compensating for its depth, and by adding in 
% skip glue in an amount that assumes the added material has \cmd\topskip\ glue above. 
%    \begin{macrocode}
\def\marry@baselines{%
 \begingroup
  \setbox\z@\lastbox
  \@ifvoid{\z@}{%
   \endgroup
  }{%
   \aftergroup\kern
   \aftergroup-%
   \expandafter\box\expandafter\z@\expandafter\endgroup\the\dp\z@\relax
  }%
 \vskip\marry@skip\relax
}%
\gdef\marry@skip{\z@skip}%
\def\set@marry@skip{%
 \begingroup
  \skip@\baselineskip\advance\skip@-\topskip
  \@ifdim{\skip@>\z@}{%
   \xdef\marry@skip{\the\skip@}%
  }{}%
 \endgroup
}%
%    \end{macrocode}
% \changes{4.1b}{2008/08/04}{Use \cs{document@inithook} instead of \cs{AtBeginDocument}}
%    \begin{macrocode}
\appdef\document@inithook{%
 \@ifxundefined\raggedcolumn@sw{\@booleanfalse\raggedcolumn@sw}{}%
}%
\def\raggedcolumn@skip{%
 \vskip\z@\raggedcolumn@sw{\@plus.0001fil\@minus.0001fil}{}\relax
}%
%    \end{macrocode}
% \end{macro}
% \end{macro}
% \end{macro}
%
% \begin{macro}{\@combinepage}
% The procedure \cmd\@combinepage\ prepends the stored page (\cmd\pagesofar) to \cmd\@outputbox\ and
% employs \cmd\@combineinserts\ to lay down the footnotes. 
% The next event will usually be shipping out the made-up page, but not always. 
% Therefore the argument of \cmd\@combinepage, which must be a Boolean, determines
% if the footnotes are to be combined into this page.
% 
% \changes{4.1b}{2008/08/04}{Change \cs{@combinepage} to \cs{@combinepage} with argument}
% \changes{4.1b}{2008/08/04}{(AO, 452) Support length checking: show size of shipped out text.}
% \changes{4.1n}{2009/12/02}{More diagnostics of column balancing}
%
% QUERY: In the following, if \cmd\box\cmd\footins\ is not void, its contents are lost. Can this ever happen?
%    \begin{macrocode}
\def\@combinepage#1{%
 \ltxgrid@foot@info@sw{\class@info{\string\@combinepage\string#1}}{}%
 \@ifvoid\pagesofar{}{%
  \setbox\@outputbox\vbox{%
   \unvbox\pagesofar
   \marry@baselines
   \unvbox\@outputbox
  }%
 }%
 #1{%
  \@ifvoid\footsofar{}{%
%    \end{macrocode}
% At this point, \cmd\footins\ is empty; all of the footnotes have been combined into \cmd\footsofar.
%    \begin{macrocode}
   \show@box@size{Combining page footnotes}\footsofar
   \setbox\footins\box\footsofar
%    \end{macrocode}
% Depending on the page grid, we compose the footnotes for placement on the page. 
%    \begin{macrocode}
   \compose@footnotes
   \@combineinserts\@outputbox\footins
  }%
 }{%
%    \end{macrocode}
%QUERY: The following line was removed, probably to fix a bug. When was this done?
%    \begin{verbatim}
% \global\setbox\footins\box\footsofar
%    \end{verbatim}
%    \begin{macrocode}
 }%
}%
%    \end{macrocode}
% \end{macro}
%
% \begin{macro}{\@cflt}
% \begin{macro}{\@cflb}
% \changes{4.1b}{2008/08/04}{}
% We modify \LaTeX's \cmd\@cflt\ and \cmd\@cflb\ to remove the unwanted glue with \cmd\unskip.
%    \begin{macrocode}
\def \@cflt{%
 \let \@elt \@comflelt
 \setbox\@tempboxa \vbox{}%
 \@toplist
 \setbox\@outputbox \vbox{%
  \boxmaxdepth \maxdepth
  \unvbox\@tempboxa\unskip
  \topfigrule\vskip \textfloatsep
  \unvbox\@outputbox
 }%
 \let\@elt\relax
 \xdef\@freelist{\@freelist\@toplist}%
 \global\let\@toplist\@empty
}%
\def \@cflb {%
 \let\@elt\@comflelt
 \setbox\@tempboxa \vbox{}%
 \@botlist
 \setbox\@outputbox \vbox{%
  \unvbox\@outputbox
  \vskip \textfloatsep\botfigrule
  \unvbox\@tempboxa\unskip
 }%
 \let\@elt\relax
 \xdef\@freelist{\@freelist\@botlist}%
 \global \let \@botlist\@empty
}%
%    \end{macrocode}
% \end{macro}
% \end{macro}
%
% \begin{macro}{\@combinedblfloats}
% We modify \LaTeX's \cmd\@combinedblfloats\ to be more appropriate for incremental page building:
% we \cmd\unvbox\ the \cmd\@outputbox.
%    \begin{macrocode}
\def\@combinedblfloats{%
 \@ifx@empty\@dbltoplist{}{%
  \setbox\@tempboxa\vbox{}%
  \let\@elt\@comdblflelt\@dbltoplist
  \let\@elt\relax\xdef\@freelist{\@freelist\@dbltoplist}%
  \global\let\@dbltoplist\@empty
  \setbox\@outputbox\vbox{%
   %\boxmaxdepth\maxdepth   %% probably not needed, CAR
   \unvbox\@tempboxa\unskip
   \@ifnum{\@dbltopnum>\m@ne}{\dblfigrule}{}%FIXME: how is \@dbltopnum maintained?
   \vskip\dbltextfloatsep
   \unvbox\@outputbox
  }%
 }%
}%
%    \end{macrocode}
% \end{macro}
%
% \begin{macro}{\set@column@hsize}%
% The procedure \cmd\set@column@hsize\ takes care of setting up the horizontal dimensions
% for the current page grid. The present routine will certainly not be adequate for more
% complex page layouts (e.g., with a side column), but works for the common ones.
% \changes{4.0a}{2001/06/18}{Introduce \cs{set@marry@height} }
%    \begin{macrocode}
\def\set@column@hsize#1{%
 \pagegrid@col#1%
 \global\columnwidth\textwidth
 \global\advance\columnwidth\columnsep
 \global\divide\columnwidth\pagegrid@col
 \global\advance\columnwidth-\columnsep
 \global\hsize\columnwidth
 \global\linewidth\columnwidth
 \skip@\baselineskip\advance\skip@-\topskip
 \@ifnum{\pagegrid@col>\@ne}{\set@marry@skip}{}%
}%
%    \end{macrocode}
% \end{macro}
%
% \begin{macro}{\set@colht}%
% \begin{macro}{\set@colroom}%
% \begin{macro}{\set@vsize}%
% \begin{macro}{\set@adj@colht}%
% The story of \cmd\textheight, \cmd\@colht, \cmd\@colroom, and \cmd\vsize.
%
% \cmd\textheight---height of the text column. Not a running parameter, however, each time a page is 
% shipped out, the \cmd\textheight\ could in principle be altered. This must be done before
%
% \cmd\@colht---\cmd\textheight\ minus the height of any full-page-width floats. The latter are committed
% only just after shipping out, and only if we are in a multicolumn page grid.
% Therefore, \cmd\@colht\ should be set after a \cmd\shipout\ (by \cmd\@outputpage) and
% will be adjusted when full-page-width floats are committed to the fresh page by \cmd\do@startpage.
%
% \cmd\@colroom---\cmd\@colht\ (adjusted by \cmd\pagesofar) minus the height of any column-width floats.
% The latter are committed anywhere on the page, at which point \cmd\@colroom\ must be adjusted.
% Therefore, \cmd\@colroom\ should be set (by \cmd\set@colroom) whenever a column is prepared (by ).
%FIXME: committed (by \cmd\output@column@) and
% will be adjusted (by \cmd\@add@float\ or \cmd\do@startcolumn) whenever a float is committted to the column.
%
% \cmd\vsize---\cmd\@colroom.
% Therefore, \cmd\vsize\ should be set (by \cmd\set@vsize) whenever
% the \cmd\@colroom\ is set (by \cmd\set@colroom) or adjusted (by \cmd\@add@float\ or \cmd\do@startcolumn)
%FIXME: or when the \cmd\pagesofar\ box is changed (after invoking \cmd\open@column@).
%
% Question: what if there are committed floats? Footnotes?
% Answer: full-page-width floats are only committed at top, and they are already reckoned with in \cmd\@colht.
% Column-width committed floats are incorporated by \cmd\@makecolumn.
%
% As to footnotes, our scheme is to keep the \cmd\footins\ insert register up to date, and to use the insert mechanism 
% to ensure room for footnotes. 
% When a change is made to the page grid, the footnotes will need to be propagated back into the MVL. 
%
% Note: FIXME: adjusting for \cmd\pagesofar\ is done at not quite the right time. I need to reexamine \cmd\set@colht,
% because \cmd\@dbltoplist\ and \cmd\pagesofar\ really should be on the same footing.
% Perhaps \cmd\@colht\ and \cmd\@colroom\ should both deal with their respective ``lists'' in the same way?
%
% These concerns will be particularly germane if we ever extend this package to deal with full-page-width floats
% placed at the bottom of the page, or committed on the same page as called out.
%
% It occurs to me that we should ditch \cmd\set@colroom\ and only ever execute \cmd\set@colht, 
% which sets \cmd\@colroom\ as a side effect.
% If so, we can make \cmd\@colht\ take \cmd\pagesofar\ into account, as it should. Then \cmd\@colht\ will return to its 
% original significance as the value that \cmd\@colroom\ is set to after a column is committed.
%
% On the other hand, why not simply forget all this caching and (re-)calculate \cmd\vsize\ as late as possible?
% Particularly, \cmd\@colht\ is an artifact of the old way of doing things, where once it was set, it would never change.
% 
% \changes{4.1n}{2009/12/02}{More diagnostics of column balancing}
%    \begin{macrocode}
\def\set@colht{%
 \set@adj@textheight\@colht
 \global\let\enlarge@colroom\@empty
 \set@colroom
}%
\def\set@adj@textheight#1{%
 \ltxgrid@info@sw{\class@info{\string\set@adj@textheight\string#1}\saythe\textheight}{}%
 #1\textheight
 \def\@elt{\adj@page#1}%
 \@booleantrue\firsttime@sw\@dbltoplist
 \let\@elt\relax
 \global#1#1\relax
 \ltxgrid@info@sw{\saythe#1}{}%
}%
\def\set@colroom{%
 \ltxgrid@info@sw{\class@info{\string\set@colroom}}{}%
 \set@adj@colht\@colroom
 \@if@empty\enlarge@colroom{}{%
  \global\advance\@colroom\enlarge@colroom\relax
  \ltxgrid@info@sw{\saythe\@colroom}{}%
 }%
 \@ifdim{\@colroom>\topskip}{}{%
  \ltxgrid@info{Not enough room: \string\@colroom=\the\@colroom; increasing to \the\topskip}%
  \@colroom\topskip
 }%
 \global\@colroom\@colroom
%<ignore> \ltxgrid@info@sw{\class@info{\string\set@colroom\string\vsize=\string\colroom}\saythe\vsize}{}%
 \set@vsize
}%
%
\def\set@vsize{%
 \global\vsize\@colroom
 \ltxgrid@info@sw{\class@info{\string\set@vsize\string\vsize=\string\colroom}\saythe\vsize}{}%
}%
%    \end{macrocode}
% \changes{4.1n}{2009/12/02}{More diagnostics of column balancing}
%    \begin{macrocode}
\def\set@adj@colht#1{%
 #1\@colht
 \ltxgrid@info@sw{\class@info{\string\set@adj@colht\string#1-\string\pagesofar}\saythe#1}{}%
 \@ifvoid\pagesofar{}{%
  \advance#1-\ht\pagesofar\advance#1-\dp\pagesofar
  \ltxgrid@info@sw{\class@info{\string\pagesofar}\saythe#1}{}%
 }%
 \def\@elt{\adj@column#1}%
 \@booleantrue\firsttime@sw\@toplist
 \@booleantrue\firsttime@sw\@botlist
 \let\@elt\relax
}%
\def\adj@column#1#2{%
 \advance#1-\ht#2%
 \advance#1-\firsttime@sw{\textfloatsep\@booleanfalse\firsttime@sw}{\floatsep}%
 \ltxgrid@info@sw{\class@info{\string\adj@column\string#1-\string#2}\saythe#1}{}%
}%
\def\adj@page#1#2{%
 \advance#1-\ht#2%
 \advance#1-\firsttime@sw{\dbltextfloatsep\@booleanfalse\firsttime@sw}{\dblfloatsep}%
 \ltxgrid@info@sw{\class@info{\string\adj@page\string#1-\string#2}\saythe#1}{}%
}%
\def\set@adj@box#1#2{%
 \@ifvoid#2{}{%
  \advance#1-\ht#2\advance#1-\dp#2%
  \@booleantrue\temp@sw
  \ltxgrid@foot@info@sw{\class@info{\string\set@adj@box\string#2}\saythe#1}{}%
 }%
}%
%    \end{macrocode}
% \end{macro}
% \end{macro}
% \end{macro}
% \end{macro}
%
% \begin{macro}{\@outputpage@tail}%
% \changes{4.1b}{2008/08/04}{Procedure \cs{@outputpage@tail} tailpatches \cs{@outputpage}}%
% In \cmd\@outputpage@tail, we set \cmd\@colht\ and the float placement parameters
% (this is the one point where it is appropriate to set \cmd\@colht).
% At \cmd\do@startpage\ time, we adjust \cmd\@colht's value to reflect committed 
% full-page-width floats.
%
% Note: with a correctly written output routine, a call to \cmd\@outputpage\ will inevitably be 
% followed by a call to \cmd\do@startpage, so these procedure calls would be unneeded.
%    \begin{macrocode}
\appdef\@outputpage@tail{%
 \set@colht          % FIXME: needed?
 \@floatplacement    % FIXME: needed?
 \@dblfloatplacement % FIXME: needed?
}%
%    \end{macrocode}
% \end{macro}
%
% \begin{macro}{balance@2}
% \changes{4.1b}{2008/08/04}{Procedure \cs{balance@2} defined more transparently}%
% We define procedures for balancing columns in a multicolumn layout.
% For now, we define only one: a procedure for the two-column grid.
% All others will simply \cmd\relax\ out.
%
% The following code defines \verb+\balance@2+ without all the clunky \cmd\csname\ commands
% in the replacement part,
% which appears on the right-hand side of the assignment to \cmd\toks@.
%
% The method is straightforward: balance the two columns of text, and balance the footnotes.
% Later on, \cmd\@combineinserts\ will be called to place the footnotes after the now-balanced columns.
%
% \changes{4.1f}{2009/07/14}{(AO, 519) \cs{footins} content must be preserved and reintegrated}
% It was necessary to deal with the case where \cmd\box\cmd\footsofar\ was not empty
% upon execution of this balancing code. We store it away in \cmd\box\cmd\footins\ and
% add it back in afterwards.
%
% Here is a conundrum: if we switch between single-, two-, and three-column page grids:
% On what measure should the footnotes be set?
%    \begin{macrocode}
\begingroup
 \catcode`\1=\cat@letter
 \catcode`\2=\cat@letter
%    \end{macrocode}
% \cmd\toks@\ contains the repacement part for an effective \cmd\def\cs{balance@2}.
%    \begin{macrocode}
 \toks@{%
%    \end{macrocode}
% \cmd\balance@two, by side effect, strips footnotes into \cmd\box\cmd\footins. 
%    \begin{macrocode}
  \setbox\footins\box\footsofar
  \balance@two\col@1\@outputbox
%    \end{macrocode}
% We ensure that the box assignments are global. 
%    \begin{macrocode}
  \global\setbox\col@1\box\col@1
  \global\setbox\@outputbox\box\@outputbox
%    \end{macrocode}
% The following line puts all footnotes into the footnote galley, \cmd\footsofar.
%    \begin{macrocode}
  \combine@foot@inserts\footsofar\footins
 }%
 \aftergroup\def\aftergroup\balance@2\expandafter
\endgroup\expandafter{\the\toks@}%
%    \end{macrocode}
% \end{macro}
%
% \begin{macro}{\balance@two}
% The procedure \cmd\balance@two\ takes two columns and balances them; in the process it removes
% any footnotes that may be present to a place of safety \cmd\footsofar, 
% for later placement at the foot of the shipped-out page. 
% The box register \cmd\box\cmd\@ne\ is the aggregate of all columns.
% The box register \cmd\box\cmd \z@\ is the last column.
% The box register \cmd\box\cmd\tw@\ is the first column.
% The \cmd\dimen\ register \cmd\dimen@\  is the trial value to \cmd\vsplit\ to, 
% initially half the height of \cmd\box\cmd\@ne.
% The \cmd\dimen\ register \cmd\dimen@i\ is the increment for the next trial;
% its initial value is equal to the initial value of \cmd\dimen@.
% The \cmd\dimen\ register \cmd\dimen@ii\ is the difference of the heights of the two columns.
%
% The procedure uses a binary search for that value of \cmd\dimen@\ which is stable to within .5\cmd\p@\ and which
% makes the last column be shorter than the others.
%
% This procedure can be extended to multiple columns simply by changing it to execute \cmd\vsplit\ multiple
% times (one less than the total number of columns in the page layout) and to calculating \cmd\dimen@ii\
% to be the difference of the heights of last column and the \cmd\dimen@.
% Upon termination of the search, one would excute the \cmd\vsplit s once again, this time 
% using the actual \cmd\col@\ box registers to store the 
% balanced columns, thereby clobbering their former contents.
%
% Bug Note:
% as originally written, this macro had a bug, which is well worth avoiding under similar circumstances anywhere.
% So, learn from the mistakes of others, as they say.
% In trying to remove the depth of the boxes created via \cmd\vsplit\ within the \cmd\loopwhile\ control,
% I originally coded
% \cmd\unvbox
% \cmd\z@\
% \cmd\setbox
% \cmd\z@
% \cmd\lastbox\
% \cmd\dimen@
% \cmd\dp
% \cmd\z@\
% \cmd\box
% \cmd\z@\
% \cmd\vskip-%
% \cmd\dimen@.
% The error here is that the (horizontal) shift of the last box in the vertical list will be lost in the process.
% Simply put, \cmd\setbox\cmd\z@\cmd\lastbox\ fails to retain the shift of the box node in the vertical list,
% and when it is put down again via \cmd\box\cmd\z@, it will no longer have the correct shift.
%
% This bug affected things placed in the MVL with \cmd\moveleft, \cmd\moveright, \cmd\parshape, and 
% \cmd\hangindent, as well as things shifted by \TeX's primitive mechanisms.
%
% A superior strategy for removing the depth of the last line of the list is more expensive, but safer:
% make a separate copy of the list, measure the depth of the last box as above, but then discard
% the list, retaining only the value of the dimension.
% 
% Note that this procedure will not work if the material within is excessively chunky.
% A particular failure mode exists where none of the material is allocated to the last (right) column.
% We detect this case and revert to unbalanced columns.
%
% Another failure mode is where a large chunk occurs at the beginning of the composite box.
% In this case, the left column may fill up even when \cmd\dimen@\ is very small.
% If this configuration leaves the left column longer than the right, then we are done,
% but \cmd\dimen@\ by  no means represents the height of either finished box.
%
% Therefore the last step in the process is to rebox the two columns to a common height determined
% independently of the balancing process.
%
% The dimension involved is checked against the current \cmd\@colroom\ to guard against the case where
% excessive material happens to fall in either column.
% \changes{4.1b}{2008/08/04}{Change \cs{set@colroom} to \cs{set@colht}}
% \changes{4.1n}{2009/12/02}{More diagnostics of column balancing}
% \changes{4.1n}{2010/01/02}{(AO, 571) coding convention: use \cs{bgroup}, \cs{egroup} (instead of braces) when a box is being built}%
% \changes{4.1n}{2010/01/02}{(AO, 571) Footnotes, when columns are balanced or when they are composed with their column}%
% \changes{4.1n}{2010/01/02}{(AO, 571) Change \cs{balance@two}'s balancing algorithm to more successfully balance extremely short columns.}
%    \begin{macrocode}
\def\balance@two#1#2{%
 \ltxgrid@info@sw{\class@info{\string\balance@two\string#1\string#2}}{}%
 \outputdebug@sw{\trace@scroll{\showbox#1\showbox#2}}{}%
%    \end{macrocode}
% The first step is to recover the footnotes from the bottoms of the two columns (globally, into \cmd\footsofar)
% and to combine the text into \cmd\box\cmd\@ne, but without voiding either of the argument boxes. 
%    \begin{macrocode}
 \setbox\thr@@\copy\footsofar
 \setbox\@ne\vbox\bgroup
  \@ifvoid{#1}{}{%
   \recover@column#1\footsofar\column@recovered\footins@recovered
   \@ifvoid{#2}{}{\marry@baselines}%
  }%
  \@ifvoid{#2}{}{%
   \recover@column#2\footsofar\column@recovered\footins@recovered
  }%
 \egroup
 \outputdebug@sw{\trace@scroll{\showbox\@ne}}{}%
 \ltxgrid@foot@info@sw{\trace@scroll{\showbox\footsofar}}{}%
%    \end{macrocode}
% Hereunder, \cmd\dimen@\ is the split value. We adjust it until the step size is small enough, while the split is acceptable. 
% Also, \cmd\dimen@i\ is the step size. Once this value is greater than a half point, we must iterate. 
%    \begin{macrocode}
 \dimen@\ht\@ne\divide\dimen@\tw@
 \dimen@i\dimen@
 \vbadness\@M
 \vfuzz\maxdimen
 \splittopskip\topskip
 \loopwhile{%
  \setbox\z@\copy\@ne\setbox\tw@\vsplit\z@ to\dimen@
  \remove@depth\z@\remove@depth\tw@
%    \end{macrocode}
% The following line would provide a diagnostic of the iterations of column balancing, were we to use it.
%    \begin{verbatim}
%  \outputdebug@sw{\trace@scroll{\showbox\tw@\showbox\z@}}{}%
%    \end{verbatim}
% Hereunder, \cmd\dimen@ii\ is used to reckon the difference in height between the left box and the right. 
%    \begin{macrocode}
  \dimen@ii\ht\tw@\advance\dimen@ii-\ht\z@
  \dimen@i=.5\dimen@i
  \ltxgrid@info@sw{\saythe\dimen@\saythe\dimen@i\saythe\dimen@ii}{}%
%    \end{macrocode}
% If the columns are within a half-point of each other, 
%    \begin{macrocode}
  \@ifdim{\dimen@ii<.5\p@}{%
   \@ifdim{\dimen@ii>-.5\p@}%
  }{%
   \false@sw
  }%
%    \end{macrocode}
% The above results in a Boolean, which now chooses between the following two brace-delimited clauses.
% If the step size is less than a half-point, then terminate the loop. 
%    \begin{macrocode}
  {%
   \true@sw
  }{%
   \@ifdim{\dimen@i<.5\p@}%
  }%
%    \end{macrocode}
% The above results in a Boolean, which now chooses between the following two brace-delimited clauses.
% The true-part terminates the loop, otherwise iterate. 
%    \begin{macrocode}
  {%
   \false@sw
  }%
  {%
%    \end{macrocode}
% For the next iteration, the candidate split dimension \cmd\dimen@\ will be one step larger 
% if the height of the left box is less than that of the right box. 
% Otherwise it will be one step smaller. 
%    \begin{macrocode}
   \advance\dimen@\@ifdim{\dimen@ii<\z@}{}{-}\dimen@i
   \true@sw
  }%
 }%
%    \end{macrocode}
% The loop has terminated.
%    \begin{macrocode}
 \ltxgrid@info@sw{\saythe\dimen@\saythe\dimen@i\saythe\dimen@ii}{}%
%    \end{macrocode}
% The algorithm has failed to find a satisfactory result if
% the left column is of non-zero height and the right column is of zero height. 
%    \begin{macrocode}
 \@ifdim{\ht\z@=\z@}{%
  \@ifdim{\ht\tw@=\z@}%
 }{%
  \true@sw
 }%
%    \end{macrocode}
% The \cmd\false@sw\ branch is executed if the algorithm has failed. 
% We restore the original boxes. 
%    \begin{macrocode}
 {%
 }{%
  \ltxgrid@info{Unsatifactorily balanced columns: giving up}%
  \setbox\tw@\box#1%
  \setbox\z@ \box#2%
  \global\setbox\footsofar\box\thr@@
 }%
 \setbox\tw@\vbox{\unvbox\tw@\vskip\z@skip}%
 \setbox\z@ \vbox{\unvbox\z@ \vskip\z@skip}%
 \set@colht
 \dimen@\ht\z@\@ifdim{\dimen@<\ht\tw@}{\dimen@\ht\tw@}{}%
 \@ifdim{\dimen@>\@colroom}{\dimen@\@colroom}{}%
 \ltxgrid@info@sw{\saythe{\ht\z@}\saythe{\ht\tw@}\saythe\@colroom\saythe\dimen@}{}%
 \setbox#1\vbox to\dimen@{\unvbox\tw@\unskip\raggedcolumn@skip}%
 \setbox#2\vbox to\dimen@{\unvbox\z@ \unskip\raggedcolumn@skip}%
 \outputdebug@sw{\trace@scroll{\showbox#1\showbox#2}}{}%
}%
%    \end{macrocode}
% Procedure \cmd\remove@depth\ rearranges the given (vertical) box register so that it has zero depth. 
%    \begin{macrocode}
\def\remove@depth#1{%
  \setbox#1\vbox\bgroup
   \unvcopy#1%
   \setbox\z@\vbox\bgroup
    \unvbox#1%
    \setbox\z@\lastbox
    \aftergroup\kern\aftergroup-\expandafter
   \egroup
   \the\dp\z@\relax
  \egroup
}%
%    \end{macrocode}
% \end{macro}
%
% \changes{4.1n}{2010/01/02}{(AO, 571) Abandon \cs{recover@footins} in favor of \cs{recover@column}}%
% Procedure \cmd\recover@column\ is a utility to separate a column box into text and footnotes;
% the former being contributed to the current (vertical) list, the latter appended to the given register, usually \cmd\footsofar. 
% 
% Argument \verb+#1+ is the input: it should be a \cmd\vbox, and it remains unaltered. 
% Argument \verb+#2+ is the box into which to (globally) add the footnotes, usually \cmd\footsofar.
% Arguments \verb+#3+ and \verb+#4+ are scratch box registers to use in this calculation.
% As a side effect, \verb+#3+ will be unboxed into whatever vertical mode we are in at the moment (should be a \cmd\vbox). 
%    \begin{macrocode}
\def\recover@column#1#2#3#4{%
 \ltxgrid@info@sw{\class@info{\string\recover@column\string#1\string#2\string#3\string#4}}{}%
 \setbox#4\vbox{\unvcopy#1}%
 \ltxgrid@foot@info@sw{\trace@scroll{\showbox#4}}{}%
 \dimen@\ht#4%
 \ltxgrid@foot@info@sw{\saythe\dimen@}{}%
 \setbox#4\vbox\bgroup
  \unvbox#4\unskip
%    \end{macrocode}
% We now strip the footnotes from the bottom of this box, adding them to \cmd\footsofar.
% The method relies on a signal, consisting of a complementary pair of kerns, placed at the bottom of the box by \cmd\@combineinserts. 
%    \begin{macrocode}
  \dimen@i\lastkern\unkern\advance\dimen@i\lastkern
  \@ifdim{\dimen@i=\z@}{%
   \dimen@i\lastkern\unkern
   \ltxgrid@foot@info@sw{\saythe\dimen@i}{}%
   \aftergroup\dimen@i
   \expandafter\egroup\the\dimen@i\relax
  }{%
   \egroup
  }%
%    \end{macrocode}
% Split the column into \verb+#3+ and the footnote into \verb+#4+.
% Append the footnote to \verb+#2+.
%    \begin{macrocode}
 \@ifdim{\dimen@i<\z@}{%
  \advance\dimen@\dimen@i
  \ltxgrid@foot@info@sw{\saythe\dimen@i\saythe\dimen@}{}%
  \splittopskip\z@skip
  \global\setbox#3\vsplit#4 to\dimen@
  \global\setbox#4\vbox{\unvbox#4}%
  \ltxgrid@foot@info@sw{\trace@scroll{\showbox#1\showbox#2\showbox#3\showbox#4}}{}%
  \global\setbox#2\vbox\bgroup\unvbox#2\vskip\z@skip\unvbox#4\egroup
 }{%
%    \end{macrocode}
% What if \cmd\dimen@i\ is zero? 
% In that case, \cmd\setbox\#3\cmd\box\#4, and do not touch \cmd\box\#2.
%    \begin{macrocode}
  \setbox#3\box#4%
  \ltxgrid@foot@info@sw{\trace@scroll{\showbox#1\showbox#2\showbox#3\showbox#4}}{}%
 }%
 \unvbox#3%
 \loopwhile{\dimen@\lastskip\@ifdim{\dimen@>\z@}{\unskip\true@sw}{\false@sw}}%
}%
\def\recover@column@null#1#2#3#4{%
 \unvcopy#1%
}%
%    \end{macrocode}
% \end{macro}
%
% \begin{macro}{\@begindocumenthook}
% Initialization:
% we initialize to the page grid named ``one''.
% If the class decides to initially set type in a different grid, it
% should execute these same commands, but changing the first to the appropriate procedure.
%
% Note that the point where this sequence is executed would be an excellent place to arrange for 
% floats to be committed to the first page of a document. 
% That is, we execute \cmd\do@startpage, which triggers \cmd\do@startcolumn.
%
% FIXME: it should be the job of the page grid to determine the procedure to execute at 
% the start of the job. Make this a hook.
%    \begin{macrocode}
\prepdef\@begindocumenthook{%
 \open@column@one\@ne
 \set@colht
 \@floatplacement
 \@dblfloatplacement
}%
%    \end{macrocode}
% \end{macro}
%
% Comment: our technique of balancing columns is severely limited, because it cannot properly work
% with \env{longtable}, which places material at the bottom and top of the column break.
%
% The proper way to handle a grid change in the middle of the page is to accumulate all the material for
% an entire article (or chapter) and then assemble finished pages therefrom. This approach is fundamentally
% superior for complex layouts: it corresponds to real-world workflows.
% Such a scheme is an excellent subject for another \LaTeX\ package.
%
%
% \subsection{Patches for the longtable package}%
%
% \LaTeX's ``required'' package \classname{longtable} (written by David P. Carlilsle),
% which is part of /latex/required/tools, is incmpatible with both
% \LaTeX's ``required'' package \classname{multicol} and with
% \LaTeX's native \cmd\twocolumn\ capability. There is no essential reason
% for this incompatability, aside from implementation details, and the
% \classname{ltxgrid} package gives us the ability to lift them.
%
% Only four of \classname{longtable}'s procedures require rewriting:
% \cmd\longtable,
% \cmd\endlongtable,
% \cmd\LT@start, and
% \cmd\LT@end@hd@ft. 
% The procedure \cmd\switch@longtable\ checks against their expected meanings
% and, if all is as expected, applies the patches.
% In the process, we simplify things considerably and also make them more
% secure.
%
% Why does \classname{longtable} need to access the output routine, anyway?
% What it comes down to, is what happens when a pagebreak falls within a
% long table. If this happens, we would like to append a row at the bottom of
% the broken table and add a row at the top of the next page.
%
% These things can be accomodated easily by the \classname{ltxgrid} output
% routine hooks.
%
% \begin{macro}{\longtable}
%    \begin{macrocode}
\def\longtable@longtable{%
 \par
 \ifx\multicols\@undefined\else\ifnum\col@number>\@ne\@twocolumntrue\fi\fi
 \if@twocolumn\LT@err{longtable not in 1-column mode}\@ehc\fi
 \begingroup
 \@ifnextchar[\LT@array{\LT@array[x]}%
}%
\def\longtable@new{%
 \par
  \@ifnextchar[\LT@array{\LT@array[x]}%
}%
%    \end{macrocode}
% \end{macro}
%
% \begin{macro}{\endlongtable}
%    \begin{macrocode}
\def\endlongtable@longtable{%
  \crcr
  \noalign{%
    \let\LT@entry\LT@entry@chop
    \xdef\LT@save@row{\LT@save@row}}%
  \LT@echunk
  \LT@start
  \unvbox\z@
  \LT@get@widths
  \if@filesw
    {\let\LT@entry\LT@entry@write\immediate\write\@auxout{%
      \gdef\expandafter\noexpand
        \csname LT@\romannumeral\c@LT@tables\endcsname
          {\LT@save@row}}}%
  \fi
  \ifx\LT@save@row\LT@@save@row
  \else
    \LT@warn{Column \@width s have changed\MessageBreak
             in table \thetable}%
    \LT@final@warn
  \fi
  \endgraf\penalty -\LT@end@pen
  \endgroup
  \global\@mparbottom\z@
  \pagegoal\vsize
  \endgraf\penalty\z@\addvspace\LTpost
  \ifvoid\footins\else\insert\footins{}\fi
}%
%    \end{macrocode}
% \changes{4.1a}{2008/06/29}{Repair error in \cs{endlongtable@new} involving \cs{@ifx}: argument not delimited.}
%    \begin{macrocode}
\def\endlongtable@new{%
  \crcr
  \noalign{%
   \let\LT@entry\LT@entry@chop
   \xdef\LT@save@row{\LT@save@row}%
  }%
  \LT@echunk
  \LT@start
  \unvbox\z@
  \LT@get@widths
  \@if@sw\if@filesw\fi{%
   {%
    \let\LT@entry\LT@entry@write
    \immediate\write\@auxout{%
     \gdef\expandafter\noexpand\csname LT@\romannumeral\c@LT@tables\endcsname
     {\LT@save@row}%
    }%
   }%
  }{}%
  \@ifx{\LT@save@row\LT@@save@row}{}{%
   \LT@warn{%
    Column \@width s have changed\MessageBreak in table \thetable
   }\LT@final@warn
  }%
  \endgraf
  \nobreak
  \box\@ifvoid\LT@lastfoot{\LT@foot}{\LT@lastfoot}%
 \global\@mparbottom\z@
 \endgraf
 \LT@post
}%
%    \end{macrocode}
% \end{macro}
%
% \begin{macro}{\LT@start}
%    \begin{macrocode}
\def\LT@start@longtable{%
  \let\LT@start\endgraf
  \endgraf\penalty\z@\vskip\LTpre
  \dimen@\pagetotal
  \advance\dimen@ \ht\ifvoid\LT@firsthead\LT@head\else\LT@firsthead\fi
  \advance\dimen@ \dp\ifvoid\LT@firsthead\LT@head\else\LT@firsthead\fi
  \advance\dimen@ \ht\LT@foot
  \dimen@ii\vfuzz
  \vfuzz\maxdimen
    \setbox\tw@\copy\z@
    \setbox\tw@\vsplit\tw@ to \ht\@arstrutbox
    \setbox\tw@\vbox{\unvbox\tw@}%
  \vfuzz\dimen@ii
  \advance\dimen@ \ht
        \ifdim\ht\@arstrutbox>\ht\tw@\@arstrutbox\else\tw@\fi
  \advance\dimen@\dp
        \ifdim\dp\@arstrutbox>\dp\tw@\@arstrutbox\else\tw@\fi
  \advance\dimen@ -\pagegoal
  \ifdim \dimen@>\z@\vfil\break\fi
      \global\@colroom\@colht
  \ifvoid\LT@foot\else
    \advance\vsize-\ht\LT@foot
    \global\advance\@colroom-\ht\LT@foot
    \dimen@\pagegoal\advance\dimen@-\ht\LT@foot\pagegoal\dimen@
    \maxdepth\z@
  \fi
  \ifvoid\LT@firsthead\copy\LT@head\else\box\LT@firsthead\fi
%    \end{macrocode}
% At some point before version 4.11, the \cmd\nobreak\ was added. 
%    \begin{macrocode}
\nobreak
%    \end{macrocode}
%
%    \begin{macrocode}
  \output{\LT@output}%
}%
\def\LT@start@new{%
 \let\LT@start\endgraf
 \endgraf
 \markthr@@{}%
 \LT@pre
 \@ifvoid\LT@firsthead{\LT@top}{\box\LT@firsthead\nobreak}%
 \mark@envir{longtable}%
}%
%    \end{macrocode}
% \end{macro}
%
% \begin{macro}{\LT@end}
%    \begin{macrocode}
\def\LT@end@hd@ft@longtable#1{%
 \LT@echunk
 \ifx\LT@start\endgraf
  \LT@err{Longtable head or foot not at start of table}{Increase LTchunksize}%
 \fi
 \setbox#1\box\z@
 \LT@get@widths\LT@bchunk
}%
\def\LT@end@hd@ft@new#1{%
 \LT@echunk
 \@ifx{\LT@start\endgraf}{%
  \LT@err{Longtable head or foot not at start of table}{Increase LTchunksize}%
 }%
 \global\setbox#1\box\z@
 \LT@get@widths
 \LT@bchunk
}%
%    \end{macrocode}
% \end{macro}
%
% \begin{macro}{\LT@array}
%
% 
%    \begin{macrocode}
\def\LT@array@longtable[#1]#2{%
  \refstepcounter{table}\stepcounter{LT@tables}%
  \if l#1%
    \LTleft\z@ \LTright\fill
  \else\if r#1%
    \LTleft\fill \LTright\z@
  \else\if c#1%
    \LTleft\fill \LTright\fill
  \fi\fi\fi
  \let\LT@mcol\multicolumn
  \let\LT@@tabarray\@tabarray
  \let\LT@@hl\hline
  \def\@tabarray{%
    \let\hline\LT@@hl
    \LT@@tabarray}%
  \let\\\LT@tabularcr\let\tabularnewline\\%
  \def\newpage{\noalign{\break}}%
  \def\pagebreak{\noalign{\ifnum`}=0\fi\@testopt{\LT@no@pgbk-}4}%
  \def\nopagebreak{\noalign{\ifnum`}=0\fi\@testopt\LT@no@pgbk4}%
  \let\hline\LT@hline \let\kill\LT@kill\let\caption\LT@caption
  \@tempdima\ht\strutbox
  \let\@endpbox\LT@endpbox
  \ifx\extrarowheight\@undefined
    \let\@acol\@tabacol
    \let\@classz\@tabclassz \let\@classiv\@tabclassiv
    \def\@startpbox{\vtop\LT@startpbox}%
    \let\@@startpbox\@startpbox
    \let\@@endpbox\@endpbox
    \let\LT@LL@FM@cr\@tabularcr
  \else
    \advance\@tempdima\extrarowheight
    \col@sep\tabcolsep
    \let\@startpbox\LT@startpbox\let\LT@LL@FM@cr\@arraycr
  \fi
  \setbox\@arstrutbox\hbox{\vrule
    \@height \arraystretch \@tempdima
    \@depth \arraystretch \dp \strutbox
    \@width \z@}%
  \let\@sharp##\let\protect\relax
   \begingroup
    \@mkpream{#2}%
    \xdef\LT@bchunk{%
       \global\advance\c@LT@chunks\@ne
       \global\LT@rows\z@\setbox\z@\vbox\bgroup
       \LT@setprevdepth
%    \end{macrocode}
% At some point before version 4.11, the \cmd\noexpand\ was added. 
% We need not change our own version, because we did it right, back in 1998 (using \cmd\appdef). 
%    \begin{macrocode}
       \tabskip\LTleft \noexpand\halign to\hsize\bgroup
      \tabskip\z@ \@arstrut \@preamble \tabskip\LTright \cr}%
  \endgroup
  \expandafter\LT@nofcols\LT@bchunk&\LT@nofcols
  \LT@make@row
  \m@th\let\par\@empty
  \everycr{}\lineskip\z@\baselineskip\z@
  \LT@bchunk}%
\def\LT@LR@l{\LTleft\z@   \LTright\fill}%
\def\LT@LR@r{\LTleft\fill \LTright\z@  }%
\def\LT@LR@c{\LTleft\fill \LTright\fill}%
\def\LT@array@new[#1]#2{%
 \refstepcounter{table}\stepcounter{LT@tables}%
 \table@hook
 \LTleft\fill \LTright\fill
 \csname LT@LR@#1\endcsname
 \let\LT@mcol\multicolumn
 \let\LT@@hl\hline
 \prepdef\@tabarray{\let\hline\LT@@hl}%
 \let\\\LT@tabularcr
 \let\tabularnewline\\%
 \def\newpage{\noalign{\break}}%
 \def\pagebreak{\noalign{\ifnum`}=0\fi\@testopt{\LT@no@pgbk-}4}%
 \def\nopagebreak{\noalign{\ifnum`}=0\fi\@testopt\LT@no@pgbk4}%
 \let\hline\LT@hline
 \let\kill\LT@kill
 \let\caption\LT@caption
 \@tempdima\ht\strutbox
 \let\@endpbox\LT@endpbox
 \@ifxundefined\extrarowheight{%
  \let\@acol\@tabacol
  \let\@classz\@tabclassz
  \let\@classiv\@tabclassiv
  \def\@startpbox{\vtop\LT@startpbox}%
  \let\@@startpbox\@startpbox
  \let\@@endpbox\@endpbox
%    \end{macrocode}
% Because \classname{ltxutil} patches \LaTeX's \cmd\@tabularcr and \cmd\@xtabularcr,
% we must restore these procedures in the scope of \env{longtable}.
% Ironically, the patches in \classname{ltxutil} were for the purpose of extending
% the \env{tabular} environment to prevent pagebreaks with the *-form of \cmd\\,
% just the same as is being done here. But the two mechanisms conflict. 
% \changes{4.1a}{2008/06/29}{Change \cs{LT@array@new}: restore \cs{@tabularcr} and \cs{@xtabularcr}}
%    \begin{macrocode}
  \let\LT@LL@FM@cr\@tabularcr@LaTeX
  \let\@xtabularcr\@xtabularcr@LaTeX
 }{%
  \advance\@tempdima\extrarowheight
  \col@sep\tabcolsep
  \let\@startpbox\LT@startpbox
%    \end{macrocode}
% \changes{4.1a}{2008/06/29}{Change \cs{LT@array@new}: set \cs{LT@LL@FM@cr} to \cs{@arraycr@array} instead of \cs{@arraycr}}
%    \begin{macrocode}
  \let\LT@LL@FM@cr\@arraycr@array
 }%
%
 \let\@acoll\@tabacoll
 \let\@acolr\@tabacolr
 \let\@acol\@tabacol
%
 \setbox\@arstrutbox\hbox{%
  \vrule
  \@height \arraystretch \@tempdima
  \@depth \arraystretch \dp \strutbox
  \@width \z@
 }%
 \let\@sharp##%
 \let\protect\relax
 \begingroup
  \@mkpream{#2}%
  \@mkpream@relax
  \edef\@preamble{\@preamble}%
  \prepdef\@preamble{%
   \global\advance\c@LT@chunks\@ne
   \global\LT@rows\z@
   \setbox\z@\vbox\bgroup
    \LT@setprevdepth
    \tabskip\LTleft
    \halign to\hsize\bgroup
     \tabskip\z@
     \@arstrut
  }%
  \appdef\@preamble{%
     \tabskip\LTright
     \cr
  }%
  \global\let\LT@bchunk\@preamble
 \endgroup
 \expandafter\LT@nofcols\LT@bchunk&\LT@nofcols
 \LT@make@row
 \m@th
 \let\par\@empty
 \everycr{}%
 \lineskip\z@
 \baselineskip\z@
 \LT@bchunk
}%
\appdef\table@hook{}%
%    \end{macrocode}
% \end{macro}
%
% \begin{macro}{\switch@longtable}
%
% Here is the switch from standard \classname{longtable} to the new, \classname{ltxgrid}-compatible values.
%
% At this point, we extend \env{longtable} with a \env{longtable*} form, which signifies that we want to 
% use the full page width for setting the table.
% You can think this way: \env{longtable*} is to \env{longtable} as \env{table*} is to \env{table}.
%
%    \begin{macrocode}
\def\switch@longtable{%
 \@ifpackageloaded{longtable}{%
  \@ifx{\longtable\longtable@longtable}{%
   \@ifx{\endlongtable\endlongtable@longtable}{%
    \@ifx{\LT@start\LT@start@longtable}{%
     \@ifx{\LT@end@hd@ft\LT@end@hd@ft@longtable}{%
      \@ifx{\LT@array\LT@array@longtable}{%
       \true@sw
      }{\false@sw}%
     }{\false@sw}%
    }{\false@sw}%
   }{\false@sw}%
  }{\false@sw}%
  {%
   \class@info{Patching longtable package}%
  }{%
   \class@info{Patching unrecognized longtable package. (Proceeding with fingers crossed)}%
  }%
  \let\longtable\longtable@new
  \let\endlongtable\endlongtable@new
  \let\LT@start\LT@start@new
  \let\LT@end@hd@ft\LT@end@hd@ft@new
  \let\LT@array\LT@array@new
  \newenvironment{longtable*}{%
   \onecolumngrid@push
   \longtable
  }{%
   \endlongtable
   \onecolumngrid@pop
  }%
%    \end{macrocode}
% Removed obsolete code.
%    \begin{macrocode}
 }{}%
}%
%    \end{macrocode}
% \end{macro}
%
% \begin{macro}{\LT@pre}
% \begin{macro}{\LT@bot}
% \begin{macro}{\LT@top}
% \begin{macro}{\LT@post}
% \begin{macro}{\LT@adj}
% Note that at the end of the longtable environment, we reestablish the \cmd\mark@envir\ of the
% containing environment. We have left \cmd\curr@envir\ alone, so this will work.
%    \begin{macrocode}
\def\LT@pre{\penalty\z@\vskip\LTpre}%
\def\LT@bot{\nobreak\copy\LT@foot\vfil}%
\def\LT@top{\copy\LT@head\nobreak}%
\def\LT@post{\penalty\z@\addvspace\LTpost\mark@envir{\curr@envir}}%
\def\LT@adj{%
 \setbox\z@\vbox{\null}\dimen@-\ht\z@
 \setbox\z@\vbox{\unvbox\z@\LT@bot}\advance\dimen@\ht\z@
 \global\advance\vsize-\dimen@
}%
%    \end{macrocode}
% \end{macro}
% \end{macro}
% \end{macro}
% \end{macro}
% \end{macro}
%
% \begin{macro}{output@init}
% \begin{macro}{output@prep}
% \begin{macro}{output@post}
%    \begin{macrocode}
\def\output@init@longtable{\LT@adj}%
\def\output@prep@longtable{\setbox\@cclv\vbox{\unvbox\@cclv\LT@bot}}%
\def\output@post@longtable{\LT@top}%
%    \end{macrocode}
% \end{macro}
% \end{macro}
% \end{macro}
%
%
% \subsection{Patches for index processing}%
%
% Another feature that uses the output routine hooks occurs within
% an index, where one wishes to apply a ``continue head'' when a
% column breaks within a primary index entry. 
% Some book designs call for the continue head to only be applied
% at a turnpage break.
%
% In any case, it is easy enough for \cmd\output@post@theindex\ 
% to do this in conjunction with component marks.
% Only the bare outlines are shown here.
%
% \begin{macro}{\output@init}
% \begin{macro}{\output@prep}
% \begin{macro}{\output@post}
%    \begin{macrocode}
\let\output@init@theindex\@empty
\let\output@prep@theindex\@empty
\def\output@post@theindex{%
 \@ifodd\c@page{}{%
  \@ifnum{\pagegrid@cur=\@ne}{%
%    \end{macrocode}
% We have the leftmost column of a verso page:
% Insert the current top-level continued head.
%    \begin{macrocode}
  }%
 }%
}%
%    \end{macrocode}
% \end{macro}
% \end{macro}
% \end{macro}
%
%
% \subsection{Checking the auxiliary file}%
%
% We relegate the checking of the auxiliary file to the output routine.
% This task must wait until the last page is shipped out, because otherwise
% the stream might get closed before the last page is shipped out.
% Obviously, we must use \cmd\do@output@MVL\ for the job.
%
% \begin{macro}{\check@aux}
%    \begin{macrocode}
\def\check@aux{\do@output@MVL{\do@check@aux}}%
%    \end{macrocode}
% \end{macro}
%
%
%
% \subsection{Dealing with stuck floats and stalled float dequeueing}%
%
% \LaTeX's float placement mechanism is fundamentally flawed, as evidenced by 
% its warning message ``too many unprocessed floats'', which users understandably find frustrating.
% The \classname{ltxgrid} package provides tools for ameliorating the situation somewhat.
%
% Two cases require detection and rectification:
% \begin{enumerate}
% \item
%  A float is ``stuck'' in the \cmd\@deferlist: for whatever reason, the float fails to be committed,
%  even at the start of a fresh page.
%  Once this condition prevails, following floats can never be committed, subsequently all of \LaTeX's 
%  float registers are used up.
%
%  If this condition is detected, we reconsider float dequeueing under permissive (\cmd\clearpage-style) processing.
%  
% \item
%  The \cmd\@freelist\ is exhausted:
%  a large concentration of floats, say, uses up all of \LaTeX's float registers all at once.
%  This condition commonly occurs when the user collects floats at the end of the document, for some reason.
%
%  When a float is encountered, \LaTeX\ uses a float register (allocated from a pool of free registers) to contain it until it can be placed.
%  However, no further action is taken until the pagebuilder is visited, so floats can accumulate.
%  Also, even after the pagebuilder is visited, deferred floats can accumulate, and these are not committed
%  until a column (or page) of text is completed.
%
%  Once the last free float register is used, action should be taken that will commit some of the deferred floats,
%  even if this might require ending the page right where we are (resulting in a short page).
%
%  Perhaps, committed floats should be stored using some mechanism other than a list, as is currently done.
%  A feasible alternative storage method would be to use a \cmd\box\ register in place of 
%  \cmd\@toplist,
%  \cmd\@botlist, and
%  \cmd\@dbltoplist.
%  This is probably just fine, since such committed floats are not reconsidered (I think).
%  
% \end{enumerate}
%
% The  emergency processing implemented here immediately ends the current page and begins to output float pages under (\cmd\clearpage-style) rules.
% It proceeds until all deferred floats have been flushed.
%
% Users should expect non-optimal page makeup under these circumstances.
%
% Note that there is a weakness in our approach that we have not attempted to repair: if floats are being
% added as part of a paragraph, we will not be able to take these remedial steps until the paragraph ends.
% This means that the approach implemented here cannot fix all \LaTeX\ documents. Users can still construct
% documents that exhaust \LaTeX's pool of float registers!
%
% \begin{macro}{\check@deferlist@stuck}
% \begin{macro}{\@outputpage@tail}
% \changes{4.1b}{2008/08/04}{Procedure \cs{@outputpage@tail} tailpatches \cs{@outputpage}}%
% We detect the case where, at the start of a fresh page, there are deferred floats, but none are
% committed. We memorize the \cmd\@deferlist\ at \cmd\shipout\ time, then examine it at the point where
% our efforts to commit floats to the new page are complete.
% If it has not changed, the first float must be stuck, and we
% attempt to fix things via \cmd\force@deferlist@stuck.
%
% This simple approach is comp[letely effective in for typical documents.
%
% Note that we try to avoid an infinite loop by examining the value of \cmd\clearpage@sw:
% if we come here with that boolean true, we are in a loop.
%    \begin{macrocode}
\def\check@deferlist@stuck#1{%
 \@ifx{\@deferlist@postshipout\@empty}{}{%
  \@ifx{\@deferlist@postshipout\@deferlist}{%
   \@fltstk
   \clearpage@sw{%
    \ltxgrid@warn{Deferred float stuck during \string\clearpage\space processing}%
   }{%
    \force@deferlist@stuck#1%
   }%
  }{%
%    \end{macrocode}
% We have successfully committed float(s)
%    \begin{macrocode}
  }%
  \global\let\@deferlist@postshipout\@empty
 }%
}%
\def\@fltstk{%
 \@latex@warning{A float is stuck (cannot be placed without \string\clearpage)}%
}%
\appdef\@outputpage@tail{%
 \global\let\@deferlist@postshipout\@deferlist
}%
%    \end{macrocode}
% \end{macro}
% \end{macro}
%
% \begin{macro}{\@next}
% \begin{macro}{\@xnext}
% We rewrite the \LaTeX\ kernel macros that dequeue float registers from, e.g., \cmd\@deferlist,
% providing a test for the condition where the pool of free registers is about to underflow.
%
% In this case, we attempt to fix things via \cmd\force@deferlist@empty.
%    \begin{macrocode}
\def\@next#1#2{%
 \@ifx{#2\@empty}{\false@sw}{%
  \expandafter\@xnext#2\@@#1#2%
  \true@sw
 }%
}%
\def\@xnext\@elt#1#2\@@#3#4{%
 \def#3{#1}%
 \gdef#4{#2}%
 \def\@tempa{#4}\def\@tempb{\@freelist}%
 \@ifx{\@tempa\@tempb}{%
  \@ifx{#4\@empty}{%
   \force@deferlist@empty%{Float register pool exhausted}%
  }{}%
 }{}%
}%
%    \end{macrocode}
% \end{macro}
% \end{macro}
%
% \begin{macro}{\force@deferlist@stuck}
% \begin{macro}{\force@deferlist@empty}
% \begin{macro}{\force@deferlist@sw}
% \begin{macro}{\do@forcecolumn@pen}
% \begin{macro}{\do@forcecolumn}
% The procedure \cmd\force@deferlist@empty\ is an attempt to rectify a situation where \LaTeX's float placement mechanism
% may fail (``too many unprocessed floats'').
%
% We put down interrupts that call for the float placement to be redone, but under permissive conditions,
% just the same as if \cmd\clearpage\ had been invoked.
%
% Note that the attempt to rectify the error is contingent on the setting of \cmd\force@deferlist@sw,
% default false. A document class using this package that wishes to enable this error recovery mechanism should
% set this boolean to true.
%
% The interrupt \cmd\do@forcecolumn@pen, which invokes the procedure \cmd\do@forcecolumn,
% does the same as \cmd\do@startcolumn, except under permissive conditions:
% we are trying to empty out the float registers completely.
%
% In order to properly with the case where there is material in \cmd\box\cmd\@cclv, 
% \cmd\@toplist, \cmd\@botlist, \cmd\@dbltoplist, etc,
% we do what amounts to \cmd\newpage\ to get things rolling.
%
% In \cmd\force@deferlist@stuck, we take advantage of already being in the output routine:
% simply reinvoke \cmd\do@startcolumn\ under permissive conditions.
%
%    \begin{macrocode}
\def\force@deferlist@stuck#1{%
 \force@deferlist@sw{%
  \@booleantrue\clearpage@sw
  \@booleantrue\forcefloats@sw
  #1%
 }{%
 }%
}%
\def\force@deferlist@empty{%
 \force@deferlist@sw{%
  \penalty-\pagebreak@pen
  \protect@penalty\do@forcecolumn@pen
 }{%
 }%
}%
\@booleanfalse\force@deferlist@sw
\mathchardef\do@forcecolumn@pen=10009
\@namedef{output@-\the\do@forcecolumn@pen}{\do@forcecolumn}%
\def\do@forcecolumn{%
 \@booleantrue\clearpage@sw
 \@booleantrue\forcefloats@sw
%    \end{macrocode}
%    \begin{verbatim}
%\unvbox\@cclv
%\vfil
%\penalty-\pagebreak@pen
%    \end{verbatim}
%    \begin{macrocode}
 \do@startcolumn
}%
%    \end{macrocode}
% \end{macro}
% \end{macro}
% \end{macro}
% \end{macro}
% \end{macro}
%
% A more thorough revision of \LaTeX's float placement mechanism would involve substituting a single \cmd\box\
% register for the \cmd\@deferlist. This way, \LaTeX's ability to have latent floats would be limited by
% box memory alone. 
%
% Because only the \cmd\box\ and \cmd\count\ components of the float box register are actually used by \LaTeX,
% our scheme can be accomplished if we can find a way to encode the information held in the \cmd\count\ component.
%
% A first-in, first-out mechanism exists, wherein a box-penalty pair is dequeued by \cmd\lastbox\cmd\lastpenalty\cmd\unpenalty\ and enqueued
% by \cmd\setbox\cmd\foo=\cmd\hbox\cmd\bgroup\cmd\penalty\cmd\floatpenalty\cmd\box\cmd\floatbox\cmd\unhbox\cmd\foo\cmd\egroup.
%
% Note that this scheme is made possible by our change to \LaTeX's float placement mechanism,
% wherein we consolidated the two \cmd\@deferlist s into one.
%
% \section{Support for legacy \LaTeX\ commands}
%
% We provide support for the \cmd\enlargethispage\ command.
%
% Note: using a command of this sort does not automatically enlarge both pages of a spread, which would be the convention in page composition.
%
% Timing Note: In a multicolumn page grid, the user should issue the \cmd\enlargethispage\ command 
% while the first column of the page is being typeset. 
% We provide a helpful message if the timing is wrong.
%
% This code can serve as a model for introducing commands that need to execute within the safety of the output routine.
% We ensure that the arguments are fully expanded, then execute \cmd\do@output@MVL\ to cause an output procedure,
% \cmd\@@enlargethispage, to execute. When it does execute, the MVL will be exposed.
%
% The \cmd\@@enlargethispage\ procedure simply adjusts the vertical dimensions of the page.
% The adjustment will persist until the column is committed, at which point the page dimension
% will revert to its standard value.
%    \begin{macrocode}
\def\enlargethispage{%
 \@ifstar{%
  \@enlargethispage{}%
 }{%
  \@enlargethispage{}%
 }%
}%
\def\@enlargethispage#1#2{%
 \begingroup
  \dimen@#2\relax
  \edef\@tempa{#1}%
  \edef\@tempa{\noexpand\@@enlargethispage{\@tempa}{\the\dimen@}}%
  \expandafter\do@output@MVL\expandafter{\@tempa}%
 \endgroup
}%
\def\@@enlargethispage#1#2{%
 \def\@tempa{one}%
 \@ifx{\thepagegrid\@tempa}{%
  \true@sw
 }{%
  \def\@tempa{mlt}%
  \@ifx{\thepagegrid\@tempa}{%
   \@ifnum{\pagegrid@cur=\@ne}{% 
%    \end{macrocode}
%   OK to adjust this page
%    \begin{macrocode}
    \gdef\enlarge@colroom{#2}%
    \true@sw
   }{% 
%    \end{macrocode}
%   Can only adjust this column; give up
%    \begin{macrocode}
    \ltxgrid@warn{Too late to enlarge this page; move the command to the first column.}%
    \false@sw
   }%
  }{% 
%    \end{macrocode}
%  Unknown page grid
%    \begin{macrocode}
   \ltxgrid@warn{Unable to enlarge a page of this kind.}%
   \false@sw
  }%
 }%
 {%
  \class@info{Enlarging page \thepage\space by #2}%
  \global\advance\@colroom#2\relax
  \set@vsize
 }{%
%    \end{macrocode}
% Could not adjust this page
%    \begin{macrocode}
 }%
}%
\let\enlarge@colroom\@empty
%    \end{macrocode}
% The \cmd\@kludgeins\ insert register is now unneeded.
% Ensure that packages using this mechanism break (preferrable to subtle bugs).
%    \begin{macrocode}
\let\@kludgeins\@undefined
%    \end{macrocode}
%
% \subsubsection{Building the page for shipout}
%
% \begin{macro}{\@outputpage@head}
% We set \cmd\@outputpage@head\ to make the \cmd\@outputbox\ be of fixed height.
% \changes{4.1b}{2008/08/04}{Procedure \cs{@outputpage@head} headpatches \cs{@outputpage}}%
%    \begin{macrocode}
\@booleantrue\textheight@sw
\prepdef\@outputpage@head{%
 \textheight@sw{%
  \count@\vbadness\vbadness\@M
  \dimen@\vfuzz\vfuzz\maxdimen
  \setbox\@outputbox\vbox to\textheight{\unvbox\@outputbox}%
  \vfuzz\dimen@
  \vbadness\count@
 }{}%
}%
%    \end{macrocode}
% \end{macro}
%
% \begin{macro}{\@outputpage@head}
% For compatibility with David Carlisle's \classname{lscape} package, we need to allow the 
% \cmd\LS@rot\ procedure to mung \cmd\@outputbox. 
%
% Implementation note: the \classname{lscape} package effectively tailpatches two \LaTeX\ internals to accomplish its purpose,
% an approach that is not robust. It is more robust to headpatch \cmd\@outputpage, which is what we do here. 
% \changes{4.1o}{2010/02/02}{(AO, 576) Allow \classname{lscape} to act on \cs{@outputbox} at the right time}%
%    \begin{macrocode}
\appdef\@outputpage@head{%
 \@ifx{\LS@rot\@undefined}{}{\LS@rot}%
}%
%    \end{macrocode}
% \end{macro}
%
% \subsubsection{Warning message}
%
% \begin{macro}{\ltxgrid@info}
% \begin{macro}{\ltxgrid@warn}
% Something has happened that the user might be interested in.
% Print a message to the log, but only if the user selected the verbose option.
%    \begin{macrocode}
\def\ltxgrid@info{%
 \ltxgrid@info@sw{\class@info}{\@gobble}%
}%
\@booleanfalse\ltxgrid@info@sw
\def\ltxgrid@warn{%
 \ltxgrid@warn@sw{\class@warn}{\@gobble}%
}%
\@booleantrue\ltxgrid@warn@sw
\@booleanfalse\ltxgrid@foot@info@sw
%    \end{macrocode}
% \end{macro}
% \end{macro}
%
% \section{Line-wise processing}%
% Sometimes we wish to process each line of type that will be placed into the galley,
% for example, applying line numbering to a document.
% To accomplish the task, we have to force a visit to the output routine after each such line, whereupon
% we can process it accordingly (in the case of line numbering, we could do as \filename{lineno.sty} and
% append an appropriately formed box to the MVL).
%
% In implementing such a scheme, we will have to instantiate interrupts for the following cases:
% \begin{description}
%
% \item[\cmd\interlinepenalty\ and friends] These include \cmd\clubpenalty, \cmd\widowpenalty, \cmd\displaywidowpenalty, and \cmd\brokenpenalty.
%
%
% \item[Display math penalties] Includes \cmd\predisplaypenalty, \cmd\postdisplaypenalty, and \cmd\interdisplaylinepenalty.
%
%
% \item[\cs{par}] The penalty following the last line of the paragraph.
%
%
% \item[\cmd\vadjust] A trap for any \cmd\vadjust\ command that falls in the paragraph.
%
% \end{description}
%
% \begin{macro}{\def@next@handler}
% \begin{macro}{\def@line@handler}
% Utility procedures \cmd\def@next@handler\ and \cmd\def@line@handler\ help in the creation of 
% interrupt handlers. 
%
% \cmd\def@next@handler\ increments the scratch count register (argument 1), 
% using this value to \cmd\mathchardef\ its second argument
% as the negative of the flag value to be used as a penalty for exciting the interrupt (argument 3).
% As a byproduct, it leaves the given scratch counter incremented. 
%    \begin{macrocode}
\def\def@next@handler#1#2#3{%
 \advance#1\@ne\mathchardef#2\the#1%
 \expandafter\def\csname output@-\the#1\endcsname{#3}%
%    \end{macrocode}
% The following line is for diagnostic purposes.
%    \begin{verbatim}
% \typeout{\string#2(\expandafter\string\csname output@\the#1\endcsname:\expandafter\meaning\csname output@\the#1\endcsname)}%
%    \end{verbatim}
%    \begin{macrocode}
}%
%    \end{macrocode}
% \cmd\def@line@handler\ uses \cmd\int@parpenalty\ as a base.
% The interrupt is the sum of that base with the first argument, 
% and the handler is the second argument.
%    \begin{macrocode}
\def\def@line@handler#1#2{%
 \begingroup
  \@tempcnta\int@parpenalty
  \advance\@tempcnta-#1%
%    \end{macrocode}
% The following line is for diagnostic purposes.
%    \begin{verbatim}
%  \typeout{Defining: \expandafter\string\csname output@\the\linenopenalty\endcsname}%
%    \end{verbatim}
%    \begin{macrocode}
  \aftergroup\def
  \expandafter\aftergroup\csname output@-\the\@tempcnta\endcsname
 \endgroup{#2}%
}%
%    \end{macrocode}
% \end{macro}
% \end{macro}
%
% \begin{macro}{\int@parpenalty}
% \begin{macro}{\@handle@line@ltx}
% \begin{macro}{\@@handle@line@ltx}
% We first set \cmd\int@parpenalty\ to our chosen base value${}\le -11012$. 
% We then define all the handlers for lines within a paragraph, of which there are 12 different cases.
%    \begin{macrocode}
\mathchardef\int@parpenalty11012
\def@line@handler\z@{\@handle@line@ltx{}{}{}}%
\def@line@handler\@ne{\@handle@line@ltx{}{}{\brokenpenalty@ltx}}%
\def@line@handler\tw@{\@handle@line@ltx{}{\clubpenalty@ltx}{}}%
\def@line@handler\thr@@{\@handle@line@ltx{\clubpenalty@ltx}{}{\brokenpenalty@ltx}}%
\def@line@handler\f@ur{\@handle@line@ltx{\widowpenalty@ltx}{}{}}%
\def@line@handler{5}{\@handle@line@ltx{\widowpenalty@ltx}{}{\brokenpenalty@ltx}}%
\def@line@handler{6}{\@handle@line@ltx{\widowpenalty@ltx}{\clubpenalty@ltx}{}}%
\def@line@handler{7}{\@handle@line@ltx{\widowpenalty@ltx}{\clubpenalty@ltx}{\brokenpenalty@ltx}}%
\def@line@handler{8}{\@handle@line@ltx{\displaywidowpenalty@ltx}{}{}}%
\def@line@handler{9}{\@handle@line@ltx{\displaywidowpenalty@ltx}{}{\brokenpenalty@ltx}}%
\def@line@handler{10}{\@handle@line@ltx{\displaywidowpenalty@ltx}{\clubpenalty@ltx}{}}%
\def@line@handler{11}{\@handle@line@ltx{\displaywidowpenalty@ltx}{\clubpenalty@ltx}{\brokenpenalty@ltx}}%
%    \end{macrocode}
%
% The default handler for lines within a paragraph simply 
% restores the value of the \cmd\penalty\ to the normal value.
% If something more useful needs to be done, we can change the definition of \cmd\@@handle@line@ltx.
%    \begin{macrocode}
\def\@handle@line@ltx#1#2#3{%
 \@@handle@line@ltx
 \@tempcnta\lastpenalty
 \@tempcntb\interlinepenalty@ltx\relax
 \@if@empty{#1}{}{\advance\@tempcntb#1\relax}%
 \@if@empty{#2}{}{\advance\@tempcntb#2\relax}%
 \@if@empty{#3}{}{\advance\@tempcntb#3\relax}%
 \penalty\@ifnum{\@tempcnta<\@tempcntb}{\@tempcntb}{\@tempcnta}%
}%
\let\@@handle@line@ltx\@empty
%    \end{macrocode}
% \end{macro}
% \end{macro}
% \end{macro}
%
% \begin{macro}{\int@postparpenalty}
% \begin{macro}{\int@vadjustpenalty}
% \begin{macro}{\int@whatsitpenalty}
% \begin{macro}{\int@predisplaypenalty}
% \begin{macro}{\int@interdisplaylinepenalty}
% \begin{macro}{\int@postdisplaypenalty}
% \begin{macro}{\@handle@display@ltx}
% \begin{macro}{\@@handle@display@ltx}
% \begin{macro}{\handle@par@ltx}
% We herewith define all the handlers for cases relating to display math:
% last line before a display math,
% last line of a display math, and a line within a display math.
% We also handle the last line of a paragraph, a whatsit node, and a \cmd\vadjust.
%    \begin{macrocode}
\@tempcnta\int@parpenalty
\def@next@handler\@tempcnta\int@postparpenalty{\reset@queues@ltx\handle@par@ltx}%
\def@next@handler\@tempcnta\int@vadjustpenalty{\handle@vadjust@ltx}%
\def@next@handler\@tempcnta\int@whatsitpenalty{\handle@whatsit@ltx}%
\def@next@handler\@tempcnta\int@predisplaypenalty{\reset@queues@ltx\@handle@display@ltx{\predisplaypenalty@ltx}}%
\def@next@handler\@tempcnta\int@interdisplaylinepenalty{\@handle@display@ltx{\interdisplaylinepenalty@ltx}}%
\def@next@handler\@tempcnta\int@postdisplaypenalty{\@handle@display@ltx{\postdisplaypenalty@ltx}}%
%    \end{macrocode}
% The default handler for display math lines simply 
% restores the value of the \cmd\penalty\ to the normal value.
% If something more useful needs to be done, we can change the definition of \cmd\@@handle@display@ltx. 
%    \begin{macrocode}
\def\@handle@display@ltx#1{%
 \@@handle@display@ltx
 \@tempcnta\lastpenalty
 \@tempcntb#1%
 \penalty\@ifnum{\@tempcnta<\@tempcntb}{\@tempcntb}{\@tempcnta}%
}%
\let\@@handle@display@ltx\@empty
%    \end{macrocode}
%
% We provide stub definitions for the handlers for the last line of a paragraph, a \cmd\vadjust, and a whatsit node (e.g., \cmd\write, \cmd\special). 
% There is no canonical penalty for such cases.
%    \begin{macrocode}
\def\handle@par@ltx{}%
%    \end{macrocode}
% \end{macro}
% \end{macro}
% \end{macro}
% \end{macro}
% \end{macro}
% \end{macro}
% \end{macro}
% \end{macro}
% \end{macro}
%
% Note that a whatsit needs to be handled differently from a \cmd\vadjust:
% a whatsit node does not affect the (crucial) depth of \cmd\box\cmd\@cclv, while the more
% general \cmd\vadjust\ may cause any kind of vertical mode material to be interposed just below the line
% we are trying to trap, in particular \cmd\vskip s and \cmd\penalty s.
%
% \begin{macro}{\set@linepenalties}
% \begin{macro}{\restore@linepenalties}
% \begin{macro}{\set@displaypenalties}
% Now we define utility procedures that set up for a paragraph to be broken into lines, 
% restoring the penalties afterwards.
% 
% Utility procedure \cmd\set@linepenalties\ systematically sets the penalties of paragraph breaking
% to flag values, meanwhile storing away the normal values for access by the output routine. 
%    \begin{macrocode}
\def\set@linepenalties{%
 \expandafter\def\expandafter\interlinepenalty@ltx\expandafter{\the\interlinepenalty}%
 \interlinepenalty-\int@parpenalty
 \expandafter\def\expandafter\brokenpenalty@ltx\expandafter{\the\brokenpenalty}%
 \brokenpenalty\@ne
 \expandafter\def\expandafter\clubpenalty@ltx\expandafter{\the\clubpenalty}%
 \clubpenalty\tw@
 \expandafter\def\expandafter\widowpenalty@ltx\expandafter{\the\widowpenalty}%
 \widowpenalty\f@ur
 \expandafter\def\expandafter\displaywidowpenalty@ltx\expandafter{\the\displaywidowpenalty}%
 \displaywidowpenalty8\relax
}%
%    \end{macrocode}
%
% Utility procedure \cmd\restore@linepenalties\ restores the values of the penalty parameters that were
% modified by \cmd\set@linepenalties.
%    \begin{macrocode}
\def\restore@linepenalties{%
 \interlinepenalty\interlinepenalty@ltx
 \brokenpenalty\brokenpenalty@ltx
 \clubpenalty\clubpenalty@ltx
 \widowpenalty\widowpenalty@ltx
 \displaywidowpenalty\displaywidowpenalty@ltx
 \relax
}%
%    \end{macrocode}
%
% In the following, the first argument should be a boolean (either \cmd\true@sw\ or \cmd\false@sw).
%    \begin{macrocode}
\def\set@displaypenalties#1{%
 \expandafter\def\expandafter\predisplaypenalty@ltx\expandafter{\the\predisplaypenalty}%
 \expandafter\def\expandafter\interdisplaylinepenalty@ltx\expandafter{\the\interdisplaylinepenalty}%
 \expandafter\def\expandafter\postdisplaypenalty@ltx\expandafter{\the\postdisplaypenalty}%
 \@ifhmode{\predisplaypenalty-\int@predisplaypenalty\relax}{}%
 #1{\interdisplaylinepenalty-\int@interdisplaylinepenalty\relax}{}%
 #1{\postdisplaypenalty-\int@postdisplaypenalty\relax}{}%
}%
%    \end{macrocode}
% We provide no procedure to restore the respective penalties, because they are altered within a group:
% \TeX 's context stack will automatically restore things.
% \end{macro}
% \end{macro}
% \end{macro}
%
% \begin{macro}{\enqueue@whatsit@ltx}%
% \begin{macro}{\handle@whatsit@ltx}%
% \begin{macro}{\do@whatsit}%
% \begin{macro}{\@g@pop@ltx}%
% Here is a facility for dealing with whatsit nodes while we are trapping paragraph lines. 
% We simply enqueue a macro that will create the desired whatsit node, dequeueing it in the output routine.
%    \begin{macrocode}
\def\enqueue@whatsit@ltx#1{%
 \gappdef\g@whatsit@queue{{#1}}%
 \vadjust{\penalty-\int@whatsitpenalty}%
}%
\def\handle@whatsit@ltx{%
 \unvbox\@cclv
 \g@pop@ltx\g@whatsit@queue\@tempa
 \expandafter\do@whatsit\expandafter{\@tempa}%
}%
\def\do@whatsit#1{}%
\def\g@pop@ltx#1#2{%
 \expandafter\@g@pop@ltx#1{}{}\@@#1#2%
}%
\def\@g@pop@ltx#1#2\@@#3#4{%
 \gdef#3{#2}%
 \def#4{#1}%
}%
%    \end{macrocode}
% \end{macro}
% \end{macro}
% \end{macro}
% \end{macro}
%
% \begin{macro}{\vspace}%
% \begin{macro}{\pagebreak}%
% \begin{macro}{\nopagebreak}%
% \begin{macro}{\\}%
% \begin{macro}{\@arrayparboxrestore}%
% We wish to prevent \filename{lineno.sty} from patching \cmd\vspace\ and \cmd\pagebreak, 
% because that package does it through global assignments, which is prone to failure. 
%
% We also wish to prevent that package from patching \cmd\@arrayparboxrestore, because
% it prevents us from \cmd\unvbox ing vertical mode material into the MVL and numbering those
% lines.
% 
% We start by retaining the original definitions of these commands, so we can restore them
% if \filename{lineno.sty} does get loaded. 
%    \begin{macrocode}
\let\vspace@ltx\vspace
\let\pagebreak@ltx\pagebreak
\let\nopagebreak@ltx\nopagebreak
\let\endline@ltx\\
\let\@arrayparboxrestore@ltx\@arrayparboxrestore
%    \end{macrocode}
%
% Next, we provide for line-wise processing by patching the procedures associated with these same three commands.
%
% There are exactly four core \LaTeX\ procedures that use \cmd\vadjust\ to insert vertical mode material 
% into the main vertical list: \cmd\vspace, \cmd\pagebreak, \cmd\nopagebreak, and \cmd\\. 
% Other commands may use \cmd\vadjust, but they are inserting an interrupt (via a penalty${}<10000$), 
% and such a thing does not mask the depth of \cmd\box\cmd\@cclv, hence is permissible.
%
% In each case, we replace the core \LaTeX\ procedure with one that itself replaces \cmd\vadjust\ with
% \cmd\ex@vadjust@ltx. The meaning of this procedure can be left as \cmd\vadjust, or it can be changed to
% one that accomplishes the equivalent without masking the depth of \cmd\box\cmd\@cclv.
% 
% The first procedure is \cmd\@vspace, here shown in original form and in the patched alternative form. 
% This procedure and \cmd\@vspacer\ implement the \cmd\vspace\ command.
%    \begin{macrocode}
\def\@vspace@org #1{%
  \ifvmode
    \vskip #1
    \vskip\z@skip
   \else
     \@bsphack
     \vadjust{\@restorepar
              \vskip #1
              \vskip\z@skip
              }%
     \@esphack
   \fi
}%
\def\@vspace@ltx#1{%
 \@ifvmode{%
  \vskip#1\vskip\z@skip
 }{%
  \@bsphack
  \ex@vadjust@ltx{%
   \@restorepar
   \nobreak
   \vskip#1\vskip\z@skip
  }%
  \@esphack
 }%
}%
%    \end{macrocode}
% 
% The second procedure is \cmd\@vspacer.
%    \begin{macrocode}
\def\@vspacer@org#1{%
  \ifvmode
    \dimen@\prevdepth
    \hrule \@height\z@
    \nobreak
    \vskip #1
    \vskip\z@skip
    \prevdepth\dimen@
  \else
    \@bsphack
    \vadjust{\@restorepar
             \hrule \@height\z@
             \nobreak
             \vskip #1
             \vskip\z@skip}%
    \@esphack
\fi
}%
\def\@vspacer@ltx#1{%
 \@ifvmode{%
  \dimen@\prevdepth
  \hrule\@height\z@
  \nobreak
  \vskip#1\vskip\z@skip
  \prevdepth\dimen@
 }{%
  \@bsphack
  \ex@vadjust@ltx{%
   \@restorepar
   \hrule\@height\z@
   \nobreak
   \vskip#1\vskip\z@skip
  }%
  \@esphack
 }%
}%
%    \end{macrocode}
%
% The procedure \cmd\@no@pgbk\ implements both \cmd\pagebreak\ and \cmd\nopagebreak.
%    \begin{macrocode}
\def\@no@pgbk@org #1[#2]{%
  \ifvmode
    \penalty #1\@getpen{#2}%
  \else
    \@bsphack
    \vadjust{\penalty #1\@getpen{#2}}%
    \@esphack
  \fi
}%
\def\@no@pgbk@ltx#1[#2]{%
 \@ifvmode{%
  \penalty#1\@getpen{#2}%
 }{%
  \@bsphack
  \ex@vadjust@ltx{%
   \penalty#1\@getpen{#2}%
  }%
  \@esphack
 }%
}%
%    \end{macrocode}
%
% The command to end a line of type, \cmd\\, is defined via \cmd\DeclareRobustCommand,
% so we must proceed carefully: 
% A procedure is defined whose \cmd\long\cmd\csname\ is contructed via the incantation:
% \cmd\csname\cmd\expandafter\cmd\@gobble\cmd\string\cmd\\ \cmd\endcsname.
% Note the non-trivial space character after the \cmd\\: it is incorporated into 
% the \cmd\csname.
%
% Here is the original core \LaTeX\ definition for 
% the procedure involved, along with our revised version. 
%    \begin{macrocode}
\long\def\end@line@org{%
 \let\reserved@e\relax
 \let\reserved@f\relax
 \@ifstar{%
  \let\reserved@e\vadjust
  \let\reserved@f\nobreak
  \@xnewline
 }%
 \@xnewline
}%
\long\def\end@line@ltx{%
 \let\reserved@e\relax
 \let\reserved@f\relax
 \@ifstar{%
  \let\reserved@e\ex@vadjust@ltx
  \let\reserved@f\nobreak
  \@xnewline
 }{%
  \@xnewline
 }%
}%
%    \end{macrocode}
% An additional procedure requiring patching has the following original core \LaTeX\
% definition; we modify it correspondingly.
%    \begin{macrocode}
\def\@newline@org[#1]{%
 \let\reserved@e\vadjust
 \@gnewline{\vskip#1}%
}%
\def\@newline@ltx[#1]{%
 \let\reserved@e\ex@vadjust@ltx
 \@gnewline{\vskip#1}%
}%
%    \end{macrocode}
% We now install our patches. 
% If some package overrides these macros, we will detect and complain.
% 
%    \begin{macrocode}
 \@ifx{\@vspace\@vspace@org}{%
  \@ifx{\@vspacer\@vspacer@org}{%
   \@ifx{\@no@pgbk\@no@pgbk@org}{%
    \@ifx{\@newline\@newline@org}{%
     \expandafter\@ifx\expandafter{\csname\expandafter\@gobble\string\\ \endcsname\end@line@org}{%
      \true@sw
     }{\false@sw}%
    }{\false@sw}%
   }{\false@sw}%
  }{\false@sw}%
 }{\false@sw}%
 {%
  \class@info{Overriding \string\@vspace, \string\@vspacer, \string\@no@pgbk, \string\@newline, and \string\\ }%
  \let\@normalcr\end@line@ltx
  \expandafter\let\csname\expandafter\@gobble\string\\ \endcsname\@normalcr
  \let\@newline\@newline@ltx
  \let\@vspace\@vspace@ltx
  \let\@vspacer\@vspacer@ltx
  \let\@no@pgbk\@no@pgbk@ltx
 }{%
  \class@warn{%
   Failed to recognize \string\@vspace, \string\@vspacer, \string\@no@pgbk, \string\@newline, and \string\\; 
   no patches applied. Please get a more up-to-date class,
  }%
 }%
%    \end{macrocode}
% Note that we have assigned the same meaning to \cmd\@normalcr, which is necessary to \LaTeX.
% \end{macro}
% \end{macro}
% \end{macro}
% \end{macro}
% \end{macro}
% 
% \begin{macro}{\ex@vadjust@ltx}
% \begin{macro}{\enqueue@vadjust@ltx}
% \begin{macro}{\handle@vadjust@ltx}
% \begin{macro}{\g@vadjust@line}
% \begin{macro}{\reset@queues@ltx}
% Here we give the default definition for \cmd\ex@vadjust@ltx\ along with
% the definitions for the alternative version and its the associated handler. 
%    \begin{macrocode}
\let\ex@vadjust@ltx\vadjust
\def\enqueue@vadjust@ltx#1{%
 \gappdef\g@vadjust@queue{{#1}}%
 \vadjust{\penalty-\int@vadjustpenalty}%
}%
\def\handle@vadjust@ltx{%
 \unvbox\@cclv
 \g@pop@ltx\g@vadjust@queue\@tempa
 \expandafter\gappdef\expandafter\g@vadjust@line\expandafter{\@tempa}%
}%
\let\g@vadjust@line\@empty
%    \end{macrocode}
% Procedure \cmd\reset@queues@ltx\ resets the whatsit queue and the \cmd\vadjust\ queues 
% to their empty state. This should be done whenever we leave horizontal mode and
% complete the processing of these queues: upon executing, effectively, primitive \cmd\par\
% or interrupting a paragraph with display math. 
%    \begin{macrocode}
\def\reset@queues@ltx{%
 \global\let\g@whatsit@queue\@empty
 \global\let\g@vadjust@queue\@empty
}%
%    \end{macrocode}
% \end{macro}
% \end{macro}
% \end{macro}
% \end{macro}
% \end{macro}
%
% \section{Patching the lineno.sty package}
%
% \filename{lineno.sty} is a \LaTeX\ package that applies line numbering to a document.
% The basic method is to give \cmd\interlinepenalty\ and like penalties such a value as 
% to force a visit to the output routine, where the line of type is given its number. 
% In order to properly measure the depth of \cmd\box\cmd\@cclv, it defers \cmd\vadjust\ 
% commands that may insert \cmd\vskip\ or \cmd\penalty\ nodes.
%
% The implementation of that package, however, manipulates \cmd\holdinginserts\ in a dangerous
% way: outside the safety of the output routine. It also alters the meaning of \cmd\vadjust\
% using global assignments. We patch its code to avoid these problems. The \filename{ltxgrid}
% package already has the needed mechanisms in place to do these jobs correctly.
%
% The methods we use can accomodate any values of penalties like \cmd\clubpenalty, etc:
% we do not make assumptions about the range of values these penalty parameters could take.
%
%
% \begin{macro}{\linenomathWithnumbers}
% \begin{macro}{\linenomathNonumbers}
% \begin{macro}{\endlinenomath}
% \begin{macro}{\linenumberpar}
% Here are the definitions of procedures in \filename{lineno.sty}
% that alter \cmd\holdinginserts. They are current as of version v4.41, 2005/11/02.
% We patch them to avoid doing this: in ltxgrid-based classes like REVTeX, the output routine properly manages \cmd\holdinginserts, so packages should not attempt to do so.
% Also, we will want \cmd\linenumberpar\ to set \cmd\interlinepenaly\ to dispatch to \cmd\MakeLineNo. 
%
%    \begin{macrocode}
\newcommand\linenomathWithnumbers@LN{%
  \ifLineNumbers 
    \ifnum\interlinepenalty>-\linenopenaltypar
      \global\holdinginserts\thr@@ 
      \advance\interlinepenalty \linenopenalty
     \ifhmode
      \advance\predisplaypenalty \linenopenalty
     \fi 
      \advance\postdisplaypenalty \linenopenalty
      \advance\interdisplaylinepenalty \linenopenalty
    \fi
  \fi
  \ignorespaces
}%
\newcommand\linenomathNonumbers@LN{%
  \ifLineNumbers 
    \ifnum\interlinepenalty>-\linenopenaltypar
      \global\holdinginserts\thr@@ 
      \advance\interlinepenalty \linenopenalty
     \ifhmode
      \advance\predisplaypenalty \linenopenalty
     \fi 
    \fi
  \fi
  \ignorespaces
}%
\def\endlinenomath@LN{% 
  \ifLineNumbers
   \global\holdinginserts\@LN@outer@holdins
  \fi 
  \global\@ignoretrue
}
\def\linenumberpar@LN{%
  \ifvmode \@@@par \else 
    \ifinner \@@@par \else
      \xdef\@LN@outer@holdins{\the\holdinginserts}%
      \advance \interlinepenalty \linenopenalty
      \linenoprevgraf \prevgraf
      \global \holdinginserts \thr@@ 
      \@@@par
      \ifnum\prevgraf>\linenoprevgraf
        \penalty-\linenopenaltypar
      \fi
      \@LN@parpgbrk 
      \global\holdinginserts\@LN@outer@holdins
      \advance\interlinepenalty -\linenopenalty
    \fi
  \fi
}%
%    \end{macrocode}
% \end{macro}
% \end{macro}
% \end{macro}
% \end{macro}
%
% \begin{macro}{\class@documenthook}
% We patch only if we recognize the definitions of all the procedures we are to patch. 
%    \begin{macrocode}
\appdef\class@documenthook{%
 \@ifpackageloaded{lineno}{%
  \@ifx{\linenomathWithnumbers\linenomathWithnumbers@LN}{%
   \@ifx{\linenomathNonumbers\linenomathNonumbers@LN}{%
    \@ifx{\endlinenomath\endlinenomath@LN}{%
     \@ifx{\linenumberpar\linenumberpar@LN}{%
      \true@sw
     }{\false@sw}%
    }{\false@sw}%
   }{\false@sw}%
  }{\false@sw}%
  {%
   \class@info{Overriding lineo.sty, restoring output routine,}%
%    \end{macrocode}
% We commence overriding the procedures of \filename{lineno.sty}.
%    \begin{macrocode}
   \let\linenumberpar\linenumberpar@ltx
   \let\endlinenomath\endlinenomath@ltx
   \expandafter\let\csname endlinenomath*\endcsname\endlinenomath@ltx
   \let\linenomathWithnumbers\linenomathWithnumbers@ltx
   \let\linenomathNonumbers\linenomathNonumbers@ltx
%    \end{macrocode}
% Override \filename{lineno.sty}'s equipment for \cmd\vadjust\ and \cmd\linelabel:
% we have existing interrupts and handlers for these purposes. 
%    \begin{macrocode}
   \let\ex@vadjust@ltx\ex@vadjust@line
   \let\@LN@postlabel\enqueue@whatsit@ltx
   \let\do@whatsit\write@linelabel
%    \end{macrocode}
% Redirect handlers to those provided by \filename{lineno.sty}, and
% give an appropriate meaning to the respective headpatch within the handlers.
%    \begin{macrocode}
   \let\handle@par@ltx\handle@par@LN
   \let\@@handle@line@ltx\Make@LineNo@ltx
   \let\@@handle@display@ltx\Make@LineNo@ltx
%    \end{macrocode}
% Next, we undo the action taken by \filename{lineno.sty} wherein it
% took over the output routine. Instead, we service \filename{lineno.sty}
% existing equipment of \filename{ltxgrid}.
% We also revert the core \LaTeX\ definitions of 
% \cmd\vspace, \cmd\pagebreak, \cmd\nopagebreak, and \cmd\\,
% which that package takes over (we have our own ways of doing these things). 
%    \begin{macrocode}
   \output@latex{\natural@output}%
   \let\vspace\vspace@ltx
   \let\pagebreak\pagebreak@ltx
   \let\nopagebreak\nopagebreak@ltx
   \let\@arrayparboxrestore\@arrayparboxrestore@ltx
   \let\\\endline@ltx
%    \end{macrocode}
% \changes{4.1f}{2009/07/07}{(AO, 515) Prevent line numbering within a footnote}
% When line numbering  is in effect, we must avoid any attempt to number the lines of a footnote.
%    \begin{macrocode}
   \appdef\set@footnotefont{%
    \let\par\@@@par
    \let\@@par\@@@par
   }%
%    \end{macrocode}
% At last, we detect if the \cmd\linenumbers\ command has already been given; 
% if so, we do its assignments again, because we have changed the meaning of \cmd\linenumberpar. 
%    \begin{macrocode}
   \@if@sw\ifLineNumbers\fi{%
    \class@info{Reinvoke \string\linenumbers}%
    \let\@@par\linenumberpar
    \@ifx{\@par\linenumberpar@LN}{\let\@par\linenumberpar}{}%
    \@ifx{\par\linenumberpar@LN}{\let\par\linenumberpar}{}%
   }{%
    \class@info{Line numbering not turned on yet}%
   }%
%    \end{macrocode}
% Here ends the ``true branch'' of the patch code.
%    \begin{macrocode}
  }{%
%    \end{macrocode}
% If the \filename{lineno.sty} package is loaded, but we fail to patch it, notify the user.
%    \begin{macrocode}
   \class@warn{Failed to recognize lineno.sty procedures; no patches applied. Please get a more up-to-date class.}%
  }%
 }{%
%    \end{macrocode}
% \filename{lineno.sty} is not loaded, so no patches are needed. 
%    \begin{macrocode}
 }%
}%
%    \end{macrocode}
% \end{macro}
%
% \begin{macro}{\linenumberpar}
% \begin{macro}{\@linenumberpar}
% Procedure \cmd\linenumberpar\ takes the place of \cmd\par\ when line numbering is in effect; 
% It executes the \cmd\par\ primitive if we are in vertical mode.
% Otherwise we are in horizontal mode in the MVL and wish to end the current paragraph,
% or we have \cmd\unvbox ed material onto the MVL. 
% 
%    \begin{macrocode}
\def\linenumberpar@ltx{\@ifvmode{\@@@par}{\@linenumberpar}}%
%    \end{macrocode}
%
% Procedure \cmd\@linenumberpar
%    \begin{macrocode}
\def\@linenumberpar{%
%    \end{macrocode}
% Prepare for our trip into the output routine by 
% saving away the current value of \cmd\prevgraf. 
%    \begin{macrocode}
   \linenoprevgraf\prevgraf
%    \end{macrocode}
% The following will be used in the output routine dispatcher
% to sense that we came from here. 
%    \begin{macrocode}
   \set@linepenalties
%    \end{macrocode}
% Finally, call primitive \cmd\par\ with the signal value of \cmd\interlinepenalty\ and friends. 
%    \begin{macrocode}
   \@@@par
%    \end{macrocode}
% We are now in vertical mode. 
% If lines of type were contributed to the MVL (non-trivial paragraph),
% we must force another trip into the output routine to apply
% line numbering to the last line of the paragraph.
%    \begin{macrocode}
   \@ifnum{\prevgraf>\linenoprevgraf}{
    \penalty-\int@postparpenalty
   }{}%
%    \end{macrocode}
% Execute procedure \cmd\@LN@parpgbrk, which has been set up in the 
% output routine for us to invoke here.
% 
%    \begin{macrocode}
   \@LN@parpgbrk
%    \end{macrocode}
% To wrap things up, we restore the original value of \cmd\interlinepenalty\ and friends.
%
% Query: why not employ \TeX's context stack to do the restore?
% Would there be something wrong with executing primitive \cmd\par\ within a group?
%    \begin{macrocode}
   \restore@linepenalties
}%
%    \end{macrocode}
% \end{macro}
% \end{macro}
%
% \begin{macro}{\linenomathWithnumbers}
% \begin{macro}{\linenomathNonumbers}
% Here are the patched definitions for the commands enabling 
% line numbering in display math. 
%    \begin{macrocode}
\newcommand\linenomathWithnumbers@ltx{\@linenomathnumbers@ltx\true@sw}%
\newcommand\linenomathNonumbers@ltx{\@linenomathnumbers@ltx\false@sw}%
%    \end{macrocode}
% \end{macro}
% \end{macro}
%
% \begin{macro}{\@linenomathnumbers}
% \begin{macro}{\endlinenomath}
% We have just begun a display math, and any paragraph we are setting will now end.
% We set all relevant penalties to interrupt values; in the visit to the output routine, 
% we will replace the penalty with its normal value. 
%    \begin{macrocode}
\def\@linenomathnumbers@ltx#1{%
 \@if@sw\ifLineNumbers\fi{%
   \set@linepenalties
   \set@displaypenalties#1%
 }{}%
 \ignorespaces
}%
\def\endlinenomath@ltx{%
 \global\@ignoretrue
}%
%    \end{macrocode}
% \end{macro}
% \end{macro}
%
% We provide a handler for the last line of a paragraph.
%    \begin{macrocode}
\def\handle@par@LN{%
 \Make@LineNo@ltx
%    \end{macrocode}
% After setting the line number, we arrange for an appropriate penalty to 
% be laid down after this visit to the output routine ends.
%
% Query: why not contribute the penalty right here in the visit to the output routine?
%    \begin{macrocode}
 \@tempcnta\lastpenalty 
 \@ifnum{\@tempcnta=\z@}{}{%
  \expandafter\gdef
  \expandafter\@LN@parpgbrk
  \expandafter{% 
   \expandafter\penalty
               \the\@tempcnta
%    \end{macrocode}
% When \cmd\@LN@parpgbrk\ is executed, it resets itself to the default value,
% \cmd\@LN@screenoff@pen. 
%
% Query: \cmd\@LN@screenoff@pen\ appears to try to restore the depth of the last box:
% why is this being done outside the safety of the output routine?
%    \begin{macrocode}
   \global\let\@LN@parpgbrk\@LN@screenoff@pen
  }%
 }%
}%
%    \end{macrocode}
%
% \begin{macro}{\Make@LineNo}
% The procedure \cmd\Make@LineNo\ sets the box containing the line number itself.
%    \begin{macrocode}
\def\Make@LineNo@ltx{%
 \@LN@maybe@normalLineNumber
%    \end{macrocode}
% We measure the depth of \cmd\box\cmd\@cclv\ and unbox it.
% At this point, it is crucial that that box have the same depth as that of the last 
% box within it. 
%
% In the simple case, \cmd\box\cmd\@cclv\ is
% a \cmd\vbox\ containing as its last box the \cmd\hbox\ of the paragraph we are processing.
%
% Query: under what circumstances will this \emph{not} be the case?
%    \begin{macrocode}
 \boxmaxdepth\maxdimen\setbox\z@\vbox{\unvbox\@cclv}%
 \@tempdima\dp\z@
 \unvbox\z@
%    \end{macrocode}
% Then we create the box with the line number, setting its height to zero.
%    \begin{macrocode}
 \sbox\@tempboxa{\hb@xt@\z@{\makeLineNumber}}%
 \ht\@tempboxa\z@
%    \end{macrocode}
% With these preparations, we invoke \cmd\@LN@depthbox, which
% lays that box down (with its depth appropriately set):
% this procedure depends on our having set \cmd\@tempdima\ and \cmd\@tempboxa
% (kinda kludgy way of passing arguments, really).
%    \begin{macrocode}
 \@LN@depthbox
%    \end{macrocode}
% Now increment the line number. I have relocated this token past \cmd\@LN@depthbox: 
% this may induce a bug, but I am going to hereby force the issue: 
% why split up the procedure that lays down boxes with a procedure that sets a register value?
%    \begin{macrocode}
 \stepLineNumber
%    \end{macrocode}
% Finally, execute the \cmd\vadjust s that fell within the line that we just handled. 
%
% Note that \cmd\enqueue@vadjust@ltx\ had queued up all the \cmd\vadjust\ commands for the paragraph
% into \cmd\g@vadjust@queue, laying down an (\cmd\int@vadjustpenalty) interrupt in each ones' place.
% The interrupts associated with this line of the paragraph have now moved the tokens to \cmd\g@vadjust@line,
% which we now expand and execute.
%    \begin{macrocode}
 \g@vadjust@line
 \global\let\g@vadjust@line\@empty
}%
%    \end{macrocode}
% 
%    \begin{macrocode}
\def\write@linelabel#1{% 
 \protected@write\@auxout{}{%
  \string\newlabel{#1}{{\theLineNumber}{\thepage}{}{}{}}%
 }% 
}%
%    \end{macrocode}
% 
%    \begin{macrocode}
\def\ex@vadjust@line{%
 \@if@sw\ifLineNumbers\fi{\enqueue@vadjust@ltx}{\vadjust}%
}%
%    \end{macrocode}
% \end{macro}
%
% Note that the \cmd\linelabel\ commands use a mechanism different from \cmd\vadjust, 
% embodied in the procedure \cmd\enqueue@vadjust@ltx, wherein the 
% \cmd\write\ primitives are enqueued while the paragraph is being processed, each replaced with
% an interrupt, then dequeued and executed by the interrupt handler, leaving a \cmd\write\ node
% in place of the interrupt (just where the \cmd\vadjust 's vertical mode material would had been) 
% just below the box containing the line of type. This \cmd\write, like all whatsits,
% does not affect the depth of \cmd\box\cmd\@cclv, unlike the case of general vertical mode material,
% which could have interfered.
%
% \section{End of the \file{ltxgrid} {\sc docstrip} module}
% Here ends the module.
%    \begin{macrocode}
%</kernel>
%    \end{macrocode}
%
% \Finale
% %Here ends the programmer's documentation.
% \endinput
%
\endinput
%%EOF
